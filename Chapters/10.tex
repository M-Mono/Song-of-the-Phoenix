\fancyhead[LO]{{\scriptsize 【百鸟朝凤】第十章}} %奇數頁眉的左邊
\fancyhead[RO]{\thepage} %奇數頁眉的右邊
\fancyhead[LE]{\thepage} %偶數頁眉的左邊
\fancyhead[RE]{{\scriptsize 【百鸟朝凤】第十章}} %偶數頁眉的右邊
\fancyfoot[LE,RO]{}
\fancyfoot[LO,CE]{}
\fancyfoot[CO,RE]{}
\chapter*{十}
\addcontentsline{toc}{chapter}{\hspace{11mm}第十章}
%\thispagestyle{empty}
游家班到底是哪一年成立的我忘了。那年我好像十九岁,抑或二十岁?我经常在夜晚寻找我的唢呐班子成立时候的一些蛛丝马迹。暗夜里抽丝样出来的那些记忆大抵都和我的唢呐班子无关,倒是一些无关紧要的事件从记忆的缝隙里顽强的冒出来,堵都堵不住。\\

最深刻的当数我的堂妹游秀芝和人私奔。秀芝是我四叔的闺女,一直是个老实的乡下女娃,脸蛋一年四季都红扑扑的。见到生人就红得更厉害了。之前没有一点迹象表明她要离开生她养她的水庄。那个普通的早晨,我的四叔发现他的闺女不见了。一家人慌张的找了一天也没有寻着。后来有人告诉四叔,天麻麻亮看见秀芝和赵水生一起翻过了水庄后面的那座大山。赵水生是水庄赵老把的儿子,刚脱掉开裆裤就和他老子去了远方,听说是个大城市。秀芝读书的时候和他是同桌,受过他不少欺负,我还替秀芝揍过这龟孙子一顿呢!\\

无容置疑的,赵水生拐走了秀芝。\\

四婶哭了好几场,说姓赵的这几天跑过来和秀芝两个躲在屋子里嘀嘀咕咕,感觉就不对头,然后就骂姓赵的,骂完姓赵的又骂自个儿的闺女;四叔则是每日都杀气腾腾的样子,多次表态要活剐了姓赵的。一年后事情才出现好转。秀芝寄回来了一封信,信里说她很好,在深圳的一家皮鞋厂上班,一个月能挣半扇肥猪,还照了照片,照片的背景是一个大水塘,比水庄的水塘可大多了。后来才知道,那不是水塘,是大海。\\

我很奇怪为什么我的记忆里都是和游家班成立无关的事件。为此我陷入了长时间的自责,并试图用记忆来缓解这种不安。可是在梳理属于游家班的丝丝缕缕时,却让我陷入了更大的危机中,因为这些记忆没有一丝亮色,相反,它像一面轰然坍塌的高墙,把我连同我的梦都埋葬掉了。\\

不知道出师四年还是五年后,师傅把他的焦家班交给了我。\\

那天师傅对一屋子的师兄弟们说:从今后,无双镇就没有焦家班了,只有游家班。一屋子的眼睛都在看着我,我很茫然,手足无措。他们的眼神都带着笑,善良而温暖。可我却感到害怕。我不知道我该干什么?能干什么?我只知道今后这一屋子人就要在我稚嫩的翅膀下混生活了。我想起了六七岁放羊的经历,父亲把七八只羊交给我,对我说,给我看好了,丢了一只你就甭想吃饭。我特别害怕山羊漫山遍野散落的情景,总是希望他们紧紧的拢成一团。在路上我就和山羊们商量好了的,可一上了坡它们就没有规矩了,眼里只有茂盛的青草,哪儿草好就往哪儿奔,弄得我眼里尽是颗粒状的白。到回家的时候,这些白就更稀疏了。我那时除了哭真是没其他的好办法的。\\

而此时,那个叫游本盛的男人正挑着一对儿箩筐在水庄的山路上轻快的飞奔。他对遇见的每一个重复着一句话:天鸣接班了,今后无双镇的唢呐就叫游家班了。他说这句话时除了自豪,更有一个伟大的预言家在自己预言降临时的自负。\\

猝然而至的交接像一场成人礼,从那天起,我眼里的水庄褪去了一贯的温润,一草一木都冰冷了,那些整日滑上滑下的石头也变得尖锐而锋利。\\
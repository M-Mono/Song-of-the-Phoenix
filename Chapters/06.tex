\fancyhead[LO]{{\scriptsize 【百鸟朝凤】第六章}} %奇數頁眉的左邊
\fancyhead[RO]{\thepage} %奇數頁眉的右邊
\fancyhead[LE]{\thepage} %偶數頁眉的左邊
\fancyhead[RE]{{\scriptsize 【百鸟朝凤】第六章}} %偶數頁眉的右邊
\fancyfoot[LE,RO]{}
\fancyfoot[LO,CE]{}
\fancyfoot[CO,RE]{}
\chapter*{六}
\addcontentsline{toc}{chapter}{\hspace{11mm}第六章}
%\thispagestyle{empty}
回到土庄我才知道,蓝玉已经把河湾里的水吸上来了。\\

一回来蓝玉就兴冲冲的问我用长芦苇吸上河湾的水用了多久,我掰着指头数了数说一个半月多一点吧。我用了十天。蓝玉骄傲的说。我心里就有些神伤了,说师傅都说了的,你的天分比我好。蓝玉就拍拍我的肩膀,说你也很好的。
\\

但是我发现我真的不好。
\\

蓝玉吸上水后本来也和我下地的,可下地才几天,事情就发生了变化。
\\

我清楚地记得那天有好大好大的雾,气势汹汹的,整个土庄都不见了。我还没起床,就听见蓝玉的尖叫声,我翻了个身,想多睡一阵子。蓝玉总是起的比我早,甚至比师傅师娘还早,为此他还得到了师傅的夸奖。说实话,我也想像他那样起得早的,我也想得到师傅的夸奖的,可我就是起不来,硬着头皮爬起来也是昏昏沉沉的,好一阵子满世界都在乱转。到后来我索性不起来了,夸奖也不想要了,只要让我多睡一会儿就阿弥陀佛了。
\\

起来,快起来,土庄不见了。蓝玉跑进来摇我。
\\

嗯!我咕哝一声,没理会他。
\\

天鸣,土庄没有了。他干脆把我的被窝抱走了。
\\

无奈,我只好起来,走到屋外我才发现土庄真的不见了。
\\

那是我一生中见到的最大的雾,天地都给吃掉了,连站在我面前的蓝玉也消失了。一眼的白,那白还泛着湿。我没有见过有这样气势的大雾,呼吸都不顺畅了。我凑近蓝玉,他正用两只手拼命的捞悬在空中的白,像一只巨大的蜘蛛,被自己拉出来的丝给网住了。
\\

你们两个进来。师傅在里屋喊。
\\

我和蓝玉折进屋,师傅说今天雾大下不了地了,正好我有事情要交代。
\\

师傅从床下拉出一个锈迹斑斑的铁皮箱子,他打开箱子,我和蓝玉都凑过去看,屋子里光线不好,只能看过大概,反正里面都是唢呐,大大小小,长长短短的唢呐。师傅弯下腰不停的翻检着箱子里面的家什,挑啊拣啊,终于,他抽出了一支略短一些的唢呐,把唢呐放进嘴里,唢呐就发出长长的一声――呜。师傅直起腰来,把唢呐递给我身边的蓝玉,说从今天开始你就不用下地了,专心吹唢呐吧,先把它吹响,我就教你基本的调儿。
\\

蓝玉当时的样子我都没法子形容,接过唢呐的那一刻,昏暗的屋子里竟然划过两道亮光,那是蓝玉眼睛里出来的。我看见蓝玉握着唢呐的手在轻轻的抖动,然后他笨拙地把唢呐塞进嘴里,腮帮子一鼓,唢呐就放出来一个闷屁,又一鼓,又出来一个闷屁。
\\

我想师傅接下来该给我派发唢呐了,说不定是支长的呢,比蓝玉的长。我就定定得盯着师傅的手,希望他能抓住一支长的唢呐不放,再放到嘴里试一试,然后递给我。但我是不会像蓝玉那样没有一点定力,当场就放几个闷屁显摆,我会找个没人的地头悄悄放。
\\

师傅是拿出了唢呐,拿出来还不止一支,拿一支出来,他先是吹吹,然后卷起袖口拭擦一番,又放回去,又捡起一支吹拭一番,照例又放回去。我眼珠子都瞪直了,总是希望下一支就是我的,开始看见短的还害怕,怕他递给我,我想要一支比蓝玉长的。可随着箱子里翻剩下的唢呐越来越少,我的心就开始绷紧了,想短的也成,就是拇指长短的我也收。
\\

“砰”的一声,师傅合上了他的箱子。
\\

我没有吹上唢呐。晚上我对蓝玉说我要回家了。蓝玉说你不是刚回过家吗?我说我不想学吹唢呐了。我现在才知道,师傅其实是看不上我的。
\\

土庄的夏天是没有水庄的好看,可土庄的秋天却老有味儿了。土庄的山小是小了些,可山上都有树,种类也繁多,常青的松和落叶的枫抱在一起,夏天还是整齐的绿,到秋天枫树就醉了。就这样,一个一个红绿间杂的山丘一排儿的往远方去了,像一排生动的省略号。我背着行李顺着省略号一直走,边走边哭,我悲伤极了,来土庄都这样老长的日子了,我就是吹不上唢呐,却成了焦家的长工。又想我连唢呐都没有摸过就回到土庄,土庄人肯定要笑我了。还有,我最担心的还是父亲,我这样回去倒不是怕他揍我,我是怕他会活活气死。
\\

我是偷偷走的,从土庄不见了的那天起,我就想走了。昨天晚上,我的师弟蓝玉又爬到我的床上吹了一回唢呐,他吹的时候还拿眼睛瞟着我,眼角得意的往上翘。我知道他是在我面前显摆,可我不恨他,因为要换着我我也是想显摆的。蓝玉的脑袋很大,所以他很聪明,他现在都能把师傅教给他的丧调吹得我眼窝子发潮了。吹到精彩的地方他还会停下来给我讲,这是滑音,这是长调。每天我和师娘下地,他就爬到我干活的地头,猴样的窜上草垛子,呜呜啦啦的就吹开了。回家的路上,我一身的疲惫,连走路都摇晃着,蓝玉却活蹦乱跳,像早晨刚刚抽上露水的青草儿样鲜活。
\\

我走了,谁都不知道我走了。我走的时候蓝玉还抱着他的唢呐在床上说梦话呢。本来我想跟他道个别的,可我又怕他大呼小叫的惊动了师傅师娘。出门我才发现天还没亮,四处都是让人心悸的黑。我摸索着在屋檐下坐下来,坐下来就想在土庄的这些日子,想师傅和师娘。师娘是个好人,像母亲,在地里还不让我多干活,吃饭老往我碗里夹菜。我最不留恋的就是师傅,我还偷偷给他起了外号,叫焦黑炭。焦黑炭没有一点好,整天绷着脸不说,还不让我吹唢呐。想了好多,我的心里五味杂陈,喉咙一硬,就悄悄呜呜的哭起来,一直哭到天色微明,回家的路也能见着了,我才站起来离开,走出一段回头看了看,眼泪又下来了。
\\

终于要离开土庄了,我这辈子怕是当不上唢呐匠了。想起上次回家时给父亲和母亲表的态,说一定学会那首百鸟朝凤回家吹给他们听。但是眼下的情形别说百鸟朝凤了,就是一段稀松的丧调都没有学会。我觉得我最对不起的人就是水庄的游本盛了,他一心一意的送他的儿子学唢呐,可他的儿子学了差不多半年,连用唢呐放两个闷屁的机会都没有,这让水庄人知道了还不笑掉大牙?又伤心了一回,却没有让我放弃回家的念头,反正迟早都是要一无所成的回家的,晚回不如早回,早回还能给家里帮把手。
\\

又看见了水庄,横在天地间,安静得像熟睡的孩子。再拐一个弯,就到我们水庄的地界了。我走的是下坡路,路细而窄,弯弯拐拐,像截扔在山坡上的鸡肠子。路两边有一溜的火棘树,那些枝枝蔓蔓都不安分的往路上凑,这样本就狭窄的小路都快看不见了。
\\

拐过弯,我听见路坎下有说话的声音。踮起脚,我看见老庄叔正领着一群人在他的新房上夯草。干活的人里还有我的父亲,水庄的游本盛。我悄悄的从火棘树下钻过去,把身子隐在草丛里。
\\

天鸣最近没回家?老庄叔问父亲。
\\

吹着呢!好多调调都会了。父亲声音很大。
\\

以前我还没看出天鸣这娃是吹唢呐的料呢!老庄叔又说。
\\

天鸣可比我强,我这娃不要平时看他不吭不响的,做起事情来可一点不含糊。父亲说,前久回来还气粗的给我和他老娘表态,要吹百鸟朝凤呢!
\\

老庄叔就笑一回,他知道父亲是吹牛。就说,百鸟朝凤!百鸟朝凤!我都好多年没听过了,上一次听还是十多年前,火庄的肖大老师去世,焦三爷给吹过一次,那场面,至今还记得,大老师的亲戚学生在院子里跪了黑压压一片,焦三爷坐在棺材前的太师椅上,气定神闲的吹了一场,那个鸟叫声哟!活灵活现的。
\\

等天鸣学回来了,我让他吹给你们听。父亲许愿。
\\

那样我们水庄就长脸了,本盛也长脸了,我就是担心,天鸣有没有那个福气,这百鸟朝凤一代弟子就传一个人呢。老庄叔说。
\\

你们可以不相信天鸣,我是相信我的娃的。父亲说。
\\

我蛇样的从草丛里梭出来,我不想回家了,我想吹唢呐,从来没有像此刻这样想吹唢呐。
\\

我顺着原路爬到山顶,回头看了看水庄。远处近处有袅袅的炊烟,水庄醒过来了。
\\

回到土庄师傅正在院子里磨刀。看见我失魂落魄的站在院子边的土墙下,师傅说:你师娘到地里去了,你也去吧!
\\
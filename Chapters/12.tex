\fancyhead[LO]{{\scriptsize 【百鸟朝凤】第十二章}} %奇數頁眉的左邊
\fancyhead[RO]{\thepage} %奇數頁眉的右邊
\fancyhead[LE]{\thepage} %偶數頁眉的左邊
\fancyhead[RE]{{\scriptsize 【百鸟朝凤】第十二章}} %偶數頁眉的右邊
\fancyfoot[LE,RO]{}
\fancyfoot[LO,CE]{}
\fancyfoot[CO,RE]{}
\chapter*{十二}
\addcontentsline{toc}{chapter}{\hspace{11mm}第十二章}
%\thispagestyle{empty}
稻谷弯腰了,我去看了一回师傅。\\

又见到土庄的秋天了,一马平川的黄一直向天边延伸。\\

师傅刚下地回来。他好像更黑了,也更瘦了,裤管高高的卷起,赤着脚,脚板有韵律的扑打着地面,地面就起来一汪浅浅的尘雾。走到我的面前,他把手里的锄头往地上一拄,下巴挂在锄把的顶端,看着我笑笑,就伸出沾满泥土的手来摸我的脑袋。\\

“看你那双爪爪哟!”师娘嗔怪师傅。师娘也赤着脚,裤管也高高的卷起,正从屋子里往外搬凳子。\\

我把从水庄带来的东西拣出来放到院子里的木桌上。有师傅喜欢的旱烟叶子,烟叶是我到金庄出活时给买的,师傅说过无双镇最好的旱烟叶在金庄;还有腊肉,腊肉是我父亲烘的,颜色和肉质都好,带给师傅的是猪屁股那一段,在乡村人眼里,猪屁股是猪身上最珍贵的部分;此外还有母亲让我捎给师娘的碎花布,让师娘做件秋衣。\\

“来就来,还叮叮当当的带这样一大堆。”师娘总是要客气一番的。\\

我和师傅坐在院子里,这时候夕阳上来了,水庄就晃眼得紧。远处的金黄在晚风中奔腾翻滚,我都看得呆了。师傅指着远处对我说:“看那片,是我的,那谷子,鼓丁饱绽的。”我说我知道的,师傅就哈哈的笑说对对,你在的那阵子下过地的嘛。\\

我给师傅装了一锅刚带来的烟叶,师傅吸了一口,再吸一口,说没买准,金庄最好的烟叶在高昌山下,那片地种出来的烟叶才是最地道的,这烟叶儿不是高昌山下的。\\

“要吃人家饭,最后还要拉屎在人家饭盆里。”一旁剥蒜的师娘给我主持公道。\\

“前几天你二师兄来过一趟,说你们那边乐师钱出得很阔呢!”师傅往地上啐了一口烟痰说。\\

“不多的,就是有钱的那几家大方些!”\\

“人心不足蛇吞象啊!”\\

晚饭时辰,师傅搬出来一土壶烧酒。\\

十年了差不多,师傅一脸兴奋的说,火庄陈家酒坊的,那年给陈家老爷出活的时候到他酒房子里接的,没掺一滴水。\\

师傅在饭桌上照例没话,低着头呼啦啦的吃,间或端着盛酒的碗对我扬扬,这时候我也端起酒碗对着他扬扬,然后就听见烧酒在牙缝里流淌的声音。\\

我在土庄整整呆了三年,没见过师傅喝过一滴酒。其实师傅是有些酒量的,三碗青幽幽的烧酒倒下去,师傅的脸就有了猪肝的颜色。两个眼睛也格外的亮。\\

最让我惊奇的是那天师傅喝完酒后在饭桌上的话,那个多哟!比我在木庄听他说了三年的话还多。那天师傅说一些话让我印象深刻,因为师傅在说这些话的时候就像一只老狼,两手撑着桌面,脸向我这边倾斜着,眼睛里则是血红的光芒。他说唢呐匠眼睛不要只盯着那几张白花花的票子,要盯着手里那杆唢呐;还说唢呐不是吹给别人听的,是吹给自己听的;最后我的师傅焦三爷终于扛不过他珍藏了十年的陈家酒坊的高度烧酒,瘫倒在桌子上了,他倒下去的那一刻,两只眼睛直直的看着说:\\

“有时间去看看你的师弟蓝玉吧!”\\

第二天起来,师傅师娘都不见了,我知道他们下地了。这就是他们的生活,规律得和日出日落一样的。我还是有些晕,走到屋外,院子里木桌上的筲箕里有煮熟的洋芋,这算是给我的早饭了。那些日子就是这样的,我和蓝玉每天早上都要为拿到大个的洋芋争斗一番的。\\

站在山梁上,我回头看了看土庄,它好像老去了不少,那些山,那些水,都似乎泛黄了。\\
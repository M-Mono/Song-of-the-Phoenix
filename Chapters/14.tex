\fancyhead[LO]{{\scriptsize 【百鸟朝凤】第十四章}} %奇數頁眉的左邊
\fancyhead[RO]{\thepage} %奇數頁眉的右邊
\fancyhead[LE]{\thepage} %偶數頁眉的左邊
\fancyhead[RE]{{\scriptsize 【百鸟朝凤】第十四章}} %偶數頁眉的右邊
\fancyfoot[LE,RO]{}
\fancyfoot[LO,CE]{}
\fancyfoot[CO,RE]{}
\chapter*{十四}
\addcontentsline{toc}{chapter}{\hspace{11mm}第十四章}
%\thispagestyle{empty}
老马的四个儿子比想象中的要阔得多。\\

老马要入土的前一天,一辆卡车开进了木庄。\\

老马的四个儿子都到庄头去列队迎接。车上下来几个人,和老马的大儿子聊了几句,老马的大儿子一挥手,庄上一群年轻人就钻进卡车里卸东西。\\

一开始那些东西还是零零碎碎的一堆,让人不知所以,东拼西凑的一倒腾,我身边的师弟蓝玉惊讶的说。\\

“妈的,这是一只乐队!”\\

游家班呈扇形站在马家大院里,我惊奇的发现,我的师兄们集体陷入了某种迷惘。他们的眼神笔直的指向同一个地方,嘴全都大大的裂着,像咫尺有了一个意想不到的惊人变化,也像遥远的天边出现了神奇的海市蜃楼,他们最后都笨拙的完成了复杂情感下简单的语言传递。\\

“到底是搞哪样卵哦!”\\

“这些狗日的是从哪里冒出来的!”\\

“哎呀!”\\

“哦哟!”\\

……\\

天黑下来,落雨了,一开始那雨细微得让人都觉察不到,落到手背上,脸上,有些淡淡的凉意,用手一抹,什么都没有。渐渐地雨就大起来了,雨滴也变大了,砸在裸露的皮肤上还有些疼痛。人群就开始往屋子里、屋檐下和灵堂里拱。\\

城里来的乐队还在雨中忙碌着。二师兄看着雨幕中的几只落汤鸡,说如何不下刀呢?我看了他一眼,他可能意识到这个愿望着实歹毒了些,又讪讪的矫正说下石头也行的。我也赞成下石头,所以我就没有说话了。但很快我发现,下石头恐怕对城里来的乐队也不会有什么实质性的伤害。老马的大儿子很快招呼人在院子里支起了一个帆布帐篷。还满脸堆笑给他们派烟,每个人的两边耳朵上堆满了他还在乐此不疲的派。\\

很快城里来的乐队就准备就绪了。他们的家伙比起乡村八台唢呐要复杂得多。从我见多识广的师弟的介绍我知道了左边那一排鼓叫架子鼓,站着的那个家伙手里抱着的像机枪一样的东西叫电吉他,案板样的是电子琴。最让我惊奇的是右边的络腮胡手里攥着的那支唢呐,他的唢呐好像更长更粗,腰身没有游家班使用的唢呐腰身好,大大咧咧的一粗到底。我就想这样粗的唢呐如何吹呢。\\

“砰!”,弹吉他的用手指拨出了一个清脆的音符。我现在还会在梦里听见那一声响,它的出现让我的梦总是充满了灰色的格调,每一次醒来,我都会双手枕着头想好久,那一声砰为什么在我的梦里不再是乐器的音符,而是极其怪异的幻化成了各式各样断裂发出的声响。譬如我正在建房,砰,房屋的大梁断裂了;或者我刚爬上高大的桑椹树,砰,大树一折为二;又或者我孤独的在一方悬崖下爬行,砰,悬崖张牙舞爪的迎面扑来。\\

……\\

我唯一可以肯定的是,在木庄马家大院的那个夜晚,仿佛从天而降的一声炸裂,搅乱了某种既定的秩序。每个人的心底都有一些莫名的东西在暗暗涌动着,像夜晚厨房木盆里那团搅和完毕的面团,正悄悄的发生着一些不为人知的变化。\\

就在那支吉他发出那声诡异的“砰”的声响的瞬间,我惊异的看见,马家大院所有一切都静止了。洒落的雨滴停在半空,在灯光下有五彩的颜色;洗菜的妇女扔进大木盆的萝卜也滞留在空中,在灯光下有耀眼的白;还有灵堂里的烛光,瞬间就收束成了一团实心的灼热,坚硬如冰;一个正在奔跑的孩子身体前倾,悬停在大门处,手臂一前一后伸展着,像一尊肉铸的雕塑。我张皇地在静止中游走,伸手去碰了一下半空里的水滴,它竟然炸裂成了一团水雾;我绷起指头弹向那团坚实的火焰,哗啦一声,散落了一桌的橘红。\\

我痛苦地捂着脑袋蹲在院子里。\\

“咚”,一声闷响。杂乱的噪音铺天盖地的向我袭来,震得我耳朵发麻。我站起来,发现一切都是活的,一切都在继续。雨一直在下,萝卜翻滚着跌进木盆,烛火在欢快的燃烧,孩子在院子里不停地奔跑。\\

“你刚才看见什么了吗?”我问蓝玉。\\

蓝玉看着我,说:“你是不是丢东西了?”,我摇头。“那你满院子找什么呢?”。蓝玉问。\\
\fancyhead[LO]{{\scriptsize 【百鸟朝凤】第二十章}} %奇數頁眉的左邊
\fancyhead[RO]{\thepage} %奇數頁眉的右邊
\fancyhead[LE]{\thepage} %偶數頁眉的左邊
\fancyhead[RE]{{\scriptsize 【百鸟朝凤】第二十章}} %偶數頁眉的右邊
\fancyfoot[LE,RO]{}
\fancyfoot[LO,CE]{}
\fancyfoot[CO,RE]{}
\chapter*{二十}
\addcontentsline{toc}{chapter}{\hspace{11mm}第二十章}
%\thispagestyle{empty}
父亲病得越来越重了,话也越来越少了,开始是整夜整夜睡不着,后来是睡过去就醒不来。母亲总是守在父亲旁边,隔一阵子就看一回,探探他的鼻孔,摸摸他的额头,怕他睡过去就永远醒不来了。\\

我则在无双镇几个庄子之间昼夜奔走。\\

在无双镇生活了这么多年,我第一次在如此密集的时间里听田间的蛙鸣,山谷的鸟叫。夜晚,我一个人在狭窄的山间小路上行走,天边的一弯冷月漠然地朗照,大地如逝者的巴掌一样冰凉,裹紧衣服才发现,寒冷正不可抗拒地到来。脑子里又浮现出父亲孤独无助的眼神和日渐枯槁的面孔。我怕他等不到我把游家班捏拢他就走了,那样我的父亲就听不到唢呐声了。对于水庄的游本盛来说,没有唢呐的葬礼是不可想象的。\\

无双镇被我的双脚丈量完毕了,我仍像一个出海旬月却两手空空的渔人。我的师兄师弟们,此刻正在繁华而遥远的城市挥汗如雨,他们就像商量好了一般,整整齐齐地离开了生养他们的土地。\\

大师兄还在。他不去城市不是他不想去,而是一次意外让他拥有了一条断腿,而这条腿也成了他和城市之间永远的屏障。我把香烟递到他手上的时候,他还满含神往的给我讲述了师弟蓝玉去年来看他时的情景。“小屁股,抽的烟一支顶你这个一盒,你还别不服气,那烟抽起来就是他奶奶的顺口。”“看来,城里这钱还真他奶奶的好挣。”\\

听完我的来意,大师兄惊奇地盯着我,然后他说,你见过两个人吹的唢呐吗?旧时一般穷苦人家都四台,你想造个两台?埋条死狗还差不多。我说不是埋死狗,是埋我的父亲。大师兄脸上才起来了一层歉意,他大大的吸了一口烟,说去火庄吧,那里起来了好几个班子,听说场面很大,都有十六台了。奶奶的,十六个人一起吹唢呐,怕死人都能给吹活呢!\\

我走了好远,大师兄站在山梁上喊:“去看看吧!如今无双镇的唢呐都成他们的天下了。”\\

我到火庄正赶上这里的唢呐班子出活。\\

确实很让人惊讶。\\

十六个唢呐匠占据了整个院坝,连死者这个理所当然的主角都被逼到了狭窄的一隅。一排条桌浩浩荡荡的拉出了雄壮的架势。条桌上的茶盘里有香烟和瓜子。瓶装的润嗓酒也精神抖擞的站成一列。唢呐匠一色暗红色西服,大宽领,下摆还卷了圆边,一个个像即将走入洞房的新郎。条桌顶头是一件银灰色西服,还扎了根猩红的领带,胸前挂了一块亮闪闪的牌子。看样子,他就该是班主了。\\

最显眼的还不是班主,而是他面前盘子里的一沓钞票,百元面额的,摞出了一道耀眼的风景。“起!”班主发声,接下来就是一场宏大的鼓噪,唢呐太多了,在步调上很难达成一致,于是就出现了群鸟出林的景象,呼啦一片,沸沸扬扬,让人感到一些惶然的惊惧。我甚至满含恶意地发现,有两个年轻的唢呐匠腮帮子从头到尾都瘪着,要知道,这个样子是吹不响唢呐的。这是我见过场面最大的唢呐班子,也是我听过的最难听的唢呐声。我的大师兄说得不对,十六台的唢呐不能把死人吹活,但没准会把活人吹死。\\

我回到家,父亲已经不能说话了,我凑到他的耳朵边说:给你请个火庄的八台吧!父亲忽然睁大眼睛,脑袋拼命地摆动,喉咙里咕咕地响着。我知道,他不要火庄的唢呐,他说过的,火庄那不是真正的唢呐。\\

水庄的游本盛是水庄的河湾开始结冰时离开这个世界的,他静悄悄的就走了,头天晚上还挣扎着吃了半碗稀饭,第二天一早,发现身体都已经变得冰凉了。他死的时候瘦的像个刚出生的婴儿,把一张木床映衬得硕大无比。我把卖牛的钱将父亲安葬了。他的葬礼冷清得如同这个季节,唢呐声自然是没有的,倒是北风从头到尾都在不停地呼啸。\\

那个黄昏,我守在父亲的坟边。从此以后,水庄再没有游本盛了,他和深秋的落叶一起,凄凄惶惶地飘落、腐烂。我在夕阳里想了好久,都没有想起我到底给了我的父亲什么。而我对于他,只有一个又一个的失望。我的唢呐没了,游家班也没了,直到死去,他连一台送葬的唢呐都没有。\\

好久没有看到水庄这样的黄昏了,在我的印象中,水庄的黄昏总是转瞬即逝的,刚发现它,它就一头栽进黑夜。其实心细一点观察,水庄的黄昏是很好看的,落日静止在山头,草的须穗摩挲着它的脸面,有了麻酥酥的微痒;风翻滚着从山梁上滑下来,撩开大山的衣襟,露出暗红的裸背。大地,就在这样简单的组合中,变得古老而温暖。\\

我从怀里抽出唢呐,对着太阳的方向,铜碗里就有了满满的一窝儿夕阳。\\

曲子黏稠地淌出来,打了几个旋儿,跌落在新鲜的坟堆上,它们顺着泥土的缝隙,渗透进了冰冷的黄土。我知道,我的父亲能听见他儿子的唢呐声。从我学艺到他离开这个世界,他还没有听我吹奏过这曲“百鸟朝凤”。开始唢呐声还高亢嘹亮着,渐渐地就低沉了,泪水把曲子染得潮湿而悲伤,低沉婉回的曲子中,我看到父亲站在我的面前,他的眼神如阳光一般温暖,那些已经一去不复返的日子,在朦胧的视线里逐渐清晰起来。\\

起风了,唢呐声愈发凌乱,褪掉了肃穆的色彩,却有了更多的凄凉。我的喉咙被一大团悲伤嗝得生疼,唢呐终于哭了,先是呜咽,继而大恸。连绵不绝的群山,被一杆唢呐搅得撕心裂肺。\\

\fancyhead[LO]{{\scriptsize 【百鸟朝凤】第十九章}} %奇數頁眉的左邊
\fancyhead[RO]{\thepage} %奇數頁眉的右邊
\fancyhead[LE]{\thepage} %偶數頁眉的左邊
\fancyhead[RE]{{\scriptsize 【百鸟朝凤】第十九章}} %偶數頁眉的右邊
\fancyfoot[LE,RO]{}
\fancyfoot[LO,CE]{}
\fancyfoot[CO,RE]{}
\chapter*{十九}
\addcontentsline{toc}{chapter}{\hspace{11mm}第十九章}
%\thispagestyle{empty}
父亲对我的态度是越来越坏了,他看我什么都不顺眼,水缸空了,他骂我眼瞎了,连水缸没水了也看不见;我把水缸挑满了,他还骂我,说我除了挑水还能干啥?\\

父亲骂得对,我都二十六七岁的人了,还窝在家里。你看水庄和我一般年纪的人,娶妻的娶妻,生子的生子,还有大部分早就打点好行装,爬上开往县城、省城的客车走了,除了过年过节能看到他们一两眼,平时像我这样的年轻人村里几乎就看不到了。\\

自从游家班解散后,我再没吹过一天唢呐。\\

游家班的解散没有什么仪式,自自然然的,仿佛空气蒸发了一样,请也没人请了,吹就更没有人吹了。我和大师兄在无双镇的集市上遇到过一次,我们互相问候,还谈了今年庄稼的长势,最后还到无双镇的馆子里喝了一顿烧酒,可谁都没有说关于游家班的事情,哪怕一丁点也没有,像这个班子从来就没有存在过似的。\\

我二十八岁了,水庄的冬天又来了,水庄的冬天如今是越来越随便了,连场像模像样的雪都没有,最近两年更是蹬鼻子上脸,连点缀性的雾凇也看不见了,整个冬天都邋里邋遢,只知道一个劲的落冰雨,钉得人脸手生疼不说,还把一个水庄搅得稀泥遍地。\\

我现在最怕和父亲照面,不光是怕他骂我,是看着他一天天老去的模样我就会内疚。别人的儿子每年都能给家里寄回来数目不等的钱,我却只能坐在家里吃吃喝喝。母亲不像父亲那样责骂我,但她总是一声接着一声的叹气,叹气的声息像一块永远挤不干水的海绵,这比父亲的责骂让我更难受。就这样,我不得不在这个狭窄的空间里逃避。父亲每天吃完饭就去庄上看人打牌去了,他不参与,只是看,其实父亲很想坐上去摸一摸的,可他的口袋不允许。母亲则是每天都在灯下一直坐着忙,忙到实在疲乏得不行了才去睡觉。\\

我每个夜晚都早早爬到床上,却往往到了天亮还没有睡着。\\

今年从稻谷返青开始就没有落过一泼雨。本来都乌云密布了的,天地也陡然黑暗了,眼看一切前奏都摆足了,一庄子人都站在天地间等着瓢泼的雨水了。结果呢,稀稀拉拉的下来几滴,在地上留小几个濡湿的坑点,立马就云开雾绽了。反复几次,水庄人的希望和耐心像田里的稻谷一样,都干枯瘪壳了。\\

父亲的背越来越佝偻,像一张松垮垮的泥弓。父亲每天都守在他的稻田边,脸色和稻子一样枯黄。他的眼神散漫无力地在一坝子干瘪的稻浪上翻滚,跟着风的摆动,晃来荡去,软弱无力。就这样一直到黄昏,他才直起腰来,在一阵吱吱嘎嘎的骨头摩擦声中,开始把枯朽的身躯往自家屋子里搬运。\\

偶尔我会在院子里遇见他,他总是呆呆的看着我,没有了愤怒,也没有了讥讽,目光蛛丝一般的柔软,缠得我有些透不过气来。\\

我清楚地记得,那一季的稻谷最后全枯死在了田里。我站在水庄后面的山头,视野里是一片灼人的枯黄,那黄一直向天边延伸,这样的颜色真让我绝望。但水庄的游本盛更让我绝望。一张脸黄得肆无忌惮。肝癌晚期,我和母亲竭力要求把圈里的老牛卖掉给他治病,可游本盛说:算了,我就是田里的稻子了,再大的雨水也缓不过来了。\\

一个月来,父亲的身体在木床上越来越小。从医院回来,父亲就再没有离开过家里那张宽大的木床。木床是爷爷留下来的,父亲当年就在这张大床上降生,如今,他又即将在这张大床上死去,像完成了一个可笑的轮回。\\

早晨我把家里的老牛牵到水庄的河滩边吃了一些草。中午回家的时候,我居然看见父亲站在庄头,阳光把他捏成一小团,他把身体靠在土坎上,土坎上有茂密的青色,这样他就像一朵从草丛里长出来的黄色蘑菇。我远远就看见了他,惊讶过后眼泪就下来了。\\

我怕他看见我的眼泪,拭干了才走近他。他颤颤巍巍地过来,像刚学走路的小孩儿。拍了拍老牛的脖子,父亲说:“把它卖了吧!”,说完了居然下来了两滴眼泪。我明白了,父亲还不想死,他毕竟才五十出头,这样年纪的水庄人,都身强体健的穿梭于田间地头,还有使不完的劲,眼前的路还远得看不到头呢!“早该卖了,早卖早治的话,也不至于这样了。”我说。\\

牛卖掉那天,我在无双镇给父亲买了一双软底布鞋,我想过了,进城治病难免要走来走去的,软底布鞋穿上不硌脚,父亲全身只剩下骨头了,什么都该是软的才对。\\

晚上回来把鞋子递到父亲手里,他竟然从床上翘起来给了我一耳光。\\

“谁叫你费这钱?狗日的就是手散!”\\

耳光一点不响亮,听见的反而是骨头炸裂的声音。\\

我没有说话,把父亲扶下躺好,他两个鼻孔和嘴都大口大口的呼着浊气。喘了好一阵子,父亲终于平静了下来,他先是长长的吁了一口气,艰难地把身体侧过来对着我说:“天鸣,我听说金庄的唢呐也吹起来了。”我点点头。\\

其实不光金庄,无双镇除了水庄其他几个庄子都有唢呐了。也不知道是从哪天开始,城里下来的乐队就从无双镇消失了,就像停留在河滩上的一团雾,一阵风过,就无影无踪了。乐队一消失,唢呐声就嘹亮起来了。\\

“把游家班捏拢来。”父亲说:“无双镇不能没有唢呐。”\\

“有哩!除了水庄其他庄子都有了。”我说。\\

“日娘,那叫啥子唢呐哟!”父亲面色灰土,喘气声也大了许多,额头上还有汗出来。\\

我呆坐在床边,不说话。父亲的喉咙里有咕咕的声音,像地下的暗河,涌动着不为人知的秘密。良久,我听见父亲发出呜呜的哭声,哭声尖而细,如同一柄锋利的尖刀,划过屋子里凝滞的气息,继而如撕裂的布匹,陡然凄厉得紧。\\

此刻我才发现,我的父亲,水庄的游本盛心里一直都希望他的儿子吹唢呐的。在游家班解散后,父亲那种看似寡毒的蔑视、打击、嘲讽,其实是伤心欲绝,是理想被终结后的破罐子破摔。我又想起了父亲带着我拜师的那个湿漉漉的日子,还有他跌倒后爬起来脸上那道殷红的血痕。\\

我伸出手,摸到了父亲夸张的锁骨,它坚硬地硌着我的手,更硌着我的心。\\

我试试吧。我说,声音很小,但父亲还是听见了。\\

尽管屋子里光线很暗,但我还是看见了父亲眼里的亮光,我的话像一根划燃的火柴,腾地点亮了父亲这盏即将油尽的枯灯。\\

“我就知道,你狗日的还想着唢呐。”笑容在父亲枯瘦狭窄的面容上铺开,氲成一团凄苦和苍凉。“知道我为什么卖牛吗?”父亲纯真得像一个孩子:“我那是给游家班买家什用的,我想过了,啥子鼓啊锣啊,都老旧了,该换新的了。”接下来就是一阵咳嗽,父亲太兴奋了,又呼啸了一阵才平静了下来,父亲又说:“我死了,给我吹个四台就行了。”\\

“我给你吹‘百鸟朝凤’。”我说。\\

父亲摆了摆枯瘦的手,半天才说:“使不得,我不配!”\\
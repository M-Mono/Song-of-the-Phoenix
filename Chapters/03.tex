\fancyhead[LO]{{\scriptsize 【百鸟朝凤】第三章}} %奇數頁眉的左邊
\fancyhead[RO]{\thepage} %奇數頁眉的右邊
\fancyhead[LE]{\thepage} %偶數頁眉的左邊
\fancyhead[RE]{{\scriptsize 【百鸟朝凤】第三章}} %偶數頁眉的右邊
\fancyfoot[LE,RO]{}
\fancyfoot[LO,CE]{}
\fancyfoot[CO,RE]{}
\chapter*{三}
\addcontentsline{toc}{chapter}{\hspace{11mm}第三章}
%\thispagestyle{empty}
好多年后师傅对我说,你知道当初我为什么收你为徒吗?我说你老人家心善,怕我父亲把我给活活打死了。师傅摇头,说你错了,我收你为徒是因为你的眼泪。我说什么眼泪?师傅说你父亲跌倒后你扶起他后掉的那滴眼泪。
\\

父亲走了,看着他离开的背影我顿时有一种无助的感觉,以往天天看见他,没觉得他有多重要,被他揍了还会在心里偷偷骂“狗日的游本盛”。现在才发现父亲原来是极重要的。他就像一棵树,可以挡风遮雨,等有一天自己离开了这棵大树,才发现雨淋在身上是冰湿的,太阳晒在脸上是烤人的。\\

从此以后,我就是一个人了。看着父亲渐渐变淡变小的背影,我忍不住哭了一场,师傅站在我旁边,伸出一只手搭在我的肩上,轻轻拍了拍,我心里一热,哭得更厉害了。
\\

晚上吃饭,师傅给我介绍了师娘,师娘很瘦,也黑。走起路来左摇右晃的,像根煮熟的荞麦面条。师娘话多,饭桌上问了我好多事情,都是关于水庄的,还说她有个亲戚就住在我们水庄。和师娘比起来,师傅的话则少了许多,一顿饭时间就说了两句话,我端碗的时候他说:吃饭。我放碗的时候他又说:吃饱。
\\

吃完饭,我主动把碗刷了。在刷碗的过程中我偷偷探头看了看坐在堂屋里的师傅和师娘,当时师娘对着我站的位置指指点点,还不住的点头,脸上也有些不易觉察的笑容。师傅却不为所动,他只是一个劲的抽烟,喷出来的烟雾也浓,让我想起在水庄和父亲烧山灰的日子。我明白师娘的笑容和我刷碗的行动有关。而我刷碗的行动又和临出门那晚母亲油灯下的唠叨有关。母亲说:出门在外不比在家,要勤快,眼要尖,要把你那根全是懒肉的尾巴夹好。
\\

刷完碗师娘对我说,她的三个儿子都成家分出去了,家里就他们两老,所以你该做些力所能及的事情。
\\

晚上我躺在床上,想明天就要吹上唢呐了,有一些兴奋,又有一些惶恐,总觉得我的人生不该就这样拐弯的,我还没有玩够,我还是个娃儿,娃儿就该玩的。想起我的伙伴马儿他们,此刻他们肯定正在水庄的木桥边抓萤火虫,把抓来的萤火虫放进透明的瓶子里,走夜路时可以当马灯用。
\\

一早,我还在梦里捉萤火虫,就听见了两声剧烈的咳嗽声,咳嗽声是师傅发出来的,我一惊,知道这是起床的信号,师傅毕竟不是亲爹,没有像父亲一样冲进来掀开被窝照着屁股就一顿猛扇。我想他一定还当我是客人,所以方式也就间接一些。穿上衣服走出门,我先喊了一声站在屋檐下的师娘,正在淘蚕豆的师娘对我点了点头。打完一个呵欠我才发现太阳还在山那头浴血挣扎,我心里头就上来了一些怨气,想这太阳都还没有出来呢,就得爬起来。在家虽然被父亲扇屁股,但那时太阳都老高了啊。看见我脸嘴不好看,师娘说你师傅到河湾去了,你也去吧!
\\

顺着师娘指的方向,我看见了木庄的河湾,木庄虽然叫木庄,可河湾却比水庄的还要大,河岸四周有烟柳,烟柳我们水庄也有,远远的看去像团滚圆的烟。烟柳四四方方的抱着一团翠绿的河湾,几只纯白的水鹤在河湾上悠闲的飞来绕去。师傅站在河滩上,静静的看着水面,他的身影很孤寂,也渺小。
\\

师傅从河岸边齐根折来一根芦苇,去掉顶端的芦苇须,把足有三尺长的芦苇杆递给我,说过去把河里的水吸上来,记住,芦苇杆只能将将伸到水面。开始我以为这是件极简单的事情,一吸我才知道没有那么简单。我脸也红了,腿也软了,小肚子都抽筋了,还是没能吸上一滴水。我回头看了看师傅,师傅脸色灰暗,说等你把水吸上来了就可以回家了。
\\

天黑尽了我才回到师傅家,师傅和师娘守着一盏如豆的油灯。看我进屋来,师娘端给我一碗饭,饭还没到我手里,师傅说话了。
\\

水吸上来了?
\\

我摇摇头。
\\

那你回来搓球啊?师傅猛地立起来,把手里的旱烟杆往地上狠狠的一掼。他的脸本来就乌黑,此刻就更黑了。
\\

我现在才意识到这个黑脸男人是认真的。
\\

我的晚饭被师傅扒掉了半碗,虽然师娘一直给我说情,说天鸣他爹可是交足了生活费用的,再说娃儿在吃长饭呢!
\\

娃?老子哪个徒弟不是娃过来的?老子当初拜师的时候,三天没有饭吃呢!
\\

夜晚我躺在床上痛快的哭了一回,哭完了就想父亲的绝情,想完父亲的绝情又想母亲的好。想着想着就睡着了,睡着好像没多久又听见了咳嗽声。我爬起来凑到窗户边,发现山那边连太阳浴血的迹象都还没有。
\\

此后十多天,我天天攥着根芦苇杆在河滩上吸水。有往来的土庄人隔得远远的就喊,焦三爷又收新徒弟了。还有的喊,这个娃子能成焦三爷的弟子,看来是有些能耐的。我听见他们的喊声里有酸溜溜的味道,肯定是自己的娃没能让师傅看上。这样我有了一些信心,就把吸水这个世间最枯燥的活儿有模有样的干起来。
\\

大约是一个黄昏,我记得那天河滩上的水鹤特别多,沿着水面低低的滑翔,在一片耀眼的绿中拉出一尾又一尾炫目的雪白。我像之前千百次的吸水一样,一沉腰,一顿足,一提气,竟然牢牢的咬住了一股冰凉。我把嘴里的水来回渡了渡,又把它轻轻的吐到掌心里,不错的,我把水吸上来了。看着掌心的一窝清澈,我恍若隔世,一股说不清道不明的东西在心窝子里上下翻滚,喉咙慢慢就变得硬硬的了。我撒腿疯了似的向师傅的土墙小屋子跑去,跑到院子里,师傅正坐在屋檐下编苇席。
\\

吸上来了。我一字一顿的说。
\\

本来以为师傅会笑一个,然后点点头,说这下你可以吹上唢呐了。但不是这样的。师傅听我说完,从脚边堆积的芦苇里挑出一根最长的,掐头去尾递给我。我把芦苇杆立起来,比我还要高,我疑惑地看着师傅,师傅依然认真地低头编着苇席,半晌才抬起头对我说,去啊!继续吸。
\\

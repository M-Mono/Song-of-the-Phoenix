\fancyhead[LO]{{\scriptsize 【百鸟朝凤】第一章}} %奇數頁眉的左邊
\fancyhead[RO]{\thepage} %奇數頁眉的右邊
\fancyhead[LE]{\thepage} %偶數頁眉的左邊
\fancyhead[RE]{{\scriptsize 【百鸟朝凤】第一章}} %偶數頁眉的右邊
\fancyfoot[LE,RO]{}
\fancyfoot[LO,CE]{}
\fancyfoot[CO,RE]{}
\chapter*{一}
\addcontentsline{toc}{chapter}{\hspace{11mm}第一章}
%\thispagestyle{empty}
过了河,父亲再一次告诫我,说不管师傅问什么,都要顺着他,知道吗?我点点头。父亲蹲下来给我整了整衣衫,我的对襟短衫是母亲两个月前就做好的,为了让我穿上去看起来老成一些,还特地选了藏青色。直到今天离开家时,母亲才把新衣服给我换上。衣服上身后,父亲不满意,蹙着眉说还是没盖住那股子嫩臭味儿。看起来藏青色的短衫并没有拉长我来到这个世界上的日子。毕竟我才十一岁,这个年龄不比衣服,过过水就能缩短或抻长的。
\\

一大早被母亲从床上掀下来的时候,还看见她一脸的怒气,她对我睡懒觉的习惯深恶痛绝。可临了出门,母亲的眼神里却布满了希冀、不舍,还有无奈。父亲则决绝得多,他的理想就是让我做个唢呐匠。我们水庄是没有唢呐匠的,遇上红白喜事,都要从外庄请,从外庄请也不是容易的事情,如果恰好遇上人家有预约,那水庄的红白喜事就冷清了。没有了那股子活泛劲头,主人面子上过不去,客人也会觉得少了点什么。所以被请来的唢呐匠在水庄都会得到极好的礼遇,烟酒茶是一刻不能断的,还得开小灶。离开那天,主人会把请来的唢呐匠送出二里多地,临别了还会奉上一点乐师钱,数量不多,但那是主人的心意。推辞一番是难免的,但最后还是要收下的。大家都明白这是规矩,给钱是规矩,收钱是规矩,连推辞都是规矩的一部分。
\\

听母亲说,父亲想让我做一名唢呐匠其实并不完全为了钱。母亲说父亲年轻时也想做一名唢呐匠,可拜了好多个师傅,人家就不收,把方圆百里的唢呐匠师傅都拜遍了,父亲还是没有吹上一天的唢呐,人家师父说了,父亲这人鬼精鬼精的,不是吹唢呐的料。许多年过去了,本以为时间已经让父亲的理想早就像深秋的落叶腐化成泥了,可事实并不是这样。自我懂事起,我就发现父亲看我的眼神变得怪怪的,像蹲在狗肉汤锅边的饿痨子,摩拳擦掌,跃跃欲试。有一次,我的老师在水庄的木桥上遇见了父亲和我,他情绪激动地给父亲反映,说我从小学一年级到五年级,数学考试从来没有超过三十分。我当时就羞愧地低下了头,想接下来理所当然的有一场暴风骤雨。老师说完了,父亲点点头,很大度的挥挥手说三十分已经不错了。然后牵起我走了。走到桥下,他回头看了一眼身后可怜的一头雾水的教书匠,嘿嘿干笑了两声,教书先生哪里知道,水庄的游本盛对他儿子有更高远的打算。
\\

我确实不喜欢念书,我们水庄大部分娃子和我一样不喜欢念书,刚开始还行,渐渐的就冷了。主要是听不懂,比如我们的数学老师,自己都没有一个准,今天给我们一个答案,明天一早站在教室里又小声的宣布,说同学们昨天我回去在火塘边想了一宿,觉得昨天那个题目的答案有鬼,不正确,所以吓得一夜都没睡安稳,今天特地给大家纠正。我们就笑一回,后来又听说数学老师其实也只是个小学毕业的,更有甚者说他根本连小学都没有读毕业。我们就无可奈何的生出一些鄙夷来。鄙夷的方式就是不上课,漫山遍野的去疯。
\\

我不喜欢念书,可我也不喜欢做唢呐匠,我也说不清为什么不喜欢作唢呐匠,可能是从小到大总听见父亲在耳边灌输唢呐匠的种种好,听得多了,也腻了,就厌恶了。而且我断定,我的父亲之所以希望我成为一个吹唢呐的,目的就是图那几个乐师钱。
\\
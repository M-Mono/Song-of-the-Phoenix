\fancyhead[LO]{{\scriptsize 【百鸟朝凤】第五章}} %奇數頁眉的左邊
\fancyhead[RO]{\thepage} %奇數頁眉的右邊
\fancyhead[LE]{\thepage} %偶數頁眉的左邊
\fancyhead[RE]{{\scriptsize 【百鸟朝凤】第五章}} %偶數頁眉的右邊
\fancyfoot[LE,RO]{}
\fancyfoot[LO,CE]{}
\fancyfoot[CO,RE]{}
\chapter*{五}
\addcontentsline{toc}{chapter}{\hspace{11mm}第五章}
%\thispagestyle{empty}
三个月了,我用一人多高的芦苇杆把河湾的水吸了上来。可我还是没有吹上唢呐。师傅只是让我和师娘下地给玉米除草。土庄六月的天气似乎比水庄的要热得多,我们水庄这个季节都是湿漉漉的。在玉米地里,我对师娘说土庄不如水庄好,我们水庄没有这样热,师娘就哈哈的笑,笑完了说游家娃是想家了。中午收工回家,经过河湾的时候,我的师弟蓝玉扎着马步在河湾上吸水。蓝玉是有天分的,他才来一个月,就接到师傅递给他的一人多高的芦苇杆了。我到这一步比蓝玉整整多用了一个月时间。
\\

吃完晚饭,蓝玉去刷碗,自从他来了以后,刷碗这个活就是他的了。刚开始我还觉得好,想终于可以不用刷碗了。可没过两天师傅对我说,跟你师娘下地吧。才下了半天的地,我又想念刷碗了。蓝玉刷碗的声音特别响,刷碗这活我是知道的,磕磕碰碰发出些声响是难免的,但绝没有这样大的声响的。连提个水壶,蓝玉都要弄得惊天动地的,一弓腰,就发出咳的一大声,仿佛他提起来的不是一个水壶,而是一扇石磨。很快,蓝玉就从厨房出来了,他甩了甩两只湿漉漉的手,眼睛看着师傅和师娘,他的意思是告诉我们,该他的活已经干完了。
\\

蓝玉得到了师娘的夸奖,师娘说蓝玉刷碗动作比天鸣麻利,顿了顿师娘又说,麻利是麻利,但没有天鸣刷的干净。
\\

蓝玉不仅话多,也会讲。他坐在师傅和师娘的中间给他们讲他们木庄的奇怪事,师娘被他逗得哈哈大笑,连师傅一直绷着的脸都会不时舒展开来。我没有蓝玉的嘴皮子,就在旁边一直闷坐着,师娘好像看出来了,就对我说,天鸣是不是想家了,想家的话就回去看看吧。他说这话的时候眼睛一直盯着师傅,我想是这个事情她做不了主,在征求师傅的意见。一提到回家,我的眼窝就一阵发热,我真想家了,想父母,还有两个妹妹,他们肯定也在想着我的。
\\

我目不转睛的看着师傅,老半天师傅才说,早去早回。
\\

我又回到水庄了。
\\

以前觉得水庄什么都不好,一脚踏进水庄的地界,我发现水庄什么都好,水庄的山比土庄的高,水比土庄的绿,连人都比土庄的耐看呢。
\\

走进我家院子,母亲正蹲在屋檐下剁猪草,父亲站在楼梯上给房顶夯草。一看见我,母亲就扔掉手里的活跑过来,她摸摸我的头,又摸摸我的脸,说天鸣回来了,还瘦了。母亲的手有一股青草的腥味,但我觉得特别好闻,我好久没有看见母亲的脸了,好像黑了不少,看着母亲,我的眼睛就模糊起来。
\\

本盛,天鸣回来了。母亲对着父亲喊。
\\

父亲没有从楼梯上下来,他弯下腰看看我,又继续给屋顶夯草。
\\

好好的,回来做啥?父亲的声音顺着楼梯滑下来。
\\

师傅让我回来的。我直着脖子说。
\\

啥?你个狗日的,烂泥糊不上墙。父亲把夯草的木片子高高的摔下来,破成了好几块。
\\

娃好好的,你骂他干啥?母亲说。
\\

好好的?好好的能让师傅赶回家?父亲从楼梯上下来,还腾出一只手狠狠的对着我戳。你啊,你啊,你——。父亲发出的声音像被他嚼碎了吐出来的。
\\

晚上母亲给我做了一顿腊肉,还不让两个妹妹多吃,拼命把好吃的往我碗里夹。父亲在饭桌上不停的对我翻白眼,像要活吞了我似的。什么时候回去?母亲把碗里最后一片腊肉夹给我问。早去早回,师傅说的。我说。真的?父亲把头歪过来问,我点点头。这时候水庄的游本盛才笑了,还用筷子敲了敲我的后脑勺,轻轻的。我发现,这顿饭父亲的筷子一直没有伸到肉碗里,我把母亲给我的最后一片腊肉夹起来放进了父亲的碗里,父亲笑得更欢了,说那就恭敬不如从命了。
\\

月亮上来了,两个妹妹都睡了。我和父亲母亲坐在院子里,我给他们讲了木庄的好多事情。
\\

爸,你知道唢呐除了四台和八台,还有什么吗?我问父亲。
\\

父亲笑了笑,然后看了看母亲,母亲也笑了笑。
\\

莫非还有十六台?母亲说。
\\

我摇摇头。说唢呐吹到顶其实是独奏呢!你们知道叫什么吗?
\\

这时候我看见父亲的笑容不见了,他的目光跑到月亮上去了,面容也变得复杂了。好半天他才把目光转向我,说你知道我为什么要送你去学吹唢呐吗?\\

我摇头。
\\

就是要你学会吹百鸟朝凤。
\\

我惊讶了,就兴奋的说原来你也知道百鸟朝凤的啊!还表态说你们放心,我学会了回来吹给你们听。
\\

没有那样简单,你师傅这十多年来收了不下二十个徒弟,可没有一个学会百鸟朝凤的。父亲说。
\\

很难学吗?我问。
\\

倒不是,这个曲子是唢呐人的看家本领,一代弟子只传授一个人,这个人必须是天赋高,德行好的,学会了这个曲子,那是十分荣耀的事情,这个曲子只在白事上用,受用的人也要口碑极好才行,否则是不配享用这个曲子的。
\\

咱家天鸣能学会吗?母亲问。
\\

父亲摇摇头,走了。院子里只剩下母亲和我,还有天上的一轮残月。\\
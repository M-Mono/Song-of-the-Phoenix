\fancyhead[LO]{{\scriptsize 【百鸟朝凤】第九章}} %奇數頁眉的左邊
\fancyhead[RO]{\thepage} %奇數頁眉的右邊
\fancyhead[LE]{\thepage} %偶數頁眉的左邊
\fancyhead[RE]{{\scriptsize 【百鸟朝凤】第九章}} %偶數頁眉的右邊
\fancyfoot[LE,RO]{}
\fancyfoot[LO,CE]{}
\fancyfoot[CO,RE]{}
\chapter*{九}
\addcontentsline{toc}{chapter}{\hspace{11mm}第九章}
%\thispagestyle{empty}
蓝玉走了,披着一身绚烂的朝霞,向着太阳升起的地方去了。我站在土庄的土堡上,看着他的身影逐渐变小变淡。太阳明天还是要升起的,可我却见不到我的师弟蓝玉了。蓝玉在我的生命里出现和消逝都突然得紧,仿佛那个落雨的日子,蓝玉就该出现在我的面前,又仿佛这个炫目的黄昏,他本就一定要离去。\\

昨晚的晚饭很丰盛,有师娘做得最好的土豆汤,师娘做土豆汤是要放番茄的,番茄在无双镇不叫番茄,叫毛辣角,毛辣角又是土庄特有的小个毛辣角,樱桃样。师娘把剁碎的毛辣角和土豆搅拌在一起,还放了半勺猪油,颜色血红,喝起来酸酸的,很开胃;另外,还有蓝玉最喜欢的灰灰菜,灰灰菜是凉拌的。我在水庄没有见到过这种野菜,蓝玉说他们火庄也没有。嫩嫩的灰灰菜在水里飞快的跑过一趟,晾干后凉拌,居然有鲜肉的味道。\\

饭桌上师娘不停地往蓝玉的碗里夹菜,一盘灰灰菜差不多都到蓝玉碗里了。蓝玉很得意,不停的对我撇嘴,还故意砸吧出嘹亮的声音。师傅吃饭是没有响动的,他每一个动作都很小心,在饭桌上你都感觉不到他的存在。直到他把一筷子灰灰菜夹到蓝玉的碗里,我才发现师傅一直都在饭桌上的。师傅的这个动作让我和蓝玉的嘴合不上了。要知道,焦家班的掌门人没有给人夹菜的习惯。他总是静悄悄的在饭桌上干他该干的事情,不要说夹菜,就是话也极少说的,有客人他也只是两句话,开饭时说吃饭,客人放碗时说吃饱。师傅看见了我和蓝玉的惊讶,就对蓝玉说,多吃点,这种灰灰菜只有土庄才有的。\\

我忽然有了一种不祥的预感。这种预感在晚饭后终于得到了证实。\\

师傅照例在油灯下吸烟,蓝玉就坐在他的面前。\\

“睡觉前把东西归置归置,明天一早就回去吧!”师傅对蓝玉说。\\

蓝玉低着头抠指甲,不说话。\\

“差不多了,红白喜事都能拿下来的。”师傅又说。\\

“师傅,是我哪里没有做好吗?”蓝玉问。\\

“你做得很好了,你是我徒弟中悟性最好的一个。”\\

“那你为什么要赶我走?”蓝玉终于哭了。\\

“你我的缘分就只能到这里了!”师傅叹了口气说。\\

“蓝玉不要哭,没事就到土庄来,师娘给你做灰灰菜吃。”师娘也有了一窝子眼泪。\\

“我吹得比天鸣都好,天鸣能学百鸟朝凤,我为什么不能?”蓝玉咬着牙说。他力气太大了,把左手的中指都抠出血来了。\\

师傅眼睛一亮,忽然又暗淡下去了。他站起来拍了拍屁股,烟袋悬在嘴上,背着两只手离开了,走到门边才把烟袋从嘴里拿出来,回过头说睡吧,明天还有事情干呢!这话听上去是对师娘说的,又好像是对屋子里所有的人说的。\\

睡在床上,我有很多的话想对蓝玉说,可有不知道说什么好。一直到天亮,我们谁都没有说一句话。焦家班的传声仪式结束后,蓝玉很是难过了一阵子。没多久他就缓过来了,他对我说,只要还留在师傅身边,他就一定能吹上百鸟朝凤。我是相信蓝玉的,我知道师傅传我百鸟朝凤是因为我老实,不传给蓝玉是觉得蓝玉花花肠子多。其实师傅是不对的,蓝玉天分比我好,他确实是比我精灵了一些,可人精灵点有什么不好的呢?我打心眼里希望师傅能把百鸟朝凤传给蓝玉,我也这样对蓝玉说过,可蓝玉不领情,还说我挤兑他呢!\\

现在师傅要让蓝玉走了。我的师弟最后的希望也就没有了。\\

蓝玉走的时候就是寻不见师傅。蓝玉在屋子里找了一圈也没寻着,师娘说定是下地去了。蓝玉就在院子里给师娘磕了六个头,说师娘我给你磕六个吧,你和师傅各自三个,我一并磕了。师娘把蓝玉扶起来,眼泪就哗哗的下来了。蓝玉走了,背着一个包袱,狠狠的转了一个身,留给我一个瘦削的背影。\\

蓝玉不见了,师傅从屋子后面的草垛子后转了出来。我回头看见了他,他对我说,从今天开始,我教你百鸟朝凤吧。\\

\fancyhead[LO]{{\scriptsize 【百鸟朝凤】第二章}} %奇數頁眉的左邊
\fancyhead[RO]{\thepage} %奇數頁眉的右邊
\fancyhead[LE]{\thepage} %偶數頁眉的左邊
\fancyhead[RE]{{\scriptsize 【百鸟朝凤】第二章}} %偶數頁眉的右邊
\fancyfoot[LE,RO]{}
\fancyfoot[LO,CE]{}
\fancyfoot[CO,RE]{}
\chapter*{二}
\addcontentsline{toc}{chapter}{\hspace{11mm}第二章}
%\thispagestyle{empty}
翻过大阴山,就能看见土庄了。那就是我未曾谋面的师傅的家。我们这一带有五个庄子,分别叫金庄、木庄、火庄,土庄,再加上我们水庄,构成了一个大镇,按理这个镇子该叫五行镇才对的,可它却叫无双镇。未来师傅的宅子在一片茂盛的竹林中,翠绿掩映下的一栋土墙房。我曾经从爷爷的旧箱子里翻出一本绣像《三国演义》,里面有一幅画,叫三顾茅庐的,眼前的这个场景就和那幅画差不多。通往土墙房的路一溜的坦途,可父亲却发出吭哧吭哧的喘气声,他额头上还有针尖大小的汗珠儿,两个拳头紧紧的握着。我看了他一眼,父亲有些不好意思起来,他想我定是把他的紧张看破了,于是他就露出一个自嘲的讪笑。\\

面子有些挂不住的父亲就转移话题。福地啊!父亲说,你看,左青龙,右白虎,后朱雀,前玄武,一看就不是一般人家。我想笑,可没敢笑出来,父亲是不识风水的,连引述有关风水的俗语都弄错了。这几句我也是听水庄的风水先生说过,不过人家说的是前朱雀,后玄武。我想父亲真的是太紧张了,他怕自己小时候的悲剧在下一代的身上重演。我顿时有了一些报复的快感,想师傅要是看不上我就好了,最好是出门了,还是远门,一年半年的都回不来。
\\

看见我左摇右晃的二流子步伐,父亲在身后焦急的吼,天杀的,你有点正形好不好!师傅看见了那还了得。
\\

父亲的运气比想象的要好,木庄名声最显赫的唢呐匠今天正好在家。
\\

我未来师傅的面皮很黑,又穿了一件黑袍子,这样就成了一截成色上好的木炭。他从屋子里踱出来的时候燃了一袋旱烟,烟火吱吱的乱炸。我很紧张,怕那点星火把他自己给点燃了。他大约是看出了我的焦虑,就抬起一条腿,架到另一条腿的膝盖上,把鞋底对着天空,将那半锅子剩烟杵灭了。做这样一个难度很大的动作只是为了杵灭一锅烟火,看来我未来的师傅真是一个不简单的人。
\\

焦师傅,我叫游本盛,这是我儿子游天鸣,打鸣的鸣,不是明白的明。父亲弓着腰,踩着碎步向屋檐下的黑脸汉子跑过去,跑的过程中又慌不迭的伸手到口袋里摸香烟,眼睛还一直对着一张黑脸行注目礼。可怜的父亲在六七步路的距离里想干的事情太多了,他又缺乏应有的镇定,这样先是左脚和右脚打了架,接着身体就笔直的向前仆倒,跌了一嘴的泥,香烟也脱手飞了出去,不偏不倚的降落在院子边的一个水坑里。我的心一紧,赶忙过去把父亲扶起来,父亲甩开我扶他的手,说扶我干什么?快去给师傅磕头啊!我没有听父亲的,毕竟我认识父亲的时间比认识师傅的时间要长,于情于理都该照看刚从地上爬起来的水庄汉子。主意打定,我仍然不屈不挠的挽着父亲的手臂,我抬起头,父亲的额头上有新鲜的创口,殷红的血珠正争先恐后的滲出来,我一阵心酸,眼泪就下来了。
\\

师傅摆摆手,说磕头?磕什么头?他为什么要给我磕头?这个头不是谁都能磕的。
\\

父亲哑然,很难堪的从水坑里捡起香烟,抽出一支来,香烟身体暴涨,还湿嗒嗒的落着泪。
\\

这?父亲伸出捏着香烟的手为难地说。
\\

屋檐下的扬了扬手里的烟锅子说,我抽这个。
\\

我、父亲,还有我未来的黑脸师傅,三个人就僵立着,谁都不说话,主要是不知道说什么。还是屋檐下的木炭坦然,不管怎么说这始终是他的地盘,所以他的面目始终都处于一种松弛的状态,他看了看天空,我也看了看天空,他肯定觉得今天是个好天气,我也觉得今天是个好天气。太阳像个刚煎好的鸡蛋,有些耀眼,我未来的师傅就用手做了一个凉棚,看了一会儿太阳,又缓慢地填了一锅烟,把烟点燃后,他终于开口了。
\\

哪个庄子的?他问话的时候既不看我,也不看父亲,但父亲对他的傲慢却欣喜如狂。父亲往前走了两步,说水庄的,是游叔华介绍过来的。父亲把游叔华三个字做了相当夸张的重音处理。游叔华是我的堂伯,同时也是我们水庄的村长。
\\

我听见唢呐匠的鼻子里有一声细微的响动,像鼻腔里爬出来一个毛毛虫。他继续低头吸烟,仿佛没有听见父亲的话。看见游村长的名号没有收到想象中的震撼力,父亲就沮丧了。
\\

多大了?唢呐匠又问。
\\

我的嘴唇动了动,刚想开口,父亲的声音就响箭般的激射过来:十三岁。比我准备说的多出了两岁。怕唢呐匠不相信,父亲还做了补充:这个月十一就十三岁满满的了。
\\

唢呐匠的规矩你是知道的,十三是个坎。唢呐匠说。
\\

知道知道。父亲答。
\\

这娃看起来不像十三的啊。唢呐匠的眼睛很厉害。
\\

这狗东西是个娃娃脸,自十岁过来就这样儿,不见熟。
\\

嗯!唢呐匠点了点头。看见唢呐匠表了态,父亲的眉毛骤然上扬,他跑到屋檐下战战抖抖的问:您老答应了?
\\

哼!还早着呢!
\\

我原本以为做个唢呐匠是件很容易的事情,拜个师,学两段调儿,就算成了,可照眼下的情形来看,道道还真不少呢。
\\

院子里摆了一张桌子,桌子上放了一个盛满水的水瓢,水瓢是个一分为二的大号葫芦。唢呐匠递给我一根一尺来长的芦苇杆,我云里雾里的接过芦苇杆,不知道唢呐匠到底什么用意。
\\

用芦苇杆一口气把水瓢里的水吸干,不准换气。我未来的师傅态度严肃的对我说。
\\

我看了看父亲,父亲对着我一个劲的点头,牙咬得紧紧的,他的鼓励显得格外的艰苦卓绝。
\\

我把芦苇杆伸进水里,又看了看他们两个人,唢呐匠的眼神和父亲形成了鲜明的对比,自然而平静,像我面前的这瓢水。
\\

我提了提气,低头把芦苇杆含住,然后一闭眼,腮帮子一紧,一股清凉顿时排山倒海的涌向喉咙。我睁开眼,看见瓢里的水正急速的消退,开始我还信心满满的,等水消退到一半的时候,气就有些喘不过了,水只剩下三分之一的时候,不光气上不来,连脑袋也开始发晕了,胸口也闷的难受,我像就要死了。
\\

快,快,快,不多了。是父亲的声音,像从天外传来的。
\\

终于,我一屁股坐倒在地,仰着头大口的喘气,我又看见太阳了,是个煎糊的鸡蛋。
\\

等太阳重新变成黄色,我听见父亲在央求唢呐匠。
\\

您老就收下他吧!父亲带着哭腔说。
\\

他气不足,不是做唢呐匠的料子。
\\

他气很足的,真的,平时吼他两个妹妹的声音全水庄都能听见。
\\

唢呐匠笑笑,不说话了。
\\

这时候我看见父亲过来了,他含着眼泪,咬牙切齿的操起桌上的水瓢,劈头盖脸的向我猛砸下来。
\\

你个狗日的,连瓢水都吸不干,你还有啥能耐?水瓢正砸在我脑门上,我听见了骨头炸裂的声音。我高喊一声,仰面倒下,太阳不见了,只有一些纷乱的蛋黄,还打着旋的四处流淌。
\\

怎么样?他叫的声音够大吧?气足吧?父亲的声音怪怪的,阴森潮湿。
\\

我努力睁开眼,又看见了父亲高高扬起的水瓢。
\\

叫啊!大声叫啊!父亲喊。
\\

我不知道父亲为什么要这样。我做不成唢呐匠怎么会令他如此气急败坏。
\\

正当我万分惊惧的时候,我看见了一只手。
\\

那只手牢牢攥住了父亲的手腕。\\
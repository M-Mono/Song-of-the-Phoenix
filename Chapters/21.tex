\fancyhead[LO]{{\scriptsize 【百鸟朝凤】第二十一章}} %奇數頁眉的左邊
\fancyhead[RO]{\thepage} %奇數頁眉的右邊
\fancyhead[LE]{\thepage} %偶數頁眉的左邊
\fancyhead[RE]{{\scriptsize 【百鸟朝凤】第二十一章}} %偶數頁眉的右邊
\fancyfoot[LE,RO]{}
\fancyfoot[LO,CE]{}
\fancyfoot[CO,RE]{}
\chapter*{二十一}
\addcontentsline{toc}{chapter}{\hspace{11mm}第二十一章}
%\thispagestyle{empty}
今年第一场雪刚过,村长领着几个人到了我家。\\

我站在院子里,村长拍着我的肩膀说:这就是无双镇游家唢呐班子的班主。\\

很年轻啊!一个戴着眼镜的中年人说。\\

是这样的,他说,我们是省里面派下来挖掘和收集民间民俗文化的。\\

我说你就说找我什么事情吧。\\

戴眼镜的说我们想听一听你的唢呐班子吹一场完整的唢呐。我说游家班已经没有了,火庄有,你们去看看吧。那人笑笑,说我们刚从那里过来,怎么说呢!他干咳了一声:“我们听过了,他们那个严格说起来还不能算纯正的唢呐。”\\

你看?他递给我一支烟说。\\

我说怕不行了,我的师兄弟们全进城了。\\

这时候站出来一个年轻一些的,村长赶忙出来介绍说这是县里来的宣传部长。年轻的部长很豪迈的一挥手,说去把他们都叫回来,费用我们来出。他的语调和姿势让我热血一下涌了上来,我仿佛看到了我的游家班整齐出场的场景,那是多么让人神往的一个场面啊!七八个人一字排开,悠悠扬扬的吹上一场。我梦里经常出现这样的场景。\\

我说好。\\

冬天快过去了,我接到了蓝玉的一封信,他在信上说,他已经在省城站住了,拥有了自己的纸箱厂。我决定去省城把我的师兄弟们找回来,我要把我的游家班重新捏拢来,我要无双镇有最纯正的唢呐。\\

省城真大,走下客车我有了溺水的感觉。\\

根据地址东寻西找了一整天,我终于在一个胡同里找到了蓝玉的纸箱厂。\\

推开铁门,一个守门的老头在门里一间昏暗的屋子里看报纸。\\

请问蓝玉在吗?\\

“蓝厂长出门去了。”老头答:“你找他什么事?”老头抬起头问。\\

“师傅!!”\\

……\\

那天夜里,蓝玉把在这个城市的师兄弟们都通知到了一处,还请大家去了一家金碧辉煌的饭店吃了一顿饭。师傅还是老样子,饭桌上一句话没有,沉默寡言的吃。我说明来意,师傅的眼里掠过一抹亮光,然后他抹了抹嘴,说上面都重视了,这是好事啊!\\

好多年没摸那玩意了。二师兄感叹。\\

我从包裹里取出来一支唢呐递给二师兄,说试试?二师兄把唢呐接过去,端平,刚把哨管放进嘴里,他的眼神暮然黯淡,然后他举起右手,我看见我在木材厂打工的二师兄中指齐根没有了。\\

让锯木机吃掉了。他说,这辈子都吹不了唢呐了。\\

在水泥厂负责卸货的四师兄接过唢呐,说我试试,他架子还在,像模像样的摆好姿势,唢呐在他嘴里没有想象和期待中的嘹亮,只闷哼了一声,就痛苦地停滞了。他抽出唢呐吐出一口浓痰,我看见地上的浓痰有水泥一样的颜色。\\

别回去了,留下来吧!蓝玉看着我说。我喝了一大口酒,说我要回去,我一定要回去。看着桌子上的师兄师弟们,我忍不住哭了,师傅也哭了。\\

我知道,唢呐已经彻底离我而去了,这个在我的生命里曾经如此崇高和诗意的东西,如同伤口里奔涌而出的热血,现在,它终于流完了,淌干了。\\

夜晚,师傅还有师兄弟们送我去火车站。我们沿着城市冰冷的道路一直走,没有人说话,只有往来的车辆拉出让人心悸的呼啸,偶尔有行人经过,都一色的低着头,把脑袋往前伸,急冲冲的扑进城市迷离慌乱的大街小巷。\\

在车站外一块巨大的广告牌下,一个衣衫褴褛的老乞丐正举着唢呐呜呜地吹,唢呐声在闪烁的夜色里凄凉高远。\\

这是一曲纯正的“百鸟朝凤”。\\

\begin{center}
	- 全书完 -
\end{center}
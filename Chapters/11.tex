\fancyhead[LO]{{\scriptsize 【百鸟朝凤】第十一章}} %奇數頁眉的左邊
\fancyhead[RO]{\thepage} %奇數頁眉的右邊
\fancyhead[LE]{\thepage} %偶數頁眉的左邊
\fancyhead[RE]{{\scriptsize 【百鸟朝凤】第十一章}} %偶數頁眉的右邊
\fancyfoot[LE,RO]{}
\fancyfoot[LO,CE]{}
\fancyfoot[CO,RE]{}
\chapter*{十一}
\addcontentsline{toc}{chapter}{\hspace{11mm}第十一章}
%\thispagestyle{empty}
游家班接的第一单活是水庄的毛长生家。\\

过来接活的是长生的侄儿。一进院子就给我父亲派烟,父亲把香烟吸得有滋有味的,一脸的幸福。这是他的唢呐匠儿子严格意义上给他带来的第一次实惠,滋味自然是与众不同的。\\

我刚从屋子里出来,父亲就冲着我喊:“八台哟!”\\

“我叔是啥人?别说八台,十六台也不在话下的。”接活的说。\\

父亲白了长生侄儿一眼:“你妈的x,哪有十六台?”\\

长生侄儿裂了裂嘴,说现在不是天鸣做主吗?自个儿造啊!别说十六台,捋出个九九八十一台也行啊!\\

父亲这回笑了,快意的猛吸了一大口烟,他从蹲着的长条木凳子上一跃而下,说:“那倒是。”\\

我点了师傅和几个师兄的名字,长生侄儿就蹦达着去通知了,走的时候又给父亲派了一支烟,父亲接过香烟说你龟儿子脚程放快些,晚上要吹一道的哟。\\

其他几个师兄都来了,师傅和蓝玉没有来,长生侄儿说他好说歹说说到口水都干了,师傅还是不来,只推说身子不太利索。我没有问他蓝玉为什么没有来。\\

我家屋子不大,寨邻来了不少,把一个院子堵得满满的,都想看看游家班的第一次出活预演。大庄叔也来了,父亲还单独给了他一条独凳子和一碗浓茶。大庄叔一脸的笑,说真没想到这唢呐班的当家人会是天鸣这崽儿,平时十棍子敲不出一个屁,吹起唢呐来还叫喳喳的呢!当年你爹说你能吹上百鸟朝凤老子还不相信呢,看来你游家真的是祖坟上冒青烟了。\\

几个师兄话不多,一直笑,父亲给每个人都倒了一碗烧酒,还不停的催促说喝啊喝啊润润嗓子啊!\\

水庄的夜晚好多年没有这样热闹了。四支唢呐呜呜啦啦的吼。奏完一曲丧调,人群里有人喊说天鸣整一曲百鸟朝凤给大家听听。我说那不行,师傅交代过的,这曲子是不能乱吹的。人群又起来一阵轰,老庄叔把凳子往我面前挪了挪,说就整一段,给大伙洗洗耳朵,这曲子当年肖大老师走的时候我听焦三爷整过一回,那阵势真他奶奶的不得了,能把人的骨头都给吹酥了。我还是摇头,父亲站在我身后对大家说今天就到这儿吧,以后机会多的是,天鸣保证给大家吹。老庄叔看见父亲发了话,也站起来说对对对,不依规矩不成,以后听的时间还多,散了吧都。\\

人群散了去,我对几个师兄说,这是游家班第一次接活,不能砸了,再走几遍吧。\\

远远的就看见了长生,他头上顶着一块雪白的孝布站在院子边等我们。看我们过来,长生给每个人派了一支烟。自己也啜上一支。我说老人家什么时候走的?长生喷出一口烟,笑着说这个月都死三四次了,死去没多久又缓了过来,直到昨天早晨才算是死透。旁边一个老人干咳了两声,说长生,快行接师礼呀!接师礼就是磕头。长生回头看了看旁边的老人,说接什么卵师呀!天鸣和我啥关系?一起比过鸡鸡的。然后他回头看着我笑笑,我也笑笑。\\

我其实倒是很希望长生给我磕个头。长生比我大五岁,是个精灵货,个子也比我大,小时候放牛我没少挨他揍,揍了我还要我喊他爹,喊过他多少回爹我都忘了。我一直想着报仇的,慢慢长大了,懂事了,报仇这个事情也就丢到一边了。今天本来是个机会,可长生还是显示着他一贯的与众不同。算起来,长生算是水庄第一个穿夹克和牛仔裤的人,这几年水庄人都前仆后继的把庇护了自己几千年的土墙房推到了,于是水庄出现了一排一排的镶着白晃晃瓷砖的砖墙房。水生看准了这个变化,拉上一群人在水庄的河滩上搞了一个砖厂。现在水庄好多人都不叫他长生了,叫他毛老板。\\

长生给游家班的待遇充分展示了他毛老板这个称呼并非浪得虚名。一人一条香烟,比起那些一支一支扔散烟的人家户,这种一次性的大额支付确实让人快意,因为我从几个师兄接过香烟的眼神可以看出,他们像打了一辈子小鱼小虾的渔民,今天忽然就网起来了一头海豹。\\

然后,你就可以看见我的几个师兄在吹奏的时候是多么的卖力,我真担心他们用力过猛会震破手里的唢呐。特别是长生打我们旁边经过的时候,我大师兄高高坟起的腮帮子像极了他妻子怀胎十月时的大肚皮。\\

除了香烟,毛老板的慷慨还体现在很多细节上,比如润嗓酒,是瓶装的老窖;再比如乐师饭,居然有虾。那玩意通体透红中规中矩的趴在盘子里,连我都看得傻了,虾我听说过的,是水里的东西,我们无双镇好多水,可我们无双镇的水里没有虾,只有一汪一汪淡绿的水草。长生最大的慷慨还不是这些,而是看见我们卖力的吹奏时,他就会过来先给每个人递上一支烟,说别太当回事了,随便吹吹就他妈结了。\\

走的那天长生没有送我们,而是每人递给我们一把钱。大师兄说了,这是他吹唢呐以来领到的最多一回钱,二师兄在一边也说,钱是最多的一次,可吹得是最轻松的一次。\\

我捏着一把钱站在水庄的木桥上,木木的看着一庄子正起来的炊烟。\\
\fancyhead[LO]{{\scriptsize 【百鸟朝凤】第十七章}} %奇數頁眉的左邊
\fancyhead[RO]{\thepage} %奇數頁眉的右邊
\fancyhead[LE]{\thepage} %偶數頁眉的左邊
\fancyhead[RE]{{\scriptsize 【百鸟朝凤】第十七章}} %偶數頁眉的右邊
\fancyfoot[LE,RO]{}
\fancyfoot[LO,CE]{}
\fancyfoot[CO,RE]{}
\chapter*{十七}
\addcontentsline{toc}{chapter}{\hspace{11mm}第十七章}
%\thispagestyle{empty}
我还没来得及去找师傅,师傅就先来找我了。\\

师傅一进院子就骂:“你个小狗日的游天鸣给老子出来。”\\

我出来看见师傅站在院子里,他的双脚沾满了泥,连衣服的下摆都有星星点点的泥点子。脸和我当初去拜师的时候一样黑,只是皱纹更多了,看见师傅老了一大截,我忽然上来了一些伤感。这个无双镇当年响当当的焦家班的掌门人,像入了冬的一棵老槐树,尽是令人沮丧的残败。最揪心的就是他一身灰布衣服了,还是老式样,对襟衫,几个地方都是补丁,要知道,现在无双镇像这样有补丁的衣服是不多见了,偶尔看见,不会有人说你艰苦朴素,下意识还会把你往穷人堆里推。\\

我喊了一声师傅。\\

“不要叫我师傅,我没有你这样的徒弟。”师傅往地上狠狠的啐了一口痰:“当初你是怎样说的,有口气就要把这活往下传,可这才过去多久?昨天就有人给我递话了,说无双镇的游家班散伙了,垮台了,有活也不接了,无双镇从今以后就没有唢呐匠了。”\\

我说师傅你先进屋,我们到屋里说。师傅一挥手:“进不起你的宝殿门,你现在哪里还瞧得上吹唢呐的?”。还是母亲出来,说焦师傅你先不要着急,进来说,天鸣正托人到处通知他的师兄弟们呢,这几天就要出活。母亲说话时不断对着我眨眼,我慌忙应和说对对对。师傅火气这才消了些。背着手走进屋,也不看我,只说,不给老子说个一二三,看老子不撕破你那张X嘴。\\

师傅坐下来,接过母亲倒来的茶,怒气冲冲的等我的解释。听完我的解释,师傅把茶碗往桌上狠狠一掼。\\

“我去找他们,几个狗日的还翻天了。”\\

师傅出了院门,看我还站在屋檐下,就吼:“傻了?游家班班主是我还是你?”,我哦了一声,才快步跟上去。\\

我跟在师傅身后,一路上他一句话都没有,但我能清晰的听见他大口大口喘气的声音。\\

二师兄对我和师傅的到来有些意外。当时二师兄正在打点行装,屋檐下,他正把一捆衣物狠命的往一个陈旧的蛇皮口袋里塞,口袋太小,装不下二师兄远涉的必须,就委屈地从口沿处往下撕裂,还发出吱吱的怪叫。二师兄骂了一句,抬起头就看见了师傅和我,他的嘴上下翕动着,是想说些什么,但从师傅的脸色他似乎已经明白了我们的来意,于是就什么也没有说。他放下手里的袋子,直起身子,从屋檐下的檐坎上下来,站在师傅面前,静悄悄的,没有一点声息。\\

师傅没有理二师兄,鼻子有了一声闷哼后,径直走到屋檐下,把口袋拎到院子里,把口袋里的东西一样一样的掏出来往院子里抛撒。师傅的这个动作持续了好长时间,我惊讶于这个看上去个儿不大的口袋居然有如此壮观的吞吐量,等师傅捋直了身子,院子里早成了花花绿绿的晾晒场。\\

师傅把干瘪的口袋踩在脚下,目光盯着二师兄,那眼神像水庄六月的日头,能把人烤晕过去的。\\

二师兄低着头,他一句话没有说,两个手交互搓揉着,这时候有几只麻雀从天而降,欢快的在院子里那些各式各样的衣物上跳跃。二师兄忽然松开了两只互握着的手,低头从师傅旁边走过去,蹲下身子把地上的衣物一件一件的拾起来搭在臂弯处,其间还拍拍打打的扇掉衣物上的灰尘。等他臂弯放不下后,他就慢慢蹲着移到师傅的脚边,伸出一只手扯师傅脚下的蛇皮口袋,师傅一动不动,师兄却执着地扯,力量也越来越大,最后我看见师傅的身体都开始摇晃起来。我站在一边看着这对奇特的师徒,他们就像在出演一出哑剧,每一个动作和眼神都极具深意,所有的表达都在你来我往的无声的动作中了。这时我的师傅伸出一只脚,狠狠的踹向了他二徒弟的面部,我看见二师兄猝然的往后倒了下去,像刚被掏空的蛇皮口袋。好半天,师兄才复苏的蛇一样从地上卷曲着爬起来,两道殷红从他的鼻孔蜿蜒而下,几乎穿越了整个面部。他没有完全站起来,依旧半蹲着,一步步挪到师傅的脚边,伸出一只手,固执的去扯师傅脚下的口袋。\\

这时候,我看见我的师傅面部完全变成了死灰色,五官也剧烈地痉挛着,像一锅煮烂的饺子。良久,他终于仰头长长的叹了一口气,叹气的感觉和水庄冬天的寒风一般,经过皮肤,直抵骨髓,能把人的那颗心都冻僵了。他终于移开了紧紧踩踏着口袋的脚,转身走了,走得很快,留给我一个颤抖不止的背影。\\

\fancyhead[LO]{{\scriptsize 【百鸟朝凤】第四章}} %奇數頁眉的左邊
\fancyhead[RO]{\thepage} %奇數頁眉的右邊
\fancyhead[LE]{\thepage} %偶數頁眉的左邊
\fancyhead[RE]{{\scriptsize 【百鸟朝凤】第四章}} %偶數頁眉的右邊
\fancyfoot[LE,RO]{}
\fancyfoot[LO,CE]{}
\fancyfoot[CO,RE]{}
\chapter*{四}
\addcontentsline{toc}{chapter}{\hspace{11mm}第四章}
%\thispagestyle{empty}
到土庄两个月零四天,蓝玉来了。
\\

蓝玉来的头天晚上,土庄下了一场罕见的暴雨。第二天一大早我起得床来,看见院子里跪着一个男娃子。他的全身上下都湿透了,衣裤上粘满了黄泥。在他的身边,是一个三十出头的汉子,也披着一身的潮湿,他两个手不停地搓着,眼睛跟着师傅转。这个时候,我的师傅正在牛圈边给牛喂草,他大把大把的把青草扔给圈里的牛,还在院子里过来过去的,就是不看院子里的蓝玉和他的父亲,仿佛院子里的两个人只是虚幻的存在。我看出了蓝玉父子的尴尬,想起自己刚来到这个院子的情景,就有些同情院子里的人。
\\

这个时候,蓝玉抬起了头,向我这边看了一眼,我给了他一个浅浅的微笑,一脸黄泥的蓝玉也笑了,他的笑意很薄很轻,仿佛往湖面上扔了一块拇指大小的石子起来的一层涟漪。好多年后蓝玉还在对我说,他说当时跪在泥水里的他都有了天地崩塌的感觉,他已经打定回家的主意了,不管他的父亲同不同意他都准备回家了,就是因为我的那个微笑,他留了下来。
\\

师傅同意收下蓝玉是在蓝玉的父亲两个膝盖也重重的跌落在泥地里后。当时师傅正抱着一捆青草往牛圈边去。那个异样的声音至今还犹然在耳,我看见蓝玉的父亲两腿一屈,接着他面前的水被砸得稀烂,咚,一个院子都颤抖起来。师傅回过头就僵在那里了,然后他说你起来吧,我可以试试他是不是吹唢呐的料,不行的话,你还得把娃领回去。
\\

和我相比,蓝玉的测试多出了好几项内容。除了吸水,还有吹鸡毛,师傅把一片鸡毛扔到天上,要蓝玉用嘴把鸡毛留在空中,一袋烟的功夫不能掉到地面。还有就是打靶,含上一口水,对着桌上的木牌,在四步外的距离用嘴里的水把木牌射倒。我很为蓝玉担心,因为我连一瓢水也是吸不完的。\\

蓝玉轻描淡写的就完成了测试,不仅我惊讶,连师傅都有些惊讶了。虽然他把这种惊讶包裹得很严实,当蓝玉把桌上的木牌射倒后,他的两条眉毛很迅速的彼此凑了凑,眉间也多出来一条窄而深的沟壑。我至今都承认,我的师弟蓝玉天分比我要高得多。
\\

蓝玉留下来了,和我住一张床。师傅还郑重的把我介绍给了蓝玉,说这是你师兄,师兄师弟,就要像亲兄弟一样的,懂不懂?蓝玉点了点头,我也点了点头。
\\

晚上蓝玉在床上问我,吹唢呐好玩吗?我说不知道,蓝玉惊讶地翻起来说你怎么会不知道呢?你不是都来两个月了吗?我说我还没吹上一天的唢呐呢!哪你在干啥?蓝玉问。喝水,喝河湾的水。我答。
\\

打蓝玉来后,土庄的河湾边吸水的娃由一个变成了两个。土庄人从河湾过就大声说焦三爷又收徒弟了,焦家唢呐班人强马壮了。
\\

在我们吸水的这段日子里,师傅和他的唢呐班共出了十多趟门。整个无双镇都跑遍了。我和蓝玉还认识了焦家唢呐班的师兄们。我的大师兄年纪和我父亲差不多,师傅让我和蓝玉叫他大师兄,我们都有些不好意思,毕竟他是个满脸胡须的大人。我们怯怯的喊罢,大师兄摸摸我们的脑袋,然后看着师傅笑笑。师傅说磨磨都能出来。大师兄又笑一回,他笑的时候嘴裂得很大,胡子满脸跑,他把唢呐凑到嘴里,唢呐的苇哨和铜围圈就不见了。
\\

接活后出门的前一晚,焦家班照例要吹一场的。院子里摆上一张桌子,桌子上有师娘煮好的苦丁茶和炸好的黄豆。师傅和他的徒弟们散坐在院子里,大家先聊一些家常。聊家常的时候有一个人声音最大,说话像打雷,他是我的二师兄。据师娘讲,二师兄是师傅最满意的徒弟,天分好,也刻苦,特别擅长吹丧调,能在灵堂把一屋子人吹得流眼抹泪。聊一阵子天,师傅就咳嗽两声,众人会意,各自从布袋子里抽出唢呐,第一步是调音,看看唢呐音调对不对;然后师傅起调,如果接的是红事,就吹喜调,喜调节奏快,轻飘飘的在院子里奔跑;如果接的是白事,就吹丧调,丧调慢,仿佛泼洒在地上的黏稠的米汤,等到师傅独奏的那一段,我和蓝玉眼窝子都有了一窝水。
\\

无双镇大部分人家接唢呐都是四台,所谓四台,就是只有四个唢呐手合奏;比四台讲究的是八台,八台除了四个唢呐手,还有一个鼓手,一个钵手,一个锣手,一个钞手。八台不仅场面大,奏起来也气势非凡。师娘告诉我,如果练的是八台,土庄的人都会来,聚在院子里,屏声静气的听完才散去。毕竟八台一是难度大,二是价钱高,一般人家是请不起的,土庄人近水楼台,运气好的话一年能听上一两回。我又问师娘,有比八台更厉害的吗?师娘笑笑,说有,我问:是什么?
\\

百鸟朝凤,师娘答。
\\

怎么个吹法?我问。
\\

独奏!师娘说这话的时候神情肃穆。
\\

独奏?谁独奏?我和蓝玉惊讶的问。
\\

夜风撩着师娘的头发,她的表情像一本历史书,好久她才说,当然是你们师傅。\\
\fancyhead[LO]{{\scriptsize 【百鸟朝凤】第十五章}} %奇數頁眉的左邊
\fancyhead[RO]{\thepage} %奇數頁眉的右邊
\fancyhead[LE]{\thepage} %偶數頁眉的左邊
\fancyhead[RE]{{\scriptsize 【百鸟朝凤】第十五章}} %偶數頁眉的右邊
\fancyfoot[LE,RO]{}
\fancyfoot[LO,CE]{}
\fancyfoot[CO,RE]{}
\chapter*{十五}
\addcontentsline{toc}{chapter}{\hspace{11mm}第十五章}
%\thispagestyle{empty}
老马的葬礼新鲜而奇特。\\

乡村的葬礼不一定非得沉痛,但起码是严肃的。七十岁以上的老人去了那头,这叫喜丧,气氛是可以鼓噪些的。老马六十不到,他的葬礼是没有资格欢欣鼓舞的。可就在他入土的头一个晚上,马家大院出现了前所未有的喜气洋洋,那些奔丧迟到的人走进马家大院都一头雾水,以为走错了门,这里怎么看都像是老马家在娶媳妇,说在办丧事打死人家都不相信。\\

让老马由死而生的,是那支乐队。\\

先是几个人叮叮咚咚的乱敲一通,然后就唱开了。\\

鼓捣吉他的边弹边唱,唱的过程中还摇头晃脑的。他唱的是什么我听不懂,我的师弟蓝玉在一旁跟着哼哼,我问蓝玉他唱的是什么,蓝玉说是时下正流行的,只能跟着哼哼几句,整个儿的记不住,曲子叫什么名字也记不住了。\\

开始,木庄的乡亲们站在院子里,脸上都有了怒气。每个人都不很适应,脸上都有矜持的不满,一个上了年纪的阿婆把手里的一棵白菜狠狠的摔在地上,眼神离奇的愤怒,嘴里还咕咕囔囔,最后很沉痛的看了看灵堂。我知道他是在为死去的老马打抱不平呢!\\

渐渐的,大家的神色开始舒展开了,有一些年轻人还饶有兴致的围在乐队的周围,环抱双手,唱到自己熟悉的曲子时还情不自禁的跟着哼哼。\\

游家班站在马家大院的屋檐下,局促得像一群刚进门的小媳妇。我低头看了看手里的唢呐,才忽然想起来我们也是有活干的。\\

雨停了,空气清爽得不行,干干净净的。院子里为游家班准备的呈扇形排开的凳子还在。我们过去坐好。我看了看几个师兄。\\

“还吹啊?”一个师兄问。\\

“怎么不吹?又不是来舔死人干鸡巴的!”我对他的怯懦出离的愤怒。\\

我还拿起脚边的酒瓶子灌了一大口烧酒,悲壮得像即将奔赴战场的战士。\\

呜呜啦啦!呜呜啦啦!\\

平日嘹亮的唢呐声此刻却细弱游丝,我使劲瞪了几个师兄两大眼,大家会意,腮帮子高鼓,眼睛瞪得斗大。还是脆弱,那边的声响骄傲而高亢,这边的声音像临死之人哀婉的残音。一曲完毕,几个师兄都一脸的沮丧,大家你看看我,我看看你。\\

吹,往死里吹,吹死那群狗日的。师弟蓝玉在一边给大家打气。\\

我们吹得很卖力,在那边气势较弱的当口,就会有高亢的唢呐声从杂乱的声音缝隙里飙出去,那是被埋在泥土中的生命扒开生命出口时的激动人心,那是伸手不见五指的暗夜里划燃一根火柴后的欣喜若狂。\\

我们都很快意,那边的几只眼睛不停的往这边看,看得出,眼神里尽是鄙夷和不屑,甚至还有厌恶。\\

说实话,我对这群不速之客眼神里的内容是能够接受的,甚至他们就应该对我手里的这支唢呐感到厌恶才对。只是我没有想到,对我手里这支唢呐感到厌恶的不光是他们。\\

一个围在乐队边唱得最欢的一个年轻人不知什么时候站在我的面前。他斜着脑袋看着我,表情怪怪的,像是在瞻仰一具刚出土的千年干尸。我把唢呐从嘴里拔出来,吞了一口唾沫问:干什么?\\

你们吹一次能得多少钱?他说。\\

和你有关系吗?我答。\\

我付你双倍的钱,条件是你们不要再吹了。\\

我摇头说那不行。\\

没人喜欢听你们几根长鸡巴吹出来的声音。\\

那我也要吹。\\

这时候我的师弟站出来了,他过来推了年轻人一把。说柳三你干啥?叫柳三的说关你啥事?蓝玉说就他妈关我的事,咋了?\\

两个人就你来我往的开始推搡。本来已经有人过来劝住了的,柳三这个时候像想起了什么来,然后他说:“哦!我差点忘记了,你原来也是个吹破唢呐的!”说完还嘿嘿的干笑两声。\\

我看见蓝玉的拳头越过三个人的脑袋,奔着柳三的脑袋呼啸去了。一声闷响后,殷红的鲜血从柳三的鼻孔里奔涌而出。场面一下子就乱了,呼喊声,叫骂声,拳头打中某个部位后的空响,夹杂在癫狂的乐曲声中,活像一锅滚热的辣油。\\

第二天是蓝玉送我们离开的。我的师弟脑袋上缠着一块纱布,左边眼圈像块圆形的晒煤场。在我们身后远处的山梁上,送葬的队伍爬行在蜿蜒的山道上,那利箭一样的乐器声响充斥着木庄的每一个角落。\\

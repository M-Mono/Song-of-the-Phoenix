\fancyhead[LO]{{\scriptsize 【百鸟朝凤】第八章}} %奇數頁眉的左邊
\fancyhead[RO]{\thepage} %奇數頁眉的右邊
\fancyhead[LE]{\thepage} %偶數頁眉的左邊
\fancyhead[RE]{{\scriptsize 【百鸟朝凤】第八章}} %偶數頁眉的右邊
\fancyfoot[LE,RO]{}
\fancyfoot[LO,CE]{}
\fancyfoot[CO,RE]{}
\chapter*{八}
\addcontentsline{toc}{chapter}{\hspace{11mm}第八章}
%\thispagestyle{empty}
春天降临了。\\

乡村的春天总是和仪式有千丝万缕的联系。像我们无双镇,春天一露头,就有拜谷节,播洒谷种的前一夜,每个村子的老老少少都要带上祭品,去本村最大的一块稻田里供奉谷神;拜谷节过去没几天,就该是迎接灶神爷的日子了,猪头是不能少的,还有小米渣,听老人们说,天上是没有小米渣的,人间全靠这点东西留住他老人家了;把灶神爷安顿好,就是晒花节了,太阳公公和花仙一起供奉,因为有两个神仙,供品自然不能少,蜂蜜、白米,干菊花,还有圆圆的玉米饼。太阳还没有出来,一庄人早就遥对着太阳升起的地方把供品摆放妥贴了,等那抹血红一上来,大家就整齐的磕头作揖,好听的话也会说不少,庄稼人没野心,就是祈求有个好年成。\\

晒花节刚过,土庄又热闹了。人们槐花串似的往焦三爷的院子里跑,扛凳子搬桌子的。遇上闲逛的路人,就有人招呼:“焦三爷传声了!”,路上的人一听,一张脸就怒放了,随即融入队伍。往焦三爷的院子迤逦而来。\\

土庄人等这个盛况的日子已经很久了。\\

无双镇的唢呐班每一代都有一个班主,上一代班主把位置腾给下一代是有仪式的,这个仪式叫“传声”,不传别的,就传那首无双镇只有少数人有耳福听到过的“百鸟朝凤”。接受传声的弟子从此就可以自立门户,纳徒授艺了,而且从此就可以有自己的名号,比如受传的弟子姓张,他的唢呐班子就叫张家班,姓王,则叫王家班。总之,那不仅仅是一门手艺,更是一种荣耀,它似乎是对一个唢呐艺人人品和艺品最有力的注脚,无双镇的五个庄子都以本庄能出这样一个人为荣。\\

这个仪式最吸引人的还不是他的稀有,而是神秘。在仪式开始之前,没有人知道谁是下一代的唢呐王。所以,焦家班所有的弟子都是要参加这个仪式的,连他们的亲人都会四里八乡的赶来参加,因为谁都可能成为新一代的唢呐王。\\

人实在太多了,师傅的院子都装不下了,于是屋子周围的树上都满满当当的挂满了人参果。我和我的一班师兄弟坐在院子正中间,两边是我们的亲人,我父母还有两个妹妹都来了;我的师弟蓝玉坐在我的旁边,他的家人也来了,比我的父母还来得早些。他们的脸上都是按捺不住的期待和兴奋。\\

屋檐下有一张八仙桌,八仙桌的下面是一头刚宰杀完毕的肥猪。此刻,这头猪是供品,仪式结束后,他将成为全土庄人的一顿牙祭。猪头的前面有个火盆,火盆里的冥纸还在燃烧。师傅坐在八仙桌后面。他一直在闷着头抽烟,师傅的烟叶是很考究的,烟叶晒得很干,吸起来烟雾特别大。很快,师傅的一张脸就不见了,他的半截身子都隐在一片雾障中,像一个踏云的神人,我竟然生出一些隐约的幻意。\\

良久,师傅才站起来,四平八稳的拄灭手里的烟袋,对着人群,平伸出双手往下压了压。喧闹的人群瞬间就安静下来。往地上吐了一口痰,师傅发话了。\\

“我快要吹不动了,可咱们这山旮旯不能没有唢呐,干够了,干累了,大家伙儿听一段还能解解乏。所以啊!在咱们这地头唢呐不能断了种。我寻思了好久,该找一个能把唢呐继续吹下去的人了!”师傅咳嗽了两声,停了停,下面又开始有响声了。这个时候我偷偷的侧目看了看蓝玉,我发现蓝玉也在偷偷的看我,他的嘴角还淌着一些笑。四目相对,我的脸刷就红了,像是心里某种隐秘的东西被戳穿了似的。蓝玉的脸没有红,他的脑袋抬得更高了,像一只刚刚得胜的大公鸡。我就升起一些不快,想还没见底呢,咋知道水底是不是石头?又想想,我的这班师兄弟里,也只有蓝玉最适合了,他人精灵,天分高,也勤苦。反正最后是他我也不会惊奇的。最后我觉得我那几个师兄也可怜,为什么师傅不全给传了呢?那样就整齐了,人人有份,个个能吹百鸟朝凤,焦家班、蓝家班、游家班,还不响亮死啊!\\

师傅又开腔了:“我这几年收了不少徒弟,大大小小的,个个都有些活儿,出活也带劲,没给吹唢呐的丢人。”顿了顿师傅接着说:“我们吹唢呐的,好算歹算也是一门匠活,既然是匠活,就得有把这个活传下去的责任,所以,我今天找的这个人,不是看他的唢呐吹得多好,而是他有没有把唢呐吹到骨头缝里,一个把唢呐吹进了骨头缝的人,就是拼了老命都会把这活保住往下传的。”师傅又咳嗽了两声,对旁边的师娘点了点头,师娘过来递给师傅一个黑绸布袋子。师傅接过来,小心翼翼的从里面抽出来一支唢呐。远远的我就感觉到了这支唢呐该有些年龄了,铜碗虽然亮得耀眼,却薄如蝉翼,杆子是老黄木的,唢呐的杆子一般就是白木,最好的也就是黄木,能用这样色泽的老黄木制成的唢呐,足见它的名贵。乡村人一般是见不到这样的稀罕货的。\\

“这支唢呐是我的师傅给我的,它已经有五六代人用过了,这支唢呐只能吹奏一个曲子,这个曲子就是百鸟朝凤。现在我把它传下去,我也希望我们无双镇的唢呐匠能把它世世代代的传下去。”师傅举着唢呐说。\\

院子里一点声音都没有,我只听见我的师弟蓝玉的喘息声,所有的眼睛都盯着师傅手里的那支唢呐。我相信这一刻的土庄是最肃穆的了,这种肃穆在了无声息中更显得黏稠,我最后只能听见自己的呼吸声了。\\

我侧目看了看我的师弟蓝玉,他紧缩着脖子,脑袋花骨朵似的。慢慢地,他的脖子被拉长了,成了一朵盛开的鲜花,花朵儿正期待着雨露的降临,焦虑、渴望在稚嫩的花瓣间涌动着。蓦然,盛开的鲜花枯萎了。几乎就在一眨眼间,正准备迎风怒放的花儿无声地凋谢了,花瓣起来了一层死灰,花杆儿也挫短了半截。这朵刚才还生机蓬勃的花儿,转眼间铺满了绝望的颜色。悲伤一下从我的心底涌起来,我的师弟蓝玉,迅速的在我眼睛里枯萎,他的目光慢慢的转向了我,我能看懂他的眼神,有不信、不甘、绝望,当然,还有怨恨,可我看到的怨恨很少,很稀薄,星星点点的。\\

这时候我的父亲,水庄的游本盛在旁边喊我:“你呆了,师傅叫你呢!”\\

父亲的声音像耍魔术的使用的道具,充满了意外和惊喜。\\
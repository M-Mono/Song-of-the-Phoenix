\fancyhead[LO]{{\scriptsize 【百鸟朝凤】第十六章}} %奇數頁眉的左邊
\fancyhead[RO]{\thepage} %奇數頁眉的右邊
\fancyhead[LE]{\thepage} %偶數頁眉的左邊
\fancyhead[RE]{{\scriptsize 【百鸟朝凤】第十六章}} %偶數頁眉的右邊
\fancyfoot[LE,RO]{}
\fancyfoot[LO,CE]{}
\fancyfoot[CO,RE]{}
\chapter*{十六}
\addcontentsline{toc}{chapter}{\hspace{11mm}第十六章}
%\thispagestyle{empty}
水庄最近变化很多,有些是那种轮回式的变化,比如蒜薹又到了采摘的时候;有些变化则是新鲜的,让人鼓舞的,比如水庄通往县城的水泥路完工了,孩子们在新修完的水泥路上撒欢,大大小小的车辆赶趟儿似的往水庄跑,仿佛一夜之间,水庄就和县城抱成一团了。要知道,以前水庄人要去趟县城可不是那样容易的,不在坑坑洼洼的山路上颠簸五六个小时,你是看不见县城的。现在好了,去趟县城就像到邻居家串个门儿。\\

这个时候,我的父亲游本盛站在自家大蒜地里,满脸堆笑。在他眼里,像水庄有了水泥路这些新鲜事儿和他没有什么关系,他更关心的是他的大蒜地。今年的大蒜地倒是争气得紧,从冒芽儿开始就顺风顺水的,该采摘了,一根根在和风里炫耀着粗壮的身躯。父亲每天都要到大蒜地走一走,看一看,然后啜着纸烟蹲在土坎上,没有比这让他更满足的事情了。\\

父亲弓着腰在剥蒜薹,一阵风过去,我看见了他两扇瘦窄的屁股。我说歇歇吧。他直起腰,回过头,一脸的怒气:“歇歇?歇歇都能有饭吃老子早歇了!”我不说话了,还后悔刚才说出来的话。我想我最好是闭嘴,我说出来的每一句话,我的父亲都能找出让我难堪的理由。\\

可我发现,我不说话也不行,我不说话父亲也会把他的不满通过诸如眼神和动作传递给我。这一年来,父亲看我的眼神总是充满了疑问和警惕,我就像一只潜入他们家偷食的野猫,不幸正好被他发现了。我这只偷食的野猫只好把尾巴藏着掖着,生怕主人那天不高兴了一脚把你踹出门去。\\

初夏是水庄一年中最好的季节,这个时候的水庄可有生机了,天空清澈碧透,水面也清澈碧透,一庄子待收割的蒜薹也清澈碧透。最打动人的不管你走到哪里,每一个水庄人的脸上都带着笑。水庄人真的没有野心,一次理所当然的丰收就能把一个村庄变得天宽地阔。父亲不和我说话,埋下头继续采摘蒜薹。我直起腰,天空没有一丝云彩,一望无际的蒜地在阳光下像一幅油画。远远的,族中的三叔对着我远远的招手。三叔是我请去通知几个师兄弟出活的人。不知道从哪一天开始,无双镇的唢呐班子省掉了接师礼,连运送出活工具这些规矩都一并没了。我三步两跳的跑过去,先递给三叔一支烟,他撩起衣角擦了擦满脸的汗水,把烟点燃后对我说。\\

“都通知了,只有你大师兄同意来。”\\

“其他人呢?他们怎么说?”\\

“还能说啥?不是说忙就是这里那里不利索咯。”\\

三叔说完走了,走出老远了他好像又想起了什么,回头大声喊:\\

“对了,你二师兄说以后不要去叫他了。”\\

“为什么?”我问。\\

“说下个月要出门了。”\\

“去哪里?”\\

“不知道,大城市咯!”\\

我悻悻的回过头,就看见了父亲那张铁青的脸,他两手叉在腰际,眼睛直直的看着我。我低着头从他旁边走过去,他在后面冷冷的笑,笑完了说:\\

“都快孤家寡人了吧?看你以后还怎么吹?吹牛X还差不多。”\\

晚上我没有吃饭,躺在床上,定定的看着天花板。天花板上有一只蜘蛛倒悬着垂下来,一直垂到我的鼻尖处,我伸出手,让蜘蛛降落在我的手心里,它就顺着我的手臂往上爬,时左时右,我不知道哪里是它想去的地方,或者它压根就没有目的地,只是这样一直往前爬,再往前爬,什么时候爬累了,织个网,就算安家落户了;又抑或被天敌给吃掉了,无声无息的,谁又会去关心一只蜘蛛的未来呢!\\

仿佛一眨眼时间,我身边这个世界一下就变得陌生了,眼里的一切都没变,山还是那座山,河也还是那条河。可有些看不见的东西却不一样了,像水庄的那条河,看上去风平浪静的,可事实不是这样的,小时候下河游泳,一个猛子下去,才发现河底下暗潮汹涌。\\

直到父亲睡了,我才从屋子里出来。母亲重新把菜给我热了热。我吃饭时,母亲还是像小时候一样静静的坐在我的旁边,目不转睛的看着我,眼神里流淌着源源不竭的爱怜。\\

“后天是不是要出活?”母亲问。\\

我点点头。\\

“听你爹说几个师兄都不来?”\\

我又点点头。\\

“唉!”母亲长叹一声,然后她接着说:“天鸣,要不这唢呐不吹了!咱干点别的,凭咱这双手干啥不能活命啊!”\\

我放下碗,转过去对着母亲。\\

“我知道这个理,可当年拜师的时候我给师傅发过誓的,只要还有一口气,就要把这唢呐吹下去。”\\

“可你看,就你一个人也吹不来啊!”\\

“过两天我去找师傅。”\\
\fancyhead[LO]{{\scriptsize 【百鸟朝凤】第十八章}} %奇數頁眉的左邊
\fancyhead[RO]{\thepage} %奇數頁眉的右邊
\fancyhead[LE]{\thepage} %偶數頁眉的左邊
\fancyhead[RE]{{\scriptsize 【百鸟朝凤】第十八章}} %偶數頁眉的右邊
\fancyfoot[LE,RO]{}
\fancyfoot[LO,CE]{}
\fancyfoot[CO,RE]{}
\chapter*{十八}
\addcontentsline{toc}{chapter}{\hspace{11mm}第十八章}
%\thispagestyle{empty}
道路弯弯拐拐,曲折迂回。乡间小路就是这样,站定一个点,极目远眺,道路伸出去没多远就倏然不见了。赶上去,才发现它又折向了某一个去处,再远眺,还是只能看到一根断面条。我们就在这样一条琢磨不定的道路上走着。最前面是我的师傅,中间两个,一个大师兄,一个蓝玉,我跟在最后头。\\

蓝玉自从离开土庄后,没有出过一次活。今天他能站在游家班的队伍里,我总有一种怪怪的感觉。我也不知道师傅是怎样说服蓝玉跟我们出这次活的。那天师傅离开二师兄家后,就直奔木庄去了。昨天晚上,蓝玉推开了我家的门。\\

师傅今天穿了一件新衣服,衣服上的折痕都还清晰可见。他走得很快,像一只老当益壮的野兔。蓝玉有意把步子放慢,很快我们的队伍就断裂成了两个块,前面是师傅和我的大师兄,后面是我和我的师弟蓝玉。\\

和我并排着的蓝玉忽然说:“师傅老了!”。我点点头,蓝玉又说:“这是我第一次正式出活,也是最后一次。”。我转过头看着蓝玉,不知道他想表达什么。过了半晌,蓝玉自言自语:“我答应师傅的,师傅也答应我的。”。\\

我的师弟蓝玉就是这样,总让我琢磨不透,说话也玄机重重。我说这话什么意思?蓝玉笑笑,没说话。我就低头自己想,等我抬起头的时候,幽静的山路上就看不见人影了。\\

在无双镇,和其他几个庄子比,火庄一直落在后面,房屋还多是拉拉杂杂的茅草屋,道路也没有其他几个庄子来得宽敞。但火庄人老实。无双镇人到集市上买鸡蛋,特别是买土鸡蛋,都要先问问是哪个庄子的。说是其他庄子的,人家不敢买。那是因为吃过亏的,问的时候一个劲给你打包票说真是土鸡蛋,买回去打开,一眼的翻白。只有火庄的土鸡蛋货真价实,黄澄澄的不说,价格也合理。今天出活的人家在火庄的西头,看上去家境一般,房屋翻了新,但屋子里却空闹闹的,只有些日常生活必须的物事,看来是屋子翻新耗光了家资。\\

家境虽是一般,可仍旧热闹。这和死去的人有莫大的关系,死者是火庄的老支书。德高望重的老支书躺在堂屋里,安静得像一只睡去的猫。师傅过去恭恭敬敬的上了三炷香。晚饭毕,我们一班人聚在堂屋里,我百无聊赖,把玩着手里的唢呐。师傅则拿出他那支老黄木杆的唢呐不停地擦拭。\\

大师兄把唢呐放进嘴里调音,咕咕唧唧的。师傅说你们都收起来,今天天鸣一个人吹。说完把擦拭好的唢呐递给我。\\

我出奇的惊讶,大师兄更惊讶,连嘴里的唢呐都忘记卸下来了。\\

“为什么?”我问。\\

“他去过朝鲜,剿过匪,带领金庄人修路被石头压断过四根肋骨。”师傅面无表情的说。\\

“百鸟朝凤!”蓝玉一扫慵懒的模样,绷直了问。\\

架势是摆出来了。灵堂前一张宽大的木靠椅,一群孝子俯首跪倒在我面前。所有的人都站在院子里,仰直了脖子往灵堂里看,连一直撒欢的那条老黄狗也规规矩矩的端坐在院子里。\\

我忽然有了一种神圣感,像一个身负特殊使命的斗士。那些眼光让人着迷,在每天来来往往,平淡无奇的生活中,你是看不到这种眼神的。它是那样的干净无邪,仿佛春雨过后山野里散发着的清新气息,又像是冬雪里萦绕在山巅的蒸腾雾霭。\\

师傅站了出来,对着灵堂鞠了三个躬,然后转过身对众人说:\\

“百鸟朝凤,上祖诸般授技之最,只传次代掌事,乃大哀之乐,非德高者弗能受也。”,我知道这几句是《百鸟朝凤》曲谱扉页上的几句话,下面的人是听不懂这几句话的,所以还是一贯的沉默。师傅接着说:“窦老支书我不多说了,他的所作所为金庄人都看在眼里,记在心里,如果无双镇还有人能受得起‘百鸟朝凤’这个曲子的,窦老支书算一个,今天,给窦老支书吹奏送行的,是游家班的班主游天鸣。”。师傅的诚恳让跪倒在我面前的一干人开始发出呜呜的低鸣声。\\

“大哀至圣,敬送亡人,起奏!”师傅高喊。\\

我把唢呐送到嘴里,忽然眼前一片漆黑。\\

直到今天我都活在那段悔恨中,我本可以从容的完成一个乡村乐师所能完成的最高使命,可以让后人提起这段近乎传奇的事件时还能提起我的名字,本可以让乐师这个职业在乡村实现最动人的谢幕演出,甚至可以用一种近于神圣的方式结束我的乐师生涯。可就在那一瞬间,这些可能统统没有了,我的行为让无双镇这个古老的职业用一种异常丑陋的形式完结掉了,连在湮没于时代变化中的最后一刻也未能保持它曾经拥有的尊严。所以,在记录下这段经历的时候,我面临着可怕的记忆煎熬,我感觉我心灵深处的一块被时间慢慢治愈的伤疤又被重新揭开,我清楚的看见它鲜血淋漓,继而是透骨的疼痛。\\

重新睁开眼,一双双焦渴的眼睛全都在看着我。我把唢呐从嘴里慢慢抽出来,站起来对我的师傅说:\\

“对不起大家,这个曲子我忘了!”\\

出人意料,师傅笑了,下面的人也笑了。下面的人还在笑,师傅却哭了,他蹲在地上放声痛哭,我、我的大师兄,还有我的师弟蓝玉,我们站在师傅的身边,谁都不说话。师傅哭了一阵,站起来对还跪在地上的孝子鞠了三个躬,说我们对不起窦老支书,也对不起各位孝子。\\

焦三爷吹一个不就行了!人群中有人建议。\\

师傅摆摆手,说我早就没有这个资格了,这个班子不是焦家班,只有游家班的班主才有这个资格。师傅说完转过身从我手里抢过那支唢呐,抬起膝盖,两手握着唢呐猛力一沉。\\

咔嚓!\\

师傅走了,他迅速消失在了金庄伸手不见五指的黑夜里。\\

蓝玉从地上把断成两截的唢呐拾起来,又看看我,说:“看来我这辈子是听不了百鸟朝凤了!”\\

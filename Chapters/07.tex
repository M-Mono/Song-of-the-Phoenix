\fancyhead[LO]{{\scriptsize 【百鸟朝凤】第七章}} %奇數頁眉的左邊
\fancyhead[RO]{\thepage} %奇數頁眉的右邊
\fancyhead[LE]{\thepage} %偶數頁眉的左邊
\fancyhead[RE]{{\scriptsize 【百鸟朝凤】第七章}} %偶數頁眉的右邊
\fancyfoot[LE,RO]{}
\fancyfoot[LO,CE]{}
\fancyfoot[CO,RE]{}
\chapter*{七}
\addcontentsline{toc}{chapter}{\hspace{11mm}第七章}
%\thispagestyle{empty}
师傅把唢呐递给我。是一支小唢呐,哨子是用芦苇制成的,蕊子是铜制的,杆子是白木的,铜碗的部分则有些斑驳了。我摩挲着它,这支唢呐比蓝玉的要小,但我已经很满足了,我终于吹上唢呐了。我使劲揪了一下大腿,生生的疼。
\\

这是当年我师傅给我的,是我的第一支唢呐。师傅蹲在大门口吸着旱烟说。
\\

别看它个儿小,但是调儿高,唢呐就是这样,调儿越高,个儿就越小。师傅吐出一口烟雾接着说。
\\

我点点头,门口的师傅渐渐就模糊了。
\\

冬天来了,木庄也热闹了。我和我的师弟蓝玉把木庄整天搅得呜呜啦啦的。河湾边,草垛上,还有庄子西边的大青石上,都能听见破烂的唢呐声,破烂的声音主要是我吹出来的,蓝玉吹的唢呐声已经很悦耳了。他吹的时候,过往的木庄人会停下来仔细听一听,听完了就远远的喊说焦家班后继有人了。我则没有这样的待遇,过往的听见我的唢呐声拔腿就跑了,我就和蓝玉哈哈的笑。
\\

师傅很吝啬,每次教给我的东西都少得可怜,一个调子就要我练习十来天。
\\

焦家班又接活了。出门的前一晚,一班人围在火塘边,木桌上还是有苦丁茶和炒黄豆。我和蓝玉一人抱着一支唢呐坐在人群中,血都滚热了。我们终于成为焦家班的一员了,也许要不了多久,我们就可以和师兄们一起到很远很远的地方去了。大家演奏完,大师兄就说两个师弟来的时间也不短了,也该露一手了。我有些怯,因为我吹得实在是不好,就推说让师弟先来吧。蓝玉也不推辞,像模像样的先抖一抖衣袖,两手举着唢呐,往前一推,再徐徐的把哨子凑进嘴里,像一个老练的唢呐手。蓝玉吹奏得确实好,我觉得和师兄们都差不多了。他演奏的是一段喜调,曲子轻快的在屋子里跳跃,他脑袋和调子一起左摇右晃的,吹得一屋子喜气洋洋。吹奏完了,大师兄就摸蓝玉的大脑袋,说不得了不得了,其他师兄也说好,只有师傅不说话,大口大口的吸烟。
\\

蓝玉吹完了,一屋子人都看着我,我的心突突的跳,握着唢呐的手也浸出好多的汗来。二师兄对着我点点头,我知道他是鼓励我。我战战抖抖的把唢呐塞进嘴里,呜呜的憋出几个滑音和颤音,然后我低下头,说我就会这点了。
\\

一屋子都无话了,只有油灯在轻轻的跳动。师兄们都神情肃穆的看着师傅,师傅还是低着头吸烟。好半天二师兄才低低的对师傅说,师傅恭喜您了。师傅把旱烟伸到凳子腿上按熄说好了今天就到这里,散了吧,明天还要赶远路呢!\\

我不知道二师兄为什么要恭喜师傅,我吹得那样烂,这样久了也只会吹一些基本的音调,师傅还一副不依不饶的样子,每天就只要我钉着几个调儿吹。
\\

就几个调,我把冬天吹来了。
\\

今年的第一场雪总算来了,都孕育了好几天了,直到昨夜才落下来。半夜我和蓝玉都听见了雪花滑过窗棂的声音。我和蓝玉都睡不着。我们睡不着倒不是等这场雪。在黑夜里大大的睁着眼睛,是等天亮后激动人心的一刻。昨天晚上,焦家班围在火塘边奏完最后一曲调子后,师傅对大家说:明天天鸣和蓝玉也和我们一起出门吧!
\\

蓝玉推开窗户对我说,落雪了,不知道我们木庄是不是也落雪了呢?我说我们水庄肯定是落雪了的,每年这个时候,雪落得可大了,漫天遍野的飞,一个庄子都陷下去了。
\\

我起得很早,草草的抹了一把脸,小心翼翼的把唢呐装好。我装唢呐的布袋子是师娘缝的,碎花青布,唢呐刚好能放进去,可熨帖了;蓝玉的唢呐也有布袋子,是藏青棉布缝制的,后来我才发现,装蓝玉唢呐的布袋子的前身是师傅的内裤。这个秘密我一直没有给蓝玉讲,再后来我又发现,我的布袋子是师娘贴肉的裤衩改的。
\\

今天要去的人家请的是白事。我刚装好唢呐,接客就到了。来接唢呐的是两个年轻人,比我和蓝玉大不了多少,嘴边刚刚长出来一些茸毛,他们一人背着一个背篼,怯生生的站在院子边。我们无双镇就是这样的,请唢呐要派接客,接客要负责运送唢呐匠的工具,等活结束了,还得送回来。
\\

很快我的七个师兄就到了,看来主人请的是八台,七个师兄加上师傅刚好八个。我和蓝玉当然还不能上阵,蓝玉其实是够了的,但师傅说了,先跟一段再说。两个接客很麻利的把锣啊鼓啊的全装进背篼,看我和蓝玉怀里还抱着唢呐,就伸过手来说都装上吧。我不让,说自己拿就成了,反正也不重的。接客不让,说哪有唢呐匠自己拿东西的道理,我们金庄没有这规矩,无双镇也没有这规矩。我还想推让,师傅在旁边说,给他吧,不依规矩,不成方圆。
\\

主人姓查,金庄漫山遍野散落的人家差不多都姓查。
\\

我们被安排进一个单独的屋子,屋子很紧凑,还有两个炭火盆。屁股还没有坐热,师傅就对大家说:“捡家伙,开锣!”。说完就往院子里去了。\\

我终于能亲眼目睹唢呐匠们正儿八经的八台大戏了。焦家班在院子里呈扇形散坐着,师傅居于正中,他的目光左右扫视了一番,众人会意,齐齐进入了状态。一声锣响,焦家班在金庄的唢呐盛会拉开了序幕。我此时听到的唢呐声和昨天晚上听见的预演有极大的差别,师傅和他的一班弟子个个全神贯注。唢呐声在高旷的天地间奔突。先是一段宏大的齐奏,低沉而哀婉;接着是师傅的独奏,我第一次听到师傅的独奏,那些让人心碎的音符从师傅唢呐的铜碗里源源不断的淌出来,有辞世前的绝望,有逝去后看不清方向的迷惘,还有孤独的哀叹和哭泣。尤其是那哭声,惟妙惟肖。一阵风过来,撩动着悬在院子边的灵幡,也吹散了师傅吹出来的哀号,天地间陡然变得肃杀了。
\\

一直在院子里劳作的人群过来了,没有人说话,目光全在师傅的一支唢呐上。渐渐有了哭声,哭声是几个孝子发出来的。没多久,哭声变得宏大了,悲伤像传染了似的,在一个院子里弥漫开来,那些和死者有关的,无关的人,都被师傅的一支唢呐吹得泪流满面。
\\

一曲终了,有人递过来一碗烫热的烧酒,说焦师傅,辛苦了,润润嗓子吧。
\\

开过晚饭,主人过来了。先是眼泪汪汪的给师傅磕了一个头。说这冰天雪地的你们还能赶过来送我老爹一程,我谢谢你们了。
\\

“他生前是我们查家的族长,可德高望重了!”主人爬起来说。
\\

师傅点点头。
\\

“做了不少好事,我都数不过来。”主人又说。
\\

师傅又点点头。
\\

“焦师傅,你受累,看能不能给吹个百鸟朝凤?”主人把脑袋伸到师傅面前问。
\\

师傅摇摇头。
\\

“钱不是问题!”\\

师傅还是摇摇头。
\\

磨了好一阵子,师傅除了摇头什么都不说。主人无奈,只好叹着气走了,走到门口又心有不甘的回头问:“我老爹真没这个福气?”。师傅抬起头说你去忙吧!
\\

主人走了,二师兄看着师傅说:“师傅,查老爷子德高望重呢!”。师傅的鼻腔哼了哼:“知道查姓为什么是金庄第一大姓吗?以前的金庄可不光是查姓,都走了,散到无双镇其他地头去了,这就是查老爷子的功劳!”。
\\

接下来几天,我和蓝玉就进天堂了。顿顿有肉吃,其间我和蓝玉还偷喝了烧酒,焦家班坐到院子里吹奏的时候,我还和蓝玉躲在屋子里抽烟,烟是主人家偷偷塞给我们的,我和蓝玉本来是不收的,可主人家不干,非得塞给我们。
\\

离开那天,死者的几个儿子把焦家班送出好远,临了就把一沓钱塞给师傅,师傅就推辞,结果两个人在分手的桥上你来我往的斗了好几个回合,师傅才很勉强的把钱收下来。
\\

几个师兄则站在一边木木的看着,眼神倦怠,眼前这个场景他们已经看够了。\\
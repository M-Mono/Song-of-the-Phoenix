\fancyhead[LO]{{\scriptsize 【百鸟朝凤】第十三章}} %奇數頁眉的左邊
\fancyhead[RO]{\thepage} %奇數頁眉的右邊
\fancyhead[LE]{\thepage} %偶數頁眉的左邊
\fancyhead[RE]{{\scriptsize 【百鸟朝凤】第十三章}} %偶數頁眉的右邊
\fancyfoot[LE,RO]{}
\fancyfoot[LO,CE]{}
\fancyfoot[CO,RE]{}
\chapter*{十三}
\addcontentsline{toc}{chapter}{\hspace{11mm}第十三章}
%\thispagestyle{empty}
马家大院看上去比五年前阔多了,楼房像个长个子的娃,几年光景就多出了三层。马家在木庄都习惯领跑了,还把后面的拉下一大截。老马家两层小平房起来了,木庄其他人家还在茅草屋子里忍饥挨饿,好不容易有了两层小平房,一瞧,老马家都五层了。木庄人总是在老马家屁股后面,怎么跑都跑不过。个中缘由除了老马脑筋好用以外,最主要的是老马有四个身强力壮的男娃子。几个娃出门早,据说中国的大城市都有他们的脚印。\\

可惜精打细算的老马还是耗不过病痛,六十不到的人,年前还背着手在木庄的石板路上检阅风景,年后就蹬腿了。四个儿子回来奔丧,每个人都有一辆小汽车,十六个轮子一码子停靠在木庄的石板街上,成了木庄人眼里一道稀有而复杂的风景。\\

游家班在马家大院里呈扇形散开。八台,也当然是八台。烟酒茶照例是不能少的,还有黄澄澄的糕点,放进嘴里又软又酥,上下颚一合拢,就化掉了。几个师兄都兴奋的交谈着,连平时话最少的三师兄都停不下口,他慌乱的说话,慌乱的把好吃的东西往嘴里扔,好几次该他的锣声响起了,他都还在为他那张嘴在奋斗。我有些火了,吼了他两声,没多久又听不见他的锣声了。\\

我忽然好惶恐。从我们进到马家大院起,好像就没有人关注过这几支呜呜啦啦的唢呐,我开始以为是大家不卖力,白了他们几眼,大家精神就抖擞不少,大师兄两个眼珠子都要给吹飞出来了,可对我们的处境仍没多少改善。人们依旧在院子里穿梭,小孩子依旧在院子里打闹,就是没人看我们。其间还有人碰倒了二师兄脚边的酒瓶子,白酒汩汩的往外流,那人像没看见一样,径直就去了。\\

我正要伸手去扶酒瓶子,眼睛就什么都看不见了。\\

猜猜,我是谁?\\

不用猜我就知道是他,我的师弟蓝玉。他的手粗壮了不少,声音也变得厚实了,嗓子也由男孩儿的蜕变成男人的了。\\

我的眼睛一下就潮湿了,其实我早看见他了的,混在来来往往的人群里,一件红色的外套招招摇摇。他的眼睛还不时的往游家班这边瞟,我没敢过去和蓝玉相认,不知道是没有相认的勇气还是其他的什么原因。\\

我的师弟蓝玉早就看见我们了,他一直没有过来,我想他不会过来了。\\

但现在他却蒙住了我的双眼,让我猜他是谁。\\

蓝玉惊慌的松开了手,惊讶的看着两只手掌中的潮湿,又抬起头看着我的眼睛,忽然他的眼泪也下来了。我和蓝玉面对面站着,我们差不多一样高,他嘴角的胡须比我的要茂盛,身子却比我瘦弱一些。\\

我忽然有了拥抱蓝玉的冲动,那种感觉热乎乎的。好多年前我们家有一条狗,黄毛,短耳朵,有一天突然不见了,刚不见的那几天还会想想它,慢慢的就忘掉了。大约过了两个月,那条狗出现在了我家院子里,一身泥污,一条腿还折了,两只眼睛弥漫着哀伤和委屈。那时候我也是这种热乎乎的感觉,跑过去抱着狗流了一回泪。\\

我看着蓝玉,蓝玉也看着我,我们谁都没有动。\\

师弟!我喊了一声。\\

蓝玉走过来,捶了我一拳。\\

“你有丢过狗的经历吗?”我问蓝玉。\\

“有,丢了整整十年!”蓝玉说。\\

几个师兄的唢呐一下嘹亮起来。\\

晚上蓝玉没有回家,一直陪着我们。喝酒、吹牛、抽烟。\\

下半夜,几个师兄都去睡觉了,人群也大多散去了。我和蓝玉坐在院子里,我把唢呐递给他,说来一调,蓝玉兴致勃勃的把唢呐接过去,苇哨刚送进嘴里又抽出来了。他把唢呐还给我,为难的笑笑说算了吧!好多年没吹了,调子都忘记了。我也笑笑说你那脑袋,十分钟就能把调调找回来。蓝玉拿来两个碗,倒了满满两海碗烧酒,我们就开始喝,一直喝到月亮下去,漫天的红霞上来,没有一点睡意。\\

这么多年来,蓝玉那晚说过的话我基本都记得。甚至他说话时的每一个表情,歪脑袋,大幅度的点头,掏耳朵等等这些细节都还在我的脑海里。比如他说当年离开土庄的时候,我一个人像条野狗一样,茫然的在田间小路上走,连死的心都有了。讲到这里他就把脑袋夸张的往下缩,等脑袋落到肩上了我才听见他喉咙里出来的那声浑浊的长叹;还有他说其实我不怪师傅,师傅让我回家是对的,要换了我,无双镇的唢呐班子早没了,我性子野,干啥都守不了多久,总会有些稀奇古怪的想法。讲到这里蓝玉的脖子忽然伸得老长,都快顶着头上那片红云了,他还呵呵的笑,笑完就猛灌下去一大口烧酒,脸也成了天边的颜色。\\

我的生命里有很多的变化,这些变化就像天气一样的让人琢磨不定,但每次变化之前又隐隐约约的看得见一些预兆。下雨之前是一定要乌云密布的,太阳带晕了,接踵而至的就是干旱,月亮带晕了,那说明接下来就该是一场连绵不绝的细雨时节了。那个木庄的夜晚,我和我的师弟蓝玉十年后相遇了,我们还有了一次酣畅淋漓的谈话,这场谈话让我隐隐的看到,也许,我的命运又到了拐角的地段了。\\

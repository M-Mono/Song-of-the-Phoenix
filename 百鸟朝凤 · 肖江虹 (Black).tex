\documentclass[oneside,openright,headings=optiontohead]{scrbook}
\renewcommand{\baselinestretch}{1.3}  %行間距倍率


\usepackage[
a4paper=true,
%CJKbookmarks,
unicode=true,
bookmarksnumbered,
bookmarksopen,
hyperfigures=true,
hyperindex=true,
pdfpagelayout = SinglePage,
pdfpagelabels = true,
pdfstartview = FitV,
colorlinks,
pdfborder=001,
%linkcolor=black,
%anchorcolor=black,
%citecolor=black,
linkcolor=TEXTColor,
anchorcolor=TEXTColor,
citecolor=TEXTColor,
pdftitle={百鸟朝凤},
pdfauthor={肖江虹},
pdfsubject={百鸟朝凤},
pdfkeywords={小说《百鸟朝凤》聚焦贵州修文县农村一支民间唢呐乐班的际遇,描写了两代唢呐艺人以及唢呐这种民间艺术形式在现代化的挤压下正逐渐消亡的过程。},
pdfcreator={https://m-mono.github.io}
]{hyperref}


%Kindle Voyage, Paperwhite (3rd gen), Oasis: 6 in diagonal, 1448 × 1072 pixels @ 300 PPI 
\usepackage[papersize={9cm,12.2cm}]{geometry} % Kindle Format
\geometry{left=1.3cm,right=1.3cm,top=2cm,bottom=1cm,foot=4cm}
\usepackage{graphics,graphicx,pdfpages}

%自动加注拼音
\usepackage{xpinyin}
\xpinyinsetup{format={\color{PinYinColor}}}
%手动加注外语及日语振假名
\usepackage{ruby}
\renewcommand\rubysize{0.4} %匹配 xpinyin 默认标注字体大小
\renewcommand\rubysep{-0.3em} %匹配 xpinyin 默认标注高度

\usepackage{xeCJK}
\usepackage{indentfirst}
\setlength{\parindent}{2.0em}

%正文字体
\setCJKmainfont[Path=Fonts/]{WenYue-GuDianMingChaoTi-NC-W5.otf}
\setCJKsansfont[Path=Fonts/]{WenYue-GuDianMingChaoTi-NC-W5.otf}
\setCJKmonofont[Path=Fonts/]{WenYue-GuDianMingChaoTi-NC-W5.otf}
\setmainfont[Path=Fonts/]{WenYue-GuDianMingChaoTi-NC-W5.otf}
\setsansfont[Path=Fonts/]{WenYue-GuDianMingChaoTi-NC-W5.otf}
\setmonofont[Path=Fonts/]{WenYue-GuDianMingChaoTi-NC-W5.otf}
%拼音字体
\newfontfamily{\Pinyin}[Path=Fonts/]{SourceSansPro-Regular.otf}


% 頁面及文字顏色
\usepackage{xcolor}
\definecolor{TEXTColor}{RGB}{130,130,130} % TEXT Color
\definecolor{PinYinColor}{RGB}{100,100,100} % TEXT Color
\definecolor{NOTEXTColor}{RGB}{130,130,130} % No TEXT Color
\definecolor{BGColor}{RGB}{0,0,0} % BG Color
\pagecolor{BGColor}
\color{TEXTColor}

\makeindex
\renewcommand{\contentsname}{百鸟朝凤}
\usepackage{fancyhdr} % 設置頁眉頁腳
\pagestyle{fancy}
\renewcommand{\headrulewidth}{0pt}  %頁眉線寬,設為0可以去頁眉線
\renewcommand{\footrulewidth}{0pt}  %頁眉線寬,設為0可以去頁眉線

\usepackage{titletoc}
\dottedcontents{section}[100em]{\bfseries}{100em}{100em} % 去掉目录虚线


\begin{document}
	\frontmatter
	\begin{figure}[ht]
		\begin{center}
			\includepdf[height=\paperheight,width=\paperwidth]{Frontmatter.jpeg}
		\end{center}
	\end{figure}
	\newpage
	{\color{TEXTColor}
		{\Huge 索引}\\
	
	\begin{quote}
		【词目】 百鸟朝凤
		\\
		
		【发音】 {\Pinyin{bǎi niǎo cháo fèng}}
		\\
		
		【释义】 朝:朝见;凤:凤凰,古代传说中的鸟王。旧时喻指君主圣明而天下依附,后也比喻德高望重者众望所归。
		\\
		
		【出处】 宋·李昉等《太平御览》九百一十五卷引《唐书》:“海州言凤见于城上,群鸟数百随之,东北飞向苍梧山。”\\
	\end{quote}
	\newpage
		\tableofcontents
		\newpage
		\mainmatter
			\fancyhead[LO]{{\scriptsize 【百鸟朝凤】第一章}} %奇數頁眉的左邊
\fancyhead[RO]{\thepage} %奇數頁眉的右邊
\fancyhead[LE]{\thepage} %偶數頁眉的左邊
\fancyhead[RE]{{\scriptsize 【百鸟朝凤】第一章}} %偶數頁眉的右邊
\fancyfoot[LE,RO]{}
\fancyfoot[LO,CE]{}
\fancyfoot[CO,RE]{}
\chapter*{一}
\addcontentsline{toc}{chapter}{\hspace{11mm}第一章}
%\thispagestyle{empty}
过了河,父亲再一次告诫我,说不管师傅问什么,都要顺着他,知道吗?我点点头。父亲蹲下来给我整了整衣衫,我的对襟短衫是母亲两个月前就做好的,为了让我穿上去看起来老成一些,还特地选了藏青色。直到今天离开家时,母亲才把新衣服给我换上。衣服上身后,父亲不满意,蹙着眉说还是没盖住那股子嫩臭味儿。看起来藏青色的短衫并没有拉长我来到这个世界上的日子。毕竟我才十一岁,这个年龄不比衣服,过过水就能缩短或抻长的。
\\

一大早被母亲从床上掀下来的时候,还看见她一脸的怒气,她对我睡懒觉的习惯深恶痛绝。可临了出门,母亲的眼神里却布满了希冀、不舍,还有无奈。父亲则决绝得多,他的理想就是让我做个唢呐匠。我们水庄是没有唢呐匠的,遇上红白喜事,都要从外庄请,从外庄请也不是容易的事情,如果恰好遇上人家有预约,那水庄的红白喜事就冷清了。没有了那股子活泛劲头,主人面子上过不去,客人也会觉得少了点什么。所以被请来的唢呐匠在水庄都会得到极好的礼遇,烟酒茶是一刻不能断的,还得开小灶。离开那天,主人会把请来的唢呐匠送出二里多地,临别了还会奉上一点乐师钱,数量不多,但那是主人的心意。推辞一番是难免的,但最后还是要收下的。大家都明白这是规矩,给钱是规矩,收钱是规矩,连推辞都是规矩的一部分。
\\

听母亲说,父亲想让我做一名唢呐匠其实并不完全为了钱。母亲说父亲年轻时也想做一名唢呐匠,可拜了好多个师傅,人家就不收,把方圆百里的唢呐匠师傅都拜遍了,父亲还是没有吹上一天的唢呐,人家师父说了,父亲这人鬼精鬼精的,不是吹唢呐的料。许多年过去了,本以为时间已经让父亲的理想早就像深秋的落叶腐化成泥了,可事实并不是这样。自我懂事起,我就发现父亲看我的眼神变得怪怪的,像蹲在狗肉汤锅边的饿痨子,摩拳擦掌,跃跃欲试。有一次,我的老师在水庄的木桥上遇见了父亲和我,他情绪激动地给父亲反映,说我从小学一年级到五年级,数学考试从来没有超过三十分。我当时就羞愧地低下了头,想接下来理所当然的有一场暴风骤雨。老师说完了,父亲点点头,很大度的挥挥手说三十分已经不错了。然后牵起我走了。走到桥下,他回头看了一眼身后可怜的一头雾水的教书匠,嘿嘿干笑了两声,教书先生哪里知道,水庄的游本盛对他儿子有更高远的打算。
\\

我确实不喜欢念书,我们水庄大部分娃子和我一样不喜欢念书,刚开始还行,渐渐的就冷了。主要是听不懂,比如我们的数学老师,自己都没有一个准,今天给我们一个答案,明天一早站在教室里又小声的宣布,说同学们昨天我回去在火塘边想了一宿,觉得昨天那个题目的答案有鬼,不正确,所以吓得一夜都没睡安稳,今天特地给大家纠正。我们就笑一回,后来又听说数学老师其实也只是个小学毕业的,更有甚者说他根本连小学都没有读毕业。我们就无可奈何的生出一些鄙夷来。鄙夷的方式就是不上课,漫山遍野的去疯。
\\

我不喜欢念书,可我也不喜欢做唢呐匠,我也说不清为什么不喜欢作唢呐匠,可能是从小到大总听见父亲在耳边灌输唢呐匠的种种好,听得多了,也腻了,就厌恶了。而且我断定,我的父亲之所以希望我成为一个吹唢呐的,目的就是图那几个乐师钱。
\\
			\fancyhead[LO]{{\scriptsize 【百鸟朝凤】第二章}} %奇數頁眉的左邊
\fancyhead[RO]{\thepage} %奇數頁眉的右邊
\fancyhead[LE]{\thepage} %偶數頁眉的左邊
\fancyhead[RE]{{\scriptsize 【百鸟朝凤】第二章}} %偶數頁眉的右邊
\fancyfoot[LE,RO]{}
\fancyfoot[LO,CE]{}
\fancyfoot[CO,RE]{}
\chapter*{二}
\addcontentsline{toc}{chapter}{\hspace{11mm}第二章}
%\thispagestyle{empty}
翻过大阴山,就能看见土庄了。那就是我未曾谋面的师傅的家。我们这一带有五个庄子,分别叫金庄、木庄、火庄,土庄,再加上我们水庄,构成了一个大镇,按理这个镇子该叫五行镇才对的,可它却叫无双镇。未来师傅的宅子在一片茂盛的竹林中,翠绿掩映下的一栋土墙房。我曾经从爷爷的旧箱子里翻出一本绣像《三国演义》,里面有一幅画,叫三顾茅庐的,眼前的这个场景就和那幅画差不多。通往土墙房的路一溜的坦途,可父亲却发出吭哧吭哧的喘气声,他额头上还有针尖大小的汗珠儿,两个拳头紧紧的握着。我看了他一眼,父亲有些不好意思起来,他想我定是把他的紧张看破了,于是他就露出一个自嘲的讪笑。\\

面子有些挂不住的父亲就转移话题。福地啊!父亲说,你看,左青龙,右白虎,后朱雀,前玄武,一看就不是一般人家。我想笑,可没敢笑出来,父亲是不识风水的,连引述有关风水的俗语都弄错了。这几句我也是听水庄的风水先生说过,不过人家说的是前朱雀,后玄武。我想父亲真的是太紧张了,他怕自己小时候的悲剧在下一代的身上重演。我顿时有了一些报复的快感,想师傅要是看不上我就好了,最好是出门了,还是远门,一年半年的都回不来。
\\

看见我左摇右晃的二流子步伐,父亲在身后焦急的吼,天杀的,你有点正形好不好!师傅看见了那还了得。
\\

父亲的运气比想象的要好,木庄名声最显赫的唢呐匠今天正好在家。
\\

我未来师傅的面皮很黑,又穿了一件黑袍子,这样就成了一截成色上好的木炭。他从屋子里踱出来的时候燃了一袋旱烟,烟火吱吱的乱炸。我很紧张,怕那点星火把他自己给点燃了。他大约是看出了我的焦虑,就抬起一条腿,架到另一条腿的膝盖上,把鞋底对着天空,将那半锅子剩烟杵灭了。做这样一个难度很大的动作只是为了杵灭一锅烟火,看来我未来的师傅真是一个不简单的人。
\\

焦师傅,我叫游本盛,这是我儿子游天鸣,打鸣的鸣,不是明白的明。父亲弓着腰,踩着碎步向屋檐下的黑脸汉子跑过去,跑的过程中又慌不迭的伸手到口袋里摸香烟,眼睛还一直对着一张黑脸行注目礼。可怜的父亲在六七步路的距离里想干的事情太多了,他又缺乏应有的镇定,这样先是左脚和右脚打了架,接着身体就笔直的向前仆倒,跌了一嘴的泥,香烟也脱手飞了出去,不偏不倚的降落在院子边的一个水坑里。我的心一紧,赶忙过去把父亲扶起来,父亲甩开我扶他的手,说扶我干什么?快去给师傅磕头啊!我没有听父亲的,毕竟我认识父亲的时间比认识师傅的时间要长,于情于理都该照看刚从地上爬起来的水庄汉子。主意打定,我仍然不屈不挠的挽着父亲的手臂,我抬起头,父亲的额头上有新鲜的创口,殷红的血珠正争先恐后的滲出来,我一阵心酸,眼泪就下来了。
\\

师傅摆摆手,说磕头?磕什么头?他为什么要给我磕头?这个头不是谁都能磕的。
\\

父亲哑然,很难堪的从水坑里捡起香烟,抽出一支来,香烟身体暴涨,还湿嗒嗒的落着泪。
\\

这?父亲伸出捏着香烟的手为难地说。
\\

屋檐下的扬了扬手里的烟锅子说,我抽这个。
\\

我、父亲,还有我未来的黑脸师傅,三个人就僵立着,谁都不说话,主要是不知道说什么。还是屋檐下的木炭坦然,不管怎么说这始终是他的地盘,所以他的面目始终都处于一种松弛的状态,他看了看天空,我也看了看天空,他肯定觉得今天是个好天气,我也觉得今天是个好天气。太阳像个刚煎好的鸡蛋,有些耀眼,我未来的师傅就用手做了一个凉棚,看了一会儿太阳,又缓慢地填了一锅烟,把烟点燃后,他终于开口了。
\\

哪个庄子的?他问话的时候既不看我,也不看父亲,但父亲对他的傲慢却欣喜如狂。父亲往前走了两步,说水庄的,是游叔华介绍过来的。父亲把游叔华三个字做了相当夸张的重音处理。游叔华是我的堂伯,同时也是我们水庄的村长。
\\

我听见唢呐匠的鼻子里有一声细微的响动,像鼻腔里爬出来一个毛毛虫。他继续低头吸烟,仿佛没有听见父亲的话。看见游村长的名号没有收到想象中的震撼力,父亲就沮丧了。
\\

多大了?唢呐匠又问。
\\

我的嘴唇动了动,刚想开口,父亲的声音就响箭般的激射过来:十三岁。比我准备说的多出了两岁。怕唢呐匠不相信,父亲还做了补充:这个月十一就十三岁满满的了。
\\

唢呐匠的规矩你是知道的,十三是个坎。唢呐匠说。
\\

知道知道。父亲答。
\\

这娃看起来不像十三的啊。唢呐匠的眼睛很厉害。
\\

这狗东西是个娃娃脸,自十岁过来就这样儿,不见熟。
\\

嗯!唢呐匠点了点头。看见唢呐匠表了态,父亲的眉毛骤然上扬,他跑到屋檐下战战抖抖的问:您老答应了?
\\

哼!还早着呢!
\\

我原本以为做个唢呐匠是件很容易的事情,拜个师,学两段调儿,就算成了,可照眼下的情形来看,道道还真不少呢。
\\

院子里摆了一张桌子,桌子上放了一个盛满水的水瓢,水瓢是个一分为二的大号葫芦。唢呐匠递给我一根一尺来长的芦苇杆,我云里雾里的接过芦苇杆,不知道唢呐匠到底什么用意。
\\

用芦苇杆一口气把水瓢里的水吸干,不准换气。我未来的师傅态度严肃的对我说。
\\

我看了看父亲,父亲对着我一个劲的点头,牙咬得紧紧的,他的鼓励显得格外的艰苦卓绝。
\\

我把芦苇杆伸进水里,又看了看他们两个人,唢呐匠的眼神和父亲形成了鲜明的对比,自然而平静,像我面前的这瓢水。
\\

我提了提气,低头把芦苇杆含住,然后一闭眼,腮帮子一紧,一股清凉顿时排山倒海的涌向喉咙。我睁开眼,看见瓢里的水正急速的消退,开始我还信心满满的,等水消退到一半的时候,气就有些喘不过了,水只剩下三分之一的时候,不光气上不来,连脑袋也开始发晕了,胸口也闷的难受,我像就要死了。
\\

快,快,快,不多了。是父亲的声音,像从天外传来的。
\\

终于,我一屁股坐倒在地,仰着头大口的喘气,我又看见太阳了,是个煎糊的鸡蛋。
\\

等太阳重新变成黄色,我听见父亲在央求唢呐匠。
\\

您老就收下他吧!父亲带着哭腔说。
\\

他气不足,不是做唢呐匠的料子。
\\

他气很足的,真的,平时吼他两个妹妹的声音全水庄都能听见。
\\

唢呐匠笑笑,不说话了。
\\

这时候我看见父亲过来了,他含着眼泪,咬牙切齿的操起桌上的水瓢,劈头盖脸的向我猛砸下来。
\\

你个狗日的,连瓢水都吸不干,你还有啥能耐?水瓢正砸在我脑门上,我听见了骨头炸裂的声音。我高喊一声,仰面倒下,太阳不见了,只有一些纷乱的蛋黄,还打着旋的四处流淌。
\\

怎么样?他叫的声音够大吧?气足吧?父亲的声音怪怪的,阴森潮湿。
\\

我努力睁开眼,又看见了父亲高高扬起的水瓢。
\\

叫啊!大声叫啊!父亲喊。
\\

我不知道父亲为什么要这样。我做不成唢呐匠怎么会令他如此气急败坏。
\\

正当我万分惊惧的时候,我看见了一只手。
\\

那只手牢牢攥住了父亲的手腕。\\
			\fancyhead[LO]{{\scriptsize 【百鸟朝凤】第三章}} %奇數頁眉的左邊
\fancyhead[RO]{\thepage} %奇數頁眉的右邊
\fancyhead[LE]{\thepage} %偶數頁眉的左邊
\fancyhead[RE]{{\scriptsize 【百鸟朝凤】第三章}} %偶數頁眉的右邊
\fancyfoot[LE,RO]{}
\fancyfoot[LO,CE]{}
\fancyfoot[CO,RE]{}
\chapter*{三}
\addcontentsline{toc}{chapter}{\hspace{11mm}第三章}
%\thispagestyle{empty}
好多年后师傅对我说,你知道当初我为什么收你为徒吗?我说你老人家心善,怕我父亲把我给活活打死了。师傅摇头,说你错了,我收你为徒是因为你的眼泪。我说什么眼泪?师傅说你父亲跌倒后你扶起他后掉的那滴眼泪。
\\

父亲走了,看着他离开的背影我顿时有一种无助的感觉,以往天天看见他,没觉得他有多重要,被他揍了还会在心里偷偷骂“狗日的游本盛”。现在才发现父亲原来是极重要的。他就像一棵树,可以挡风遮雨,等有一天自己离开了这棵大树,才发现雨淋在身上是冰湿的,太阳晒在脸上是烤人的。\\

从此以后,我就是一个人了。看着父亲渐渐变淡变小的背影,我忍不住哭了一场,师傅站在我旁边,伸出一只手搭在我的肩上,轻轻拍了拍,我心里一热,哭得更厉害了。
\\

晚上吃饭,师傅给我介绍了师娘,师娘很瘦,也黑。走起路来左摇右晃的,像根煮熟的荞麦面条。师娘话多,饭桌上问了我好多事情,都是关于水庄的,还说她有个亲戚就住在我们水庄。和师娘比起来,师傅的话则少了许多,一顿饭时间就说了两句话,我端碗的时候他说:吃饭。我放碗的时候他又说:吃饱。
\\

吃完饭,我主动把碗刷了。在刷碗的过程中我偷偷探头看了看坐在堂屋里的师傅和师娘,当时师娘对着我站的位置指指点点,还不住的点头,脸上也有些不易觉察的笑容。师傅却不为所动,他只是一个劲的抽烟,喷出来的烟雾也浓,让我想起在水庄和父亲烧山灰的日子。我明白师娘的笑容和我刷碗的行动有关。而我刷碗的行动又和临出门那晚母亲油灯下的唠叨有关。母亲说:出门在外不比在家,要勤快,眼要尖,要把你那根全是懒肉的尾巴夹好。
\\

刷完碗师娘对我说,她的三个儿子都成家分出去了,家里就他们两老,所以你该做些力所能及的事情。
\\

晚上我躺在床上,想明天就要吹上唢呐了,有一些兴奋,又有一些惶恐,总觉得我的人生不该就这样拐弯的,我还没有玩够,我还是个娃儿,娃儿就该玩的。想起我的伙伴马儿他们,此刻他们肯定正在水庄的木桥边抓萤火虫,把抓来的萤火虫放进透明的瓶子里,走夜路时可以当马灯用。
\\

一早,我还在梦里捉萤火虫,就听见了两声剧烈的咳嗽声,咳嗽声是师傅发出来的,我一惊,知道这是起床的信号,师傅毕竟不是亲爹,没有像父亲一样冲进来掀开被窝照着屁股就一顿猛扇。我想他一定还当我是客人,所以方式也就间接一些。穿上衣服走出门,我先喊了一声站在屋檐下的师娘,正在淘蚕豆的师娘对我点了点头。打完一个呵欠我才发现太阳还在山那头浴血挣扎,我心里头就上来了一些怨气,想这太阳都还没有出来呢,就得爬起来。在家虽然被父亲扇屁股,但那时太阳都老高了啊。看见我脸嘴不好看,师娘说你师傅到河湾去了,你也去吧!
\\

顺着师娘指的方向,我看见了木庄的河湾,木庄虽然叫木庄,可河湾却比水庄的还要大,河岸四周有烟柳,烟柳我们水庄也有,远远的看去像团滚圆的烟。烟柳四四方方的抱着一团翠绿的河湾,几只纯白的水鹤在河湾上悠闲的飞来绕去。师傅站在河滩上,静静的看着水面,他的身影很孤寂,也渺小。
\\

师傅从河岸边齐根折来一根芦苇,去掉顶端的芦苇须,把足有三尺长的芦苇杆递给我,说过去把河里的水吸上来,记住,芦苇杆只能将将伸到水面。开始我以为这是件极简单的事情,一吸我才知道没有那么简单。我脸也红了,腿也软了,小肚子都抽筋了,还是没能吸上一滴水。我回头看了看师傅,师傅脸色灰暗,说等你把水吸上来了就可以回家了。
\\

天黑尽了我才回到师傅家,师傅和师娘守着一盏如豆的油灯。看我进屋来,师娘端给我一碗饭,饭还没到我手里,师傅说话了。
\\

水吸上来了?
\\

我摇摇头。
\\

那你回来搓球啊?师傅猛地立起来,把手里的旱烟杆往地上狠狠的一掼。他的脸本来就乌黑,此刻就更黑了。
\\

我现在才意识到这个黑脸男人是认真的。
\\

我的晚饭被师傅扒掉了半碗,虽然师娘一直给我说情,说天鸣他爹可是交足了生活费用的,再说娃儿在吃长饭呢!
\\

娃?老子哪个徒弟不是娃过来的?老子当初拜师的时候,三天没有饭吃呢!
\\

夜晚我躺在床上痛快的哭了一回,哭完了就想父亲的绝情,想完父亲的绝情又想母亲的好。想着想着就睡着了,睡着好像没多久又听见了咳嗽声。我爬起来凑到窗户边,发现山那边连太阳浴血的迹象都还没有。
\\

此后十多天,我天天攥着根芦苇杆在河滩上吸水。有往来的土庄人隔得远远的就喊,焦三爷又收新徒弟了。还有的喊,这个娃子能成焦三爷的弟子,看来是有些能耐的。我听见他们的喊声里有酸溜溜的味道,肯定是自己的娃没能让师傅看上。这样我有了一些信心,就把吸水这个世间最枯燥的活儿有模有样的干起来。
\\

大约是一个黄昏,我记得那天河滩上的水鹤特别多,沿着水面低低的滑翔,在一片耀眼的绿中拉出一尾又一尾炫目的雪白。我像之前千百次的吸水一样,一沉腰,一顿足,一提气,竟然牢牢的咬住了一股冰凉。我把嘴里的水来回渡了渡,又把它轻轻的吐到掌心里,不错的,我把水吸上来了。看着掌心的一窝清澈,我恍若隔世,一股说不清道不明的东西在心窝子里上下翻滚,喉咙慢慢就变得硬硬的了。我撒腿疯了似的向师傅的土墙小屋子跑去,跑到院子里,师傅正坐在屋檐下编苇席。
\\

吸上来了。我一字一顿的说。
\\

本来以为师傅会笑一个,然后点点头,说这下你可以吹上唢呐了。但不是这样的。师傅听我说完,从脚边堆积的芦苇里挑出一根最长的,掐头去尾递给我。我把芦苇杆立起来,比我还要高,我疑惑地看着师傅,师傅依然认真地低头编着苇席,半晌才抬起头对我说,去啊!继续吸。
\\

			\fancyhead[LO]{{\scriptsize 【百鸟朝凤】第四章}} %奇數頁眉的左邊
\fancyhead[RO]{\thepage} %奇數頁眉的右邊
\fancyhead[LE]{\thepage} %偶數頁眉的左邊
\fancyhead[RE]{{\scriptsize 【百鸟朝凤】第四章}} %偶數頁眉的右邊
\fancyfoot[LE,RO]{}
\fancyfoot[LO,CE]{}
\fancyfoot[CO,RE]{}
\chapter*{四}
\addcontentsline{toc}{chapter}{\hspace{11mm}第四章}
%\thispagestyle{empty}
到土庄两个月零四天,蓝玉来了。
\\

蓝玉来的头天晚上,土庄下了一场罕见的暴雨。第二天一大早我起得床来,看见院子里跪着一个男娃子。他的全身上下都湿透了,衣裤上粘满了黄泥。在他的身边,是一个三十出头的汉子,也披着一身的潮湿,他两个手不停地搓着,眼睛跟着师傅转。这个时候,我的师傅正在牛圈边给牛喂草,他大把大把的把青草扔给圈里的牛,还在院子里过来过去的,就是不看院子里的蓝玉和他的父亲,仿佛院子里的两个人只是虚幻的存在。我看出了蓝玉父子的尴尬,想起自己刚来到这个院子的情景,就有些同情院子里的人。
\\

这个时候,蓝玉抬起了头,向我这边看了一眼,我给了他一个浅浅的微笑,一脸黄泥的蓝玉也笑了,他的笑意很薄很轻,仿佛往湖面上扔了一块拇指大小的石子起来的一层涟漪。好多年后蓝玉还在对我说,他说当时跪在泥水里的他都有了天地崩塌的感觉,他已经打定回家的主意了,不管他的父亲同不同意他都准备回家了,就是因为我的那个微笑,他留了下来。
\\

师傅同意收下蓝玉是在蓝玉的父亲两个膝盖也重重的跌落在泥地里后。当时师傅正抱着一捆青草往牛圈边去。那个异样的声音至今还犹然在耳,我看见蓝玉的父亲两腿一屈,接着他面前的水被砸得稀烂,咚,一个院子都颤抖起来。师傅回过头就僵在那里了,然后他说你起来吧,我可以试试他是不是吹唢呐的料,不行的话,你还得把娃领回去。
\\

和我相比,蓝玉的测试多出了好几项内容。除了吸水,还有吹鸡毛,师傅把一片鸡毛扔到天上,要蓝玉用嘴把鸡毛留在空中,一袋烟的功夫不能掉到地面。还有就是打靶,含上一口水,对着桌上的木牌,在四步外的距离用嘴里的水把木牌射倒。我很为蓝玉担心,因为我连一瓢水也是吸不完的。\\

蓝玉轻描淡写的就完成了测试,不仅我惊讶,连师傅都有些惊讶了。虽然他把这种惊讶包裹得很严实,当蓝玉把桌上的木牌射倒后,他的两条眉毛很迅速的彼此凑了凑,眉间也多出来一条窄而深的沟壑。我至今都承认,我的师弟蓝玉天分比我要高得多。
\\

蓝玉留下来了,和我住一张床。师傅还郑重的把我介绍给了蓝玉,说这是你师兄,师兄师弟,就要像亲兄弟一样的,懂不懂?蓝玉点了点头,我也点了点头。
\\

晚上蓝玉在床上问我,吹唢呐好玩吗?我说不知道,蓝玉惊讶地翻起来说你怎么会不知道呢?你不是都来两个月了吗?我说我还没吹上一天的唢呐呢!哪你在干啥?蓝玉问。喝水,喝河湾的水。我答。
\\

打蓝玉来后,土庄的河湾边吸水的娃由一个变成了两个。土庄人从河湾过就大声说焦三爷又收徒弟了,焦家唢呐班人强马壮了。
\\

在我们吸水的这段日子里,师傅和他的唢呐班共出了十多趟门。整个无双镇都跑遍了。我和蓝玉还认识了焦家唢呐班的师兄们。我的大师兄年纪和我父亲差不多,师傅让我和蓝玉叫他大师兄,我们都有些不好意思,毕竟他是个满脸胡须的大人。我们怯怯的喊罢,大师兄摸摸我们的脑袋,然后看着师傅笑笑。师傅说磨磨都能出来。大师兄又笑一回,他笑的时候嘴裂得很大,胡子满脸跑,他把唢呐凑到嘴里,唢呐的苇哨和铜围圈就不见了。
\\

接活后出门的前一晚,焦家班照例要吹一场的。院子里摆上一张桌子,桌子上有师娘煮好的苦丁茶和炸好的黄豆。师傅和他的徒弟们散坐在院子里,大家先聊一些家常。聊家常的时候有一个人声音最大,说话像打雷,他是我的二师兄。据师娘讲,二师兄是师傅最满意的徒弟,天分好,也刻苦,特别擅长吹丧调,能在灵堂把一屋子人吹得流眼抹泪。聊一阵子天,师傅就咳嗽两声,众人会意,各自从布袋子里抽出唢呐,第一步是调音,看看唢呐音调对不对;然后师傅起调,如果接的是红事,就吹喜调,喜调节奏快,轻飘飘的在院子里奔跑;如果接的是白事,就吹丧调,丧调慢,仿佛泼洒在地上的黏稠的米汤,等到师傅独奏的那一段,我和蓝玉眼窝子都有了一窝水。
\\

无双镇大部分人家接唢呐都是四台,所谓四台,就是只有四个唢呐手合奏;比四台讲究的是八台,八台除了四个唢呐手,还有一个鼓手,一个钵手,一个锣手,一个钞手。八台不仅场面大,奏起来也气势非凡。师娘告诉我,如果练的是八台,土庄的人都会来,聚在院子里,屏声静气的听完才散去。毕竟八台一是难度大,二是价钱高,一般人家是请不起的,土庄人近水楼台,运气好的话一年能听上一两回。我又问师娘,有比八台更厉害的吗?师娘笑笑,说有,我问:是什么?
\\

百鸟朝凤,师娘答。
\\

怎么个吹法?我问。
\\

独奏!师娘说这话的时候神情肃穆。
\\

独奏?谁独奏?我和蓝玉惊讶的问。
\\

夜风撩着师娘的头发,她的表情像一本历史书,好久她才说,当然是你们师傅。\\
			\fancyhead[LO]{{\scriptsize 【百鸟朝凤】第五章}} %奇數頁眉的左邊
\fancyhead[RO]{\thepage} %奇數頁眉的右邊
\fancyhead[LE]{\thepage} %偶數頁眉的左邊
\fancyhead[RE]{{\scriptsize 【百鸟朝凤】第五章}} %偶數頁眉的右邊
\fancyfoot[LE,RO]{}
\fancyfoot[LO,CE]{}
\fancyfoot[CO,RE]{}
\chapter*{五}
\addcontentsline{toc}{chapter}{\hspace{11mm}第五章}
%\thispagestyle{empty}
三个月了,我用一人多高的芦苇杆把河湾的水吸了上来。可我还是没有吹上唢呐。师傅只是让我和师娘下地给玉米除草。土庄六月的天气似乎比水庄的要热得多,我们水庄这个季节都是湿漉漉的。在玉米地里,我对师娘说土庄不如水庄好,我们水庄没有这样热,师娘就哈哈的笑,笑完了说游家娃是想家了。中午收工回家,经过河湾的时候,我的师弟蓝玉扎着马步在河湾上吸水。蓝玉是有天分的,他才来一个月,就接到师傅递给他的一人多高的芦苇杆了。我到这一步比蓝玉整整多用了一个月时间。
\\

吃完晚饭,蓝玉去刷碗,自从他来了以后,刷碗这个活就是他的了。刚开始我还觉得好,想终于可以不用刷碗了。可没过两天师傅对我说,跟你师娘下地吧。才下了半天的地,我又想念刷碗了。蓝玉刷碗的声音特别响,刷碗这活我是知道的,磕磕碰碰发出些声响是难免的,但绝没有这样大的声响的。连提个水壶,蓝玉都要弄得惊天动地的,一弓腰,就发出咳的一大声,仿佛他提起来的不是一个水壶,而是一扇石磨。很快,蓝玉就从厨房出来了,他甩了甩两只湿漉漉的手,眼睛看着师傅和师娘,他的意思是告诉我们,该他的活已经干完了。
\\

蓝玉得到了师娘的夸奖,师娘说蓝玉刷碗动作比天鸣麻利,顿了顿师娘又说,麻利是麻利,但没有天鸣刷的干净。
\\

蓝玉不仅话多,也会讲。他坐在师傅和师娘的中间给他们讲他们木庄的奇怪事,师娘被他逗得哈哈大笑,连师傅一直绷着的脸都会不时舒展开来。我没有蓝玉的嘴皮子,就在旁边一直闷坐着,师娘好像看出来了,就对我说,天鸣是不是想家了,想家的话就回去看看吧。他说这话的时候眼睛一直盯着师傅,我想是这个事情她做不了主,在征求师傅的意见。一提到回家,我的眼窝就一阵发热,我真想家了,想父母,还有两个妹妹,他们肯定也在想着我的。
\\

我目不转睛的看着师傅,老半天师傅才说,早去早回。
\\

我又回到水庄了。
\\

以前觉得水庄什么都不好,一脚踏进水庄的地界,我发现水庄什么都好,水庄的山比土庄的高,水比土庄的绿,连人都比土庄的耐看呢。
\\

走进我家院子,母亲正蹲在屋檐下剁猪草,父亲站在楼梯上给房顶夯草。一看见我,母亲就扔掉手里的活跑过来,她摸摸我的头,又摸摸我的脸,说天鸣回来了,还瘦了。母亲的手有一股青草的腥味,但我觉得特别好闻,我好久没有看见母亲的脸了,好像黑了不少,看着母亲,我的眼睛就模糊起来。
\\

本盛,天鸣回来了。母亲对着父亲喊。
\\

父亲没有从楼梯上下来,他弯下腰看看我,又继续给屋顶夯草。
\\

好好的,回来做啥?父亲的声音顺着楼梯滑下来。
\\

师傅让我回来的。我直着脖子说。
\\

啥?你个狗日的,烂泥糊不上墙。父亲把夯草的木片子高高的摔下来,破成了好几块。
\\

娃好好的,你骂他干啥?母亲说。
\\

好好的?好好的能让师傅赶回家?父亲从楼梯上下来,还腾出一只手狠狠的对着我戳。你啊,你啊,你——。父亲发出的声音像被他嚼碎了吐出来的。
\\

晚上母亲给我做了一顿腊肉,还不让两个妹妹多吃,拼命把好吃的往我碗里夹。父亲在饭桌上不停的对我翻白眼,像要活吞了我似的。什么时候回去?母亲把碗里最后一片腊肉夹给我问。早去早回,师傅说的。我说。真的?父亲把头歪过来问,我点点头。这时候水庄的游本盛才笑了,还用筷子敲了敲我的后脑勺,轻轻的。我发现,这顿饭父亲的筷子一直没有伸到肉碗里,我把母亲给我的最后一片腊肉夹起来放进了父亲的碗里,父亲笑得更欢了,说那就恭敬不如从命了。
\\

月亮上来了,两个妹妹都睡了。我和父亲母亲坐在院子里,我给他们讲了木庄的好多事情。
\\

爸,你知道唢呐除了四台和八台,还有什么吗?我问父亲。
\\

父亲笑了笑,然后看了看母亲,母亲也笑了笑。
\\

莫非还有十六台?母亲说。
\\

我摇摇头。说唢呐吹到顶其实是独奏呢!你们知道叫什么吗?
\\

这时候我看见父亲的笑容不见了,他的目光跑到月亮上去了,面容也变得复杂了。好半天他才把目光转向我,说你知道我为什么要送你去学吹唢呐吗?\\

我摇头。
\\

就是要你学会吹百鸟朝凤。
\\

我惊讶了,就兴奋的说原来你也知道百鸟朝凤的啊!还表态说你们放心,我学会了回来吹给你们听。
\\

没有那样简单,你师傅这十多年来收了不下二十个徒弟,可没有一个学会百鸟朝凤的。父亲说。
\\

很难学吗?我问。
\\

倒不是,这个曲子是唢呐人的看家本领,一代弟子只传授一个人,这个人必须是天赋高,德行好的,学会了这个曲子,那是十分荣耀的事情,这个曲子只在白事上用,受用的人也要口碑极好才行,否则是不配享用这个曲子的。
\\

咱家天鸣能学会吗?母亲问。
\\

父亲摇摇头,走了。院子里只剩下母亲和我,还有天上的一轮残月。\\
			\fancyhead[LO]{{\scriptsize 【百鸟朝凤】第六章}} %奇數頁眉的左邊
\fancyhead[RO]{\thepage} %奇數頁眉的右邊
\fancyhead[LE]{\thepage} %偶數頁眉的左邊
\fancyhead[RE]{{\scriptsize 【百鸟朝凤】第六章}} %偶數頁眉的右邊
\fancyfoot[LE,RO]{}
\fancyfoot[LO,CE]{}
\fancyfoot[CO,RE]{}
\chapter*{六}
\addcontentsline{toc}{chapter}{\hspace{11mm}第六章}
%\thispagestyle{empty}
回到土庄我才知道,蓝玉已经把河湾里的水吸上来了。\\

一回来蓝玉就兴冲冲的问我用长芦苇吸上河湾的水用了多久,我掰着指头数了数说一个半月多一点吧。我用了十天。蓝玉骄傲的说。我心里就有些神伤了,说师傅都说了的,你的天分比我好。蓝玉就拍拍我的肩膀,说你也很好的。
\\

但是我发现我真的不好。
\\

蓝玉吸上水后本来也和我下地的,可下地才几天,事情就发生了变化。
\\

我清楚地记得那天有好大好大的雾,气势汹汹的,整个土庄都不见了。我还没起床,就听见蓝玉的尖叫声,我翻了个身,想多睡一阵子。蓝玉总是起的比我早,甚至比师傅师娘还早,为此他还得到了师傅的夸奖。说实话,我也想像他那样起得早的,我也想得到师傅的夸奖的,可我就是起不来,硬着头皮爬起来也是昏昏沉沉的,好一阵子满世界都在乱转。到后来我索性不起来了,夸奖也不想要了,只要让我多睡一会儿就阿弥陀佛了。
\\

起来,快起来,土庄不见了。蓝玉跑进来摇我。
\\

嗯!我咕哝一声,没理会他。
\\

天鸣,土庄没有了。他干脆把我的被窝抱走了。
\\

无奈,我只好起来,走到屋外我才发现土庄真的不见了。
\\

那是我一生中见到的最大的雾,天地都给吃掉了,连站在我面前的蓝玉也消失了。一眼的白,那白还泛着湿。我没有见过有这样气势的大雾,呼吸都不顺畅了。我凑近蓝玉,他正用两只手拼命的捞悬在空中的白,像一只巨大的蜘蛛,被自己拉出来的丝给网住了。
\\

你们两个进来。师傅在里屋喊。
\\

我和蓝玉折进屋,师傅说今天雾大下不了地了,正好我有事情要交代。
\\

师傅从床下拉出一个锈迹斑斑的铁皮箱子,他打开箱子,我和蓝玉都凑过去看,屋子里光线不好,只能看过大概,反正里面都是唢呐,大大小小,长长短短的唢呐。师傅弯下腰不停的翻检着箱子里面的家什,挑啊拣啊,终于,他抽出了一支略短一些的唢呐,把唢呐放进嘴里,唢呐就发出长长的一声――呜。师傅直起腰来,把唢呐递给我身边的蓝玉,说从今天开始你就不用下地了,专心吹唢呐吧,先把它吹响,我就教你基本的调儿。
\\

蓝玉当时的样子我都没法子形容,接过唢呐的那一刻,昏暗的屋子里竟然划过两道亮光,那是蓝玉眼睛里出来的。我看见蓝玉握着唢呐的手在轻轻的抖动,然后他笨拙地把唢呐塞进嘴里,腮帮子一鼓,唢呐就放出来一个闷屁,又一鼓,又出来一个闷屁。
\\

我想师傅接下来该给我派发唢呐了,说不定是支长的呢,比蓝玉的长。我就定定得盯着师傅的手,希望他能抓住一支长的唢呐不放,再放到嘴里试一试,然后递给我。但我是不会像蓝玉那样没有一点定力,当场就放几个闷屁显摆,我会找个没人的地头悄悄放。
\\

师傅是拿出了唢呐,拿出来还不止一支,拿一支出来,他先是吹吹,然后卷起袖口拭擦一番,又放回去,又捡起一支吹拭一番,照例又放回去。我眼珠子都瞪直了,总是希望下一支就是我的,开始看见短的还害怕,怕他递给我,我想要一支比蓝玉长的。可随着箱子里翻剩下的唢呐越来越少,我的心就开始绷紧了,想短的也成,就是拇指长短的我也收。
\\

“砰”的一声,师傅合上了他的箱子。
\\

我没有吹上唢呐。晚上我对蓝玉说我要回家了。蓝玉说你不是刚回过家吗?我说我不想学吹唢呐了。我现在才知道,师傅其实是看不上我的。
\\

土庄的夏天是没有水庄的好看,可土庄的秋天却老有味儿了。土庄的山小是小了些,可山上都有树,种类也繁多,常青的松和落叶的枫抱在一起,夏天还是整齐的绿,到秋天枫树就醉了。就这样,一个一个红绿间杂的山丘一排儿的往远方去了,像一排生动的省略号。我背着行李顺着省略号一直走,边走边哭,我悲伤极了,来土庄都这样老长的日子了,我就是吹不上唢呐,却成了焦家的长工。又想我连唢呐都没有摸过就回到土庄,土庄人肯定要笑我了。还有,我最担心的还是父亲,我这样回去倒不是怕他揍我,我是怕他会活活气死。
\\

我是偷偷走的,从土庄不见了的那天起,我就想走了。昨天晚上,我的师弟蓝玉又爬到我的床上吹了一回唢呐,他吹的时候还拿眼睛瞟着我,眼角得意的往上翘。我知道他是在我面前显摆,可我不恨他,因为要换着我我也是想显摆的。蓝玉的脑袋很大,所以他很聪明,他现在都能把师傅教给他的丧调吹得我眼窝子发潮了。吹到精彩的地方他还会停下来给我讲,这是滑音,这是长调。每天我和师娘下地,他就爬到我干活的地头,猴样的窜上草垛子,呜呜啦啦的就吹开了。回家的路上,我一身的疲惫,连走路都摇晃着,蓝玉却活蹦乱跳,像早晨刚刚抽上露水的青草儿样鲜活。
\\

我走了,谁都不知道我走了。我走的时候蓝玉还抱着他的唢呐在床上说梦话呢。本来我想跟他道个别的,可我又怕他大呼小叫的惊动了师傅师娘。出门我才发现天还没亮,四处都是让人心悸的黑。我摸索着在屋檐下坐下来,坐下来就想在土庄的这些日子,想师傅和师娘。师娘是个好人,像母亲,在地里还不让我多干活,吃饭老往我碗里夹菜。我最不留恋的就是师傅,我还偷偷给他起了外号,叫焦黑炭。焦黑炭没有一点好,整天绷着脸不说,还不让我吹唢呐。想了好多,我的心里五味杂陈,喉咙一硬,就悄悄呜呜的哭起来,一直哭到天色微明,回家的路也能见着了,我才站起来离开,走出一段回头看了看,眼泪又下来了。
\\

终于要离开土庄了,我这辈子怕是当不上唢呐匠了。想起上次回家时给父亲和母亲表的态,说一定学会那首百鸟朝凤回家吹给他们听。但是眼下的情形别说百鸟朝凤了,就是一段稀松的丧调都没有学会。我觉得我最对不起的人就是水庄的游本盛了,他一心一意的送他的儿子学唢呐,可他的儿子学了差不多半年,连用唢呐放两个闷屁的机会都没有,这让水庄人知道了还不笑掉大牙?又伤心了一回,却没有让我放弃回家的念头,反正迟早都是要一无所成的回家的,晚回不如早回,早回还能给家里帮把手。
\\

又看见了水庄,横在天地间,安静得像熟睡的孩子。再拐一个弯,就到我们水庄的地界了。我走的是下坡路,路细而窄,弯弯拐拐,像截扔在山坡上的鸡肠子。路两边有一溜的火棘树,那些枝枝蔓蔓都不安分的往路上凑,这样本就狭窄的小路都快看不见了。
\\

拐过弯,我听见路坎下有说话的声音。踮起脚,我看见老庄叔正领着一群人在他的新房上夯草。干活的人里还有我的父亲,水庄的游本盛。我悄悄的从火棘树下钻过去,把身子隐在草丛里。
\\

天鸣最近没回家?老庄叔问父亲。
\\

吹着呢!好多调调都会了。父亲声音很大。
\\

以前我还没看出天鸣这娃是吹唢呐的料呢!老庄叔又说。
\\

天鸣可比我强,我这娃不要平时看他不吭不响的,做起事情来可一点不含糊。父亲说,前久回来还气粗的给我和他老娘表态,要吹百鸟朝凤呢!
\\

老庄叔就笑一回,他知道父亲是吹牛。就说,百鸟朝凤!百鸟朝凤!我都好多年没听过了,上一次听还是十多年前,火庄的肖大老师去世,焦三爷给吹过一次,那场面,至今还记得,大老师的亲戚学生在院子里跪了黑压压一片,焦三爷坐在棺材前的太师椅上,气定神闲的吹了一场,那个鸟叫声哟!活灵活现的。
\\

等天鸣学回来了,我让他吹给你们听。父亲许愿。
\\

那样我们水庄就长脸了,本盛也长脸了,我就是担心,天鸣有没有那个福气,这百鸟朝凤一代弟子就传一个人呢。老庄叔说。
\\

你们可以不相信天鸣,我是相信我的娃的。父亲说。
\\

我蛇样的从草丛里梭出来,我不想回家了,我想吹唢呐,从来没有像此刻这样想吹唢呐。
\\

我顺着原路爬到山顶,回头看了看水庄。远处近处有袅袅的炊烟,水庄醒过来了。
\\

回到土庄师傅正在院子里磨刀。看见我失魂落魄的站在院子边的土墙下,师傅说:你师娘到地里去了,你也去吧!
\\
			\fancyhead[LO]{{\scriptsize 【百鸟朝凤】第七章}} %奇數頁眉的左邊
\fancyhead[RO]{\thepage} %奇數頁眉的右邊
\fancyhead[LE]{\thepage} %偶數頁眉的左邊
\fancyhead[RE]{{\scriptsize 【百鸟朝凤】第七章}} %偶數頁眉的右邊
\fancyfoot[LE,RO]{}
\fancyfoot[LO,CE]{}
\fancyfoot[CO,RE]{}
\chapter*{七}
\addcontentsline{toc}{chapter}{\hspace{11mm}第七章}
%\thispagestyle{empty}
师傅把唢呐递给我。是一支小唢呐,哨子是用芦苇制成的,蕊子是铜制的,杆子是白木的,铜碗的部分则有些斑驳了。我摩挲着它,这支唢呐比蓝玉的要小,但我已经很满足了,我终于吹上唢呐了。我使劲揪了一下大腿,生生的疼。
\\

这是当年我师傅给我的,是我的第一支唢呐。师傅蹲在大门口吸着旱烟说。
\\

别看它个儿小,但是调儿高,唢呐就是这样,调儿越高,个儿就越小。师傅吐出一口烟雾接着说。
\\

我点点头,门口的师傅渐渐就模糊了。
\\

冬天来了,木庄也热闹了。我和我的师弟蓝玉把木庄整天搅得呜呜啦啦的。河湾边,草垛上,还有庄子西边的大青石上,都能听见破烂的唢呐声,破烂的声音主要是我吹出来的,蓝玉吹的唢呐声已经很悦耳了。他吹的时候,过往的木庄人会停下来仔细听一听,听完了就远远的喊说焦家班后继有人了。我则没有这样的待遇,过往的听见我的唢呐声拔腿就跑了,我就和蓝玉哈哈的笑。
\\

师傅很吝啬,每次教给我的东西都少得可怜,一个调子就要我练习十来天。
\\

焦家班又接活了。出门的前一晚,一班人围在火塘边,木桌上还是有苦丁茶和炒黄豆。我和蓝玉一人抱着一支唢呐坐在人群中,血都滚热了。我们终于成为焦家班的一员了,也许要不了多久,我们就可以和师兄们一起到很远很远的地方去了。大家演奏完,大师兄就说两个师弟来的时间也不短了,也该露一手了。我有些怯,因为我吹得实在是不好,就推说让师弟先来吧。蓝玉也不推辞,像模像样的先抖一抖衣袖,两手举着唢呐,往前一推,再徐徐的把哨子凑进嘴里,像一个老练的唢呐手。蓝玉吹奏得确实好,我觉得和师兄们都差不多了。他演奏的是一段喜调,曲子轻快的在屋子里跳跃,他脑袋和调子一起左摇右晃的,吹得一屋子喜气洋洋。吹奏完了,大师兄就摸蓝玉的大脑袋,说不得了不得了,其他师兄也说好,只有师傅不说话,大口大口的吸烟。
\\

蓝玉吹完了,一屋子人都看着我,我的心突突的跳,握着唢呐的手也浸出好多的汗来。二师兄对着我点点头,我知道他是鼓励我。我战战抖抖的把唢呐塞进嘴里,呜呜的憋出几个滑音和颤音,然后我低下头,说我就会这点了。
\\

一屋子都无话了,只有油灯在轻轻的跳动。师兄们都神情肃穆的看着师傅,师傅还是低着头吸烟。好半天二师兄才低低的对师傅说,师傅恭喜您了。师傅把旱烟伸到凳子腿上按熄说好了今天就到这里,散了吧,明天还要赶远路呢!\\

我不知道二师兄为什么要恭喜师傅,我吹得那样烂,这样久了也只会吹一些基本的音调,师傅还一副不依不饶的样子,每天就只要我钉着几个调儿吹。
\\

就几个调,我把冬天吹来了。
\\

今年的第一场雪总算来了,都孕育了好几天了,直到昨夜才落下来。半夜我和蓝玉都听见了雪花滑过窗棂的声音。我和蓝玉都睡不着。我们睡不着倒不是等这场雪。在黑夜里大大的睁着眼睛,是等天亮后激动人心的一刻。昨天晚上,焦家班围在火塘边奏完最后一曲调子后,师傅对大家说:明天天鸣和蓝玉也和我们一起出门吧!
\\

蓝玉推开窗户对我说,落雪了,不知道我们木庄是不是也落雪了呢?我说我们水庄肯定是落雪了的,每年这个时候,雪落得可大了,漫天遍野的飞,一个庄子都陷下去了。
\\

我起得很早,草草的抹了一把脸,小心翼翼的把唢呐装好。我装唢呐的布袋子是师娘缝的,碎花青布,唢呐刚好能放进去,可熨帖了;蓝玉的唢呐也有布袋子,是藏青棉布缝制的,后来我才发现,装蓝玉唢呐的布袋子的前身是师傅的内裤。这个秘密我一直没有给蓝玉讲,再后来我又发现,我的布袋子是师娘贴肉的裤衩改的。
\\

今天要去的人家请的是白事。我刚装好唢呐,接客就到了。来接唢呐的是两个年轻人,比我和蓝玉大不了多少,嘴边刚刚长出来一些茸毛,他们一人背着一个背篼,怯生生的站在院子边。我们无双镇就是这样的,请唢呐要派接客,接客要负责运送唢呐匠的工具,等活结束了,还得送回来。
\\

很快我的七个师兄就到了,看来主人请的是八台,七个师兄加上师傅刚好八个。我和蓝玉当然还不能上阵,蓝玉其实是够了的,但师傅说了,先跟一段再说。两个接客很麻利的把锣啊鼓啊的全装进背篼,看我和蓝玉怀里还抱着唢呐,就伸过手来说都装上吧。我不让,说自己拿就成了,反正也不重的。接客不让,说哪有唢呐匠自己拿东西的道理,我们金庄没有这规矩,无双镇也没有这规矩。我还想推让,师傅在旁边说,给他吧,不依规矩,不成方圆。
\\

主人姓查,金庄漫山遍野散落的人家差不多都姓查。
\\

我们被安排进一个单独的屋子,屋子很紧凑,还有两个炭火盆。屁股还没有坐热,师傅就对大家说:“捡家伙,开锣!”。说完就往院子里去了。\\

我终于能亲眼目睹唢呐匠们正儿八经的八台大戏了。焦家班在院子里呈扇形散坐着,师傅居于正中,他的目光左右扫视了一番,众人会意,齐齐进入了状态。一声锣响,焦家班在金庄的唢呐盛会拉开了序幕。我此时听到的唢呐声和昨天晚上听见的预演有极大的差别,师傅和他的一班弟子个个全神贯注。唢呐声在高旷的天地间奔突。先是一段宏大的齐奏,低沉而哀婉;接着是师傅的独奏,我第一次听到师傅的独奏,那些让人心碎的音符从师傅唢呐的铜碗里源源不断的淌出来,有辞世前的绝望,有逝去后看不清方向的迷惘,还有孤独的哀叹和哭泣。尤其是那哭声,惟妙惟肖。一阵风过来,撩动着悬在院子边的灵幡,也吹散了师傅吹出来的哀号,天地间陡然变得肃杀了。
\\

一直在院子里劳作的人群过来了,没有人说话,目光全在师傅的一支唢呐上。渐渐有了哭声,哭声是几个孝子发出来的。没多久,哭声变得宏大了,悲伤像传染了似的,在一个院子里弥漫开来,那些和死者有关的,无关的人,都被师傅的一支唢呐吹得泪流满面。
\\

一曲终了,有人递过来一碗烫热的烧酒,说焦师傅,辛苦了,润润嗓子吧。
\\

开过晚饭,主人过来了。先是眼泪汪汪的给师傅磕了一个头。说这冰天雪地的你们还能赶过来送我老爹一程,我谢谢你们了。
\\

“他生前是我们查家的族长,可德高望重了!”主人爬起来说。
\\

师傅点点头。
\\

“做了不少好事,我都数不过来。”主人又说。
\\

师傅又点点头。
\\

“焦师傅,你受累,看能不能给吹个百鸟朝凤?”主人把脑袋伸到师傅面前问。
\\

师傅摇摇头。
\\

“钱不是问题!”\\

师傅还是摇摇头。
\\

磨了好一阵子,师傅除了摇头什么都不说。主人无奈,只好叹着气走了,走到门口又心有不甘的回头问:“我老爹真没这个福气?”。师傅抬起头说你去忙吧!
\\

主人走了,二师兄看着师傅说:“师傅,查老爷子德高望重呢!”。师傅的鼻腔哼了哼:“知道查姓为什么是金庄第一大姓吗?以前的金庄可不光是查姓,都走了,散到无双镇其他地头去了,这就是查老爷子的功劳!”。
\\

接下来几天,我和蓝玉就进天堂了。顿顿有肉吃,其间我和蓝玉还偷喝了烧酒,焦家班坐到院子里吹奏的时候,我还和蓝玉躲在屋子里抽烟,烟是主人家偷偷塞给我们的,我和蓝玉本来是不收的,可主人家不干,非得塞给我们。
\\

离开那天,死者的几个儿子把焦家班送出好远,临了就把一沓钱塞给师傅,师傅就推辞,结果两个人在分手的桥上你来我往的斗了好几个回合,师傅才很勉强的把钱收下来。
\\

几个师兄则站在一边木木的看着,眼神倦怠,眼前这个场景他们已经看够了。\\
			\fancyhead[LO]{{\scriptsize 【百鸟朝凤】第八章}} %奇數頁眉的左邊
\fancyhead[RO]{\thepage} %奇數頁眉的右邊
\fancyhead[LE]{\thepage} %偶數頁眉的左邊
\fancyhead[RE]{{\scriptsize 【百鸟朝凤】第八章}} %偶數頁眉的右邊
\fancyfoot[LE,RO]{}
\fancyfoot[LO,CE]{}
\fancyfoot[CO,RE]{}
\chapter*{八}
\addcontentsline{toc}{chapter}{\hspace{11mm}第八章}
%\thispagestyle{empty}
春天降临了。\\

乡村的春天总是和仪式有千丝万缕的联系。像我们无双镇,春天一露头,就有拜谷节,播洒谷种的前一夜,每个村子的老老少少都要带上祭品,去本村最大的一块稻田里供奉谷神;拜谷节过去没几天,就该是迎接灶神爷的日子了,猪头是不能少的,还有小米渣,听老人们说,天上是没有小米渣的,人间全靠这点东西留住他老人家了;把灶神爷安顿好,就是晒花节了,太阳公公和花仙一起供奉,因为有两个神仙,供品自然不能少,蜂蜜、白米,干菊花,还有圆圆的玉米饼。太阳还没有出来,一庄人早就遥对着太阳升起的地方把供品摆放妥贴了,等那抹血红一上来,大家就整齐的磕头作揖,好听的话也会说不少,庄稼人没野心,就是祈求有个好年成。\\

晒花节刚过,土庄又热闹了。人们槐花串似的往焦三爷的院子里跑,扛凳子搬桌子的。遇上闲逛的路人,就有人招呼:“焦三爷传声了!”,路上的人一听,一张脸就怒放了,随即融入队伍。往焦三爷的院子迤逦而来。\\

土庄人等这个盛况的日子已经很久了。\\

无双镇的唢呐班每一代都有一个班主,上一代班主把位置腾给下一代是有仪式的,这个仪式叫“传声”,不传别的,就传那首无双镇只有少数人有耳福听到过的“百鸟朝凤”。接受传声的弟子从此就可以自立门户,纳徒授艺了,而且从此就可以有自己的名号,比如受传的弟子姓张,他的唢呐班子就叫张家班,姓王,则叫王家班。总之,那不仅仅是一门手艺,更是一种荣耀,它似乎是对一个唢呐艺人人品和艺品最有力的注脚,无双镇的五个庄子都以本庄能出这样一个人为荣。\\

这个仪式最吸引人的还不是他的稀有,而是神秘。在仪式开始之前,没有人知道谁是下一代的唢呐王。所以,焦家班所有的弟子都是要参加这个仪式的,连他们的亲人都会四里八乡的赶来参加,因为谁都可能成为新一代的唢呐王。\\

人实在太多了,师傅的院子都装不下了,于是屋子周围的树上都满满当当的挂满了人参果。我和我的一班师兄弟坐在院子正中间,两边是我们的亲人,我父母还有两个妹妹都来了;我的师弟蓝玉坐在我的旁边,他的家人也来了,比我的父母还来得早些。他们的脸上都是按捺不住的期待和兴奋。\\

屋檐下有一张八仙桌,八仙桌的下面是一头刚宰杀完毕的肥猪。此刻,这头猪是供品,仪式结束后,他将成为全土庄人的一顿牙祭。猪头的前面有个火盆,火盆里的冥纸还在燃烧。师傅坐在八仙桌后面。他一直在闷着头抽烟,师傅的烟叶是很考究的,烟叶晒得很干,吸起来烟雾特别大。很快,师傅的一张脸就不见了,他的半截身子都隐在一片雾障中,像一个踏云的神人,我竟然生出一些隐约的幻意。\\

良久,师傅才站起来,四平八稳的拄灭手里的烟袋,对着人群,平伸出双手往下压了压。喧闹的人群瞬间就安静下来。往地上吐了一口痰,师傅发话了。\\

“我快要吹不动了,可咱们这山旮旯不能没有唢呐,干够了,干累了,大家伙儿听一段还能解解乏。所以啊!在咱们这地头唢呐不能断了种。我寻思了好久,该找一个能把唢呐继续吹下去的人了!”师傅咳嗽了两声,停了停,下面又开始有响声了。这个时候我偷偷的侧目看了看蓝玉,我发现蓝玉也在偷偷的看我,他的嘴角还淌着一些笑。四目相对,我的脸刷就红了,像是心里某种隐秘的东西被戳穿了似的。蓝玉的脸没有红,他的脑袋抬得更高了,像一只刚刚得胜的大公鸡。我就升起一些不快,想还没见底呢,咋知道水底是不是石头?又想想,我的这班师兄弟里,也只有蓝玉最适合了,他人精灵,天分高,也勤苦。反正最后是他我也不会惊奇的。最后我觉得我那几个师兄也可怜,为什么师傅不全给传了呢?那样就整齐了,人人有份,个个能吹百鸟朝凤,焦家班、蓝家班、游家班,还不响亮死啊!\\

师傅又开腔了:“我这几年收了不少徒弟,大大小小的,个个都有些活儿,出活也带劲,没给吹唢呐的丢人。”顿了顿师傅接着说:“我们吹唢呐的,好算歹算也是一门匠活,既然是匠活,就得有把这个活传下去的责任,所以,我今天找的这个人,不是看他的唢呐吹得多好,而是他有没有把唢呐吹到骨头缝里,一个把唢呐吹进了骨头缝的人,就是拼了老命都会把这活保住往下传的。”师傅又咳嗽了两声,对旁边的师娘点了点头,师娘过来递给师傅一个黑绸布袋子。师傅接过来,小心翼翼的从里面抽出来一支唢呐。远远的我就感觉到了这支唢呐该有些年龄了,铜碗虽然亮得耀眼,却薄如蝉翼,杆子是老黄木的,唢呐的杆子一般就是白木,最好的也就是黄木,能用这样色泽的老黄木制成的唢呐,足见它的名贵。乡村人一般是见不到这样的稀罕货的。\\

“这支唢呐是我的师傅给我的,它已经有五六代人用过了,这支唢呐只能吹奏一个曲子,这个曲子就是百鸟朝凤。现在我把它传下去,我也希望我们无双镇的唢呐匠能把它世世代代的传下去。”师傅举着唢呐说。\\

院子里一点声音都没有,我只听见我的师弟蓝玉的喘息声,所有的眼睛都盯着师傅手里的那支唢呐。我相信这一刻的土庄是最肃穆的了,这种肃穆在了无声息中更显得黏稠,我最后只能听见自己的呼吸声了。\\

我侧目看了看我的师弟蓝玉,他紧缩着脖子,脑袋花骨朵似的。慢慢地,他的脖子被拉长了,成了一朵盛开的鲜花,花朵儿正期待着雨露的降临,焦虑、渴望在稚嫩的花瓣间涌动着。蓦然,盛开的鲜花枯萎了。几乎就在一眨眼间,正准备迎风怒放的花儿无声地凋谢了,花瓣起来了一层死灰,花杆儿也挫短了半截。这朵刚才还生机蓬勃的花儿,转眼间铺满了绝望的颜色。悲伤一下从我的心底涌起来,我的师弟蓝玉,迅速的在我眼睛里枯萎,他的目光慢慢的转向了我,我能看懂他的眼神,有不信、不甘、绝望,当然,还有怨恨,可我看到的怨恨很少,很稀薄,星星点点的。\\

这时候我的父亲,水庄的游本盛在旁边喊我:“你呆了,师傅叫你呢!”\\

父亲的声音像耍魔术的使用的道具,充满了意外和惊喜。\\
			\fancyhead[LO]{{\scriptsize 【百鸟朝凤】第九章}} %奇數頁眉的左邊
\fancyhead[RO]{\thepage} %奇數頁眉的右邊
\fancyhead[LE]{\thepage} %偶數頁眉的左邊
\fancyhead[RE]{{\scriptsize 【百鸟朝凤】第九章}} %偶數頁眉的右邊
\fancyfoot[LE,RO]{}
\fancyfoot[LO,CE]{}
\fancyfoot[CO,RE]{}
\chapter*{九}
\addcontentsline{toc}{chapter}{\hspace{11mm}第九章}
%\thispagestyle{empty}
蓝玉走了,披着一身绚烂的朝霞,向着太阳升起的地方去了。我站在土庄的土堡上,看着他的身影逐渐变小变淡。太阳明天还是要升起的,可我却见不到我的师弟蓝玉了。蓝玉在我的生命里出现和消逝都突然得紧,仿佛那个落雨的日子,蓝玉就该出现在我的面前,又仿佛这个炫目的黄昏,他本就一定要离去。\\

昨晚的晚饭很丰盛,有师娘做得最好的土豆汤,师娘做土豆汤是要放番茄的,番茄在无双镇不叫番茄,叫毛辣角,毛辣角又是土庄特有的小个毛辣角,樱桃样。师娘把剁碎的毛辣角和土豆搅拌在一起,还放了半勺猪油,颜色血红,喝起来酸酸的,很开胃;另外,还有蓝玉最喜欢的灰灰菜,灰灰菜是凉拌的。我在水庄没有见到过这种野菜,蓝玉说他们火庄也没有。嫩嫩的灰灰菜在水里飞快的跑过一趟,晾干后凉拌,居然有鲜肉的味道。\\

饭桌上师娘不停地往蓝玉的碗里夹菜,一盘灰灰菜差不多都到蓝玉碗里了。蓝玉很得意,不停的对我撇嘴,还故意砸吧出嘹亮的声音。师傅吃饭是没有响动的,他每一个动作都很小心,在饭桌上你都感觉不到他的存在。直到他把一筷子灰灰菜夹到蓝玉的碗里,我才发现师傅一直都在饭桌上的。师傅的这个动作让我和蓝玉的嘴合不上了。要知道,焦家班的掌门人没有给人夹菜的习惯。他总是静悄悄的在饭桌上干他该干的事情,不要说夹菜,就是话也极少说的,有客人他也只是两句话,开饭时说吃饭,客人放碗时说吃饱。师傅看见了我和蓝玉的惊讶,就对蓝玉说,多吃点,这种灰灰菜只有土庄才有的。\\

我忽然有了一种不祥的预感。这种预感在晚饭后终于得到了证实。\\

师傅照例在油灯下吸烟,蓝玉就坐在他的面前。\\

“睡觉前把东西归置归置,明天一早就回去吧!”师傅对蓝玉说。\\

蓝玉低着头抠指甲,不说话。\\

“差不多了,红白喜事都能拿下来的。”师傅又说。\\

“师傅,是我哪里没有做好吗?”蓝玉问。\\

“你做得很好了,你是我徒弟中悟性最好的一个。”\\

“那你为什么要赶我走?”蓝玉终于哭了。\\

“你我的缘分就只能到这里了!”师傅叹了口气说。\\

“蓝玉不要哭,没事就到土庄来,师娘给你做灰灰菜吃。”师娘也有了一窝子眼泪。\\

“我吹得比天鸣都好,天鸣能学百鸟朝凤,我为什么不能?”蓝玉咬着牙说。他力气太大了,把左手的中指都抠出血来了。\\

师傅眼睛一亮,忽然又暗淡下去了。他站起来拍了拍屁股,烟袋悬在嘴上,背着两只手离开了,走到门边才把烟袋从嘴里拿出来,回过头说睡吧,明天还有事情干呢!这话听上去是对师娘说的,又好像是对屋子里所有的人说的。\\

睡在床上,我有很多的话想对蓝玉说,可有不知道说什么好。一直到天亮,我们谁都没有说一句话。焦家班的传声仪式结束后,蓝玉很是难过了一阵子。没多久他就缓过来了,他对我说,只要还留在师傅身边,他就一定能吹上百鸟朝凤。我是相信蓝玉的,我知道师傅传我百鸟朝凤是因为我老实,不传给蓝玉是觉得蓝玉花花肠子多。其实师傅是不对的,蓝玉天分比我好,他确实是比我精灵了一些,可人精灵点有什么不好的呢?我打心眼里希望师傅能把百鸟朝凤传给蓝玉,我也这样对蓝玉说过,可蓝玉不领情,还说我挤兑他呢!\\

现在师傅要让蓝玉走了。我的师弟最后的希望也就没有了。\\

蓝玉走的时候就是寻不见师傅。蓝玉在屋子里找了一圈也没寻着,师娘说定是下地去了。蓝玉就在院子里给师娘磕了六个头,说师娘我给你磕六个吧,你和师傅各自三个,我一并磕了。师娘把蓝玉扶起来,眼泪就哗哗的下来了。蓝玉走了,背着一个包袱,狠狠的转了一个身,留给我一个瘦削的背影。\\

蓝玉不见了,师傅从屋子后面的草垛子后转了出来。我回头看见了他,他对我说,从今天开始,我教你百鸟朝凤吧。\\
			\fancyhead[LO]{{\scriptsize 【百鸟朝凤】第十章}} %奇數頁眉的左邊
\fancyhead[RO]{\thepage} %奇數頁眉的右邊
\fancyhead[LE]{\thepage} %偶數頁眉的左邊
\fancyhead[RE]{{\scriptsize 【百鸟朝凤】第十章}} %偶數頁眉的右邊
\fancyfoot[LE,RO]{}
\fancyfoot[LO,CE]{}
\fancyfoot[CO,RE]{}
\chapter*{十}
\addcontentsline{toc}{chapter}{\hspace{11mm}第十章}
%\thispagestyle{empty}
游家班到底是哪一年成立的我忘了。那年我好像十九岁,抑或二十岁?我经常在夜晚寻找我的唢呐班子成立时候的一些蛛丝马迹。暗夜里抽丝样出来的那些记忆大抵都和我的唢呐班子无关,倒是一些无关紧要的事件从记忆的缝隙里顽强的冒出来,堵都堵不住。\\

最深刻的当数我的堂妹游秀芝和人私奔。秀芝是我四叔的闺女,一直是个老实的乡下女娃,脸蛋一年四季都红扑扑的。见到生人就红得更厉害了。之前没有一点迹象表明她要离开生她养她的水庄。那个普通的早晨,我的四叔发现他的闺女不见了。一家人慌张的找了一天也没有寻着。后来有人告诉四叔,天麻麻亮看见秀芝和赵水生一起翻过了水庄后面的那座大山。赵水生是水庄赵老把的儿子,刚脱掉开裆裤就和他老子去了远方,听说是个大城市。秀芝读书的时候和他是同桌,受过他不少欺负,我还替秀芝揍过这龟孙子一顿呢!\\

无容置疑的,赵水生拐走了秀芝。\\

四婶哭了好几场,说姓赵的这几天跑过来和秀芝两个躲在屋子里嘀嘀咕咕,感觉就不对头,然后就骂姓赵的,骂完姓赵的又骂自个儿的闺女;四叔则是每日都杀气腾腾的样子,多次表态要活剐了姓赵的。一年后事情才出现好转。秀芝寄回来了一封信,信里说她很好,在深圳的一家皮鞋厂上班,一个月能挣半扇肥猪,还照了照片,照片的背景是一个大水塘,比水庄的水塘可大多了。后来才知道,那不是水塘,是大海。\\

我很奇怪为什么我的记忆里都是和游家班成立无关的事件。为此我陷入了长时间的自责,并试图用记忆来缓解这种不安。可是在梳理属于游家班的丝丝缕缕时,却让我陷入了更大的危机中,因为这些记忆没有一丝亮色,相反,它像一面轰然坍塌的高墙,把我连同我的梦都埋葬掉了。\\

不知道出师四年还是五年后,师傅把他的焦家班交给了我。\\

那天师傅对一屋子的师兄弟们说:从今后,无双镇就没有焦家班了,只有游家班。一屋子的眼睛都在看着我,我很茫然,手足无措。他们的眼神都带着笑,善良而温暖。可我却感到害怕。我不知道我该干什么?能干什么?我只知道今后这一屋子人就要在我稚嫩的翅膀下混生活了。我想起了六七岁放羊的经历,父亲把七八只羊交给我,对我说,给我看好了,丢了一只你就甭想吃饭。我特别害怕山羊漫山遍野散落的情景,总是希望他们紧紧的拢成一团。在路上我就和山羊们商量好了的,可一上了坡它们就没有规矩了,眼里只有茂盛的青草,哪儿草好就往哪儿奔,弄得我眼里尽是颗粒状的白。到回家的时候,这些白就更稀疏了。我那时除了哭真是没其他的好办法的。\\

而此时,那个叫游本盛的男人正挑着一对儿箩筐在水庄的山路上轻快的飞奔。他对遇见的每一个重复着一句话:天鸣接班了,今后无双镇的唢呐就叫游家班了。他说这句话时除了自豪,更有一个伟大的预言家在自己预言降临时的自负。\\

猝然而至的交接像一场成人礼,从那天起,我眼里的水庄褪去了一贯的温润,一草一木都冰冷了,那些整日滑上滑下的石头也变得尖锐而锋利。\\
			\fancyhead[LO]{{\scriptsize 【百鸟朝凤】第十一章}} %奇數頁眉的左邊
\fancyhead[RO]{\thepage} %奇數頁眉的右邊
\fancyhead[LE]{\thepage} %偶數頁眉的左邊
\fancyhead[RE]{{\scriptsize 【百鸟朝凤】第十一章}} %偶數頁眉的右邊
\fancyfoot[LE,RO]{}
\fancyfoot[LO,CE]{}
\fancyfoot[CO,RE]{}
\chapter*{十一}
\addcontentsline{toc}{chapter}{\hspace{11mm}第十一章}
%\thispagestyle{empty}
游家班接的第一单活是水庄的毛长生家。\\

过来接活的是长生的侄儿。一进院子就给我父亲派烟,父亲把香烟吸得有滋有味的,一脸的幸福。这是他的唢呐匠儿子严格意义上给他带来的第一次实惠,滋味自然是与众不同的。\\

我刚从屋子里出来,父亲就冲着我喊:“八台哟!”\\

“我叔是啥人?别说八台,十六台也不在话下的。”接活的说。\\

父亲白了长生侄儿一眼:“你妈的x,哪有十六台?”\\

长生侄儿裂了裂嘴,说现在不是天鸣做主吗?自个儿造啊!别说十六台,捋出个九九八十一台也行啊!\\

父亲这回笑了,快意的猛吸了一大口烟,他从蹲着的长条木凳子上一跃而下,说:“那倒是。”\\

我点了师傅和几个师兄的名字,长生侄儿就蹦达着去通知了,走的时候又给父亲派了一支烟,父亲接过香烟说你龟儿子脚程放快些,晚上要吹一道的哟。\\

其他几个师兄都来了,师傅和蓝玉没有来,长生侄儿说他好说歹说说到口水都干了,师傅还是不来,只推说身子不太利索。我没有问他蓝玉为什么没有来。\\

我家屋子不大,寨邻来了不少,把一个院子堵得满满的,都想看看游家班的第一次出活预演。大庄叔也来了,父亲还单独给了他一条独凳子和一碗浓茶。大庄叔一脸的笑,说真没想到这唢呐班的当家人会是天鸣这崽儿,平时十棍子敲不出一个屁,吹起唢呐来还叫喳喳的呢!当年你爹说你能吹上百鸟朝凤老子还不相信呢,看来你游家真的是祖坟上冒青烟了。\\

几个师兄话不多,一直笑,父亲给每个人都倒了一碗烧酒,还不停的催促说喝啊喝啊润润嗓子啊!\\

水庄的夜晚好多年没有这样热闹了。四支唢呐呜呜啦啦的吼。奏完一曲丧调,人群里有人喊说天鸣整一曲百鸟朝凤给大家听听。我说那不行,师傅交代过的,这曲子是不能乱吹的。人群又起来一阵轰,老庄叔把凳子往我面前挪了挪,说就整一段,给大伙洗洗耳朵,这曲子当年肖大老师走的时候我听焦三爷整过一回,那阵势真他奶奶的不得了,能把人的骨头都给吹酥了。我还是摇头,父亲站在我身后对大家说今天就到这儿吧,以后机会多的是,天鸣保证给大家吹。老庄叔看见父亲发了话,也站起来说对对对,不依规矩不成,以后听的时间还多,散了吧都。\\

人群散了去,我对几个师兄说,这是游家班第一次接活,不能砸了,再走几遍吧。\\

远远的就看见了长生,他头上顶着一块雪白的孝布站在院子边等我们。看我们过来,长生给每个人派了一支烟。自己也啜上一支。我说老人家什么时候走的?长生喷出一口烟,笑着说这个月都死三四次了,死去没多久又缓了过来,直到昨天早晨才算是死透。旁边一个老人干咳了两声,说长生,快行接师礼呀!接师礼就是磕头。长生回头看了看旁边的老人,说接什么卵师呀!天鸣和我啥关系?一起比过鸡鸡的。然后他回头看着我笑笑,我也笑笑。\\

我其实倒是很希望长生给我磕个头。长生比我大五岁,是个精灵货,个子也比我大,小时候放牛我没少挨他揍,揍了我还要我喊他爹,喊过他多少回爹我都忘了。我一直想着报仇的,慢慢长大了,懂事了,报仇这个事情也就丢到一边了。今天本来是个机会,可长生还是显示着他一贯的与众不同。算起来,长生算是水庄第一个穿夹克和牛仔裤的人,这几年水庄人都前仆后继的把庇护了自己几千年的土墙房推到了,于是水庄出现了一排一排的镶着白晃晃瓷砖的砖墙房。水生看准了这个变化,拉上一群人在水庄的河滩上搞了一个砖厂。现在水庄好多人都不叫他长生了,叫他毛老板。\\

长生给游家班的待遇充分展示了他毛老板这个称呼并非浪得虚名。一人一条香烟,比起那些一支一支扔散烟的人家户,这种一次性的大额支付确实让人快意,因为我从几个师兄接过香烟的眼神可以看出,他们像打了一辈子小鱼小虾的渔民,今天忽然就网起来了一头海豹。\\

然后,你就可以看见我的几个师兄在吹奏的时候是多么的卖力,我真担心他们用力过猛会震破手里的唢呐。特别是长生打我们旁边经过的时候,我大师兄高高坟起的腮帮子像极了他妻子怀胎十月时的大肚皮。\\

除了香烟,毛老板的慷慨还体现在很多细节上,比如润嗓酒,是瓶装的老窖;再比如乐师饭,居然有虾。那玩意通体透红中规中矩的趴在盘子里,连我都看得傻了,虾我听说过的,是水里的东西,我们无双镇好多水,可我们无双镇的水里没有虾,只有一汪一汪淡绿的水草。长生最大的慷慨还不是这些,而是看见我们卖力的吹奏时,他就会过来先给每个人递上一支烟,说别太当回事了,随便吹吹就他妈结了。\\

走的那天长生没有送我们,而是每人递给我们一把钱。大师兄说了,这是他吹唢呐以来领到的最多一回钱,二师兄在一边也说,钱是最多的一次,可吹得是最轻松的一次。\\

我捏着一把钱站在水庄的木桥上,木木的看着一庄子正起来的炊烟。\\
			\fancyhead[LO]{{\scriptsize 【百鸟朝凤】第十二章}} %奇數頁眉的左邊
\fancyhead[RO]{\thepage} %奇數頁眉的右邊
\fancyhead[LE]{\thepage} %偶數頁眉的左邊
\fancyhead[RE]{{\scriptsize 【百鸟朝凤】第十二章}} %偶數頁眉的右邊
\fancyfoot[LE,RO]{}
\fancyfoot[LO,CE]{}
\fancyfoot[CO,RE]{}
\chapter*{十二}
\addcontentsline{toc}{chapter}{\hspace{11mm}第十二章}
%\thispagestyle{empty}
稻谷弯腰了,我去看了一回师傅。\\

又见到土庄的秋天了,一马平川的黄一直向天边延伸。\\

师傅刚下地回来。他好像更黑了,也更瘦了,裤管高高的卷起,赤着脚,脚板有韵律的扑打着地面,地面就起来一汪浅浅的尘雾。走到我的面前,他把手里的锄头往地上一拄,下巴挂在锄把的顶端,看着我笑笑,就伸出沾满泥土的手来摸我的脑袋。\\

“看你那双爪爪哟!”师娘嗔怪师傅。师娘也赤着脚,裤管也高高的卷起,正从屋子里往外搬凳子。\\

我把从水庄带来的东西拣出来放到院子里的木桌上。有师傅喜欢的旱烟叶子,烟叶是我到金庄出活时给买的,师傅说过无双镇最好的旱烟叶在金庄;还有腊肉,腊肉是我父亲烘的,颜色和肉质都好,带给师傅的是猪屁股那一段,在乡村人眼里,猪屁股是猪身上最珍贵的部分;此外还有母亲让我捎给师娘的碎花布,让师娘做件秋衣。\\

“来就来,还叮叮当当的带这样一大堆。”师娘总是要客气一番的。\\

我和师傅坐在院子里,这时候夕阳上来了,水庄就晃眼得紧。远处的金黄在晚风中奔腾翻滚,我都看得呆了。师傅指着远处对我说:“看那片,是我的,那谷子,鼓丁饱绽的。”我说我知道的,师傅就哈哈的笑说对对,你在的那阵子下过地的嘛。\\

我给师傅装了一锅刚带来的烟叶,师傅吸了一口,再吸一口,说没买准,金庄最好的烟叶在高昌山下,那片地种出来的烟叶才是最地道的,这烟叶儿不是高昌山下的。\\

“要吃人家饭,最后还要拉屎在人家饭盆里。”一旁剥蒜的师娘给我主持公道。\\

“前几天你二师兄来过一趟,说你们那边乐师钱出得很阔呢!”师傅往地上啐了一口烟痰说。\\

“不多的,就是有钱的那几家大方些!”\\

“人心不足蛇吞象啊!”\\

晚饭时辰,师傅搬出来一土壶烧酒。\\

十年了差不多,师傅一脸兴奋的说,火庄陈家酒坊的,那年给陈家老爷出活的时候到他酒房子里接的,没掺一滴水。\\

师傅在饭桌上照例没话,低着头呼啦啦的吃,间或端着盛酒的碗对我扬扬,这时候我也端起酒碗对着他扬扬,然后就听见烧酒在牙缝里流淌的声音。\\

我在土庄整整呆了三年,没见过师傅喝过一滴酒。其实师傅是有些酒量的,三碗青幽幽的烧酒倒下去,师傅的脸就有了猪肝的颜色。两个眼睛也格外的亮。\\

最让我惊奇的是那天师傅喝完酒后在饭桌上的话,那个多哟!比我在木庄听他说了三年的话还多。那天师傅说一些话让我印象深刻,因为师傅在说这些话的时候就像一只老狼,两手撑着桌面,脸向我这边倾斜着,眼睛里则是血红的光芒。他说唢呐匠眼睛不要只盯着那几张白花花的票子,要盯着手里那杆唢呐;还说唢呐不是吹给别人听的,是吹给自己听的;最后我的师傅焦三爷终于扛不过他珍藏了十年的陈家酒坊的高度烧酒,瘫倒在桌子上了,他倒下去的那一刻,两只眼睛直直的看着说:\\

“有时间去看看你的师弟蓝玉吧!”\\

第二天起来,师傅师娘都不见了,我知道他们下地了。这就是他们的生活,规律得和日出日落一样的。我还是有些晕,走到屋外,院子里木桌上的筲箕里有煮熟的洋芋,这算是给我的早饭了。那些日子就是这样的,我和蓝玉每天早上都要为拿到大个的洋芋争斗一番的。\\

站在山梁上,我回头看了看土庄,它好像老去了不少,那些山,那些水,都似乎泛黄了。\\
			\fancyhead[LO]{{\scriptsize 【百鸟朝凤】第十三章}} %奇數頁眉的左邊
\fancyhead[RO]{\thepage} %奇數頁眉的右邊
\fancyhead[LE]{\thepage} %偶數頁眉的左邊
\fancyhead[RE]{{\scriptsize 【百鸟朝凤】第十三章}} %偶數頁眉的右邊
\fancyfoot[LE,RO]{}
\fancyfoot[LO,CE]{}
\fancyfoot[CO,RE]{}
\chapter*{十三}
\addcontentsline{toc}{chapter}{\hspace{11mm}第十三章}
%\thispagestyle{empty}
马家大院看上去比五年前阔多了,楼房像个长个子的娃,几年光景就多出了三层。马家在木庄都习惯领跑了,还把后面的拉下一大截。老马家两层小平房起来了,木庄其他人家还在茅草屋子里忍饥挨饿,好不容易有了两层小平房,一瞧,老马家都五层了。木庄人总是在老马家屁股后面,怎么跑都跑不过。个中缘由除了老马脑筋好用以外,最主要的是老马有四个身强力壮的男娃子。几个娃出门早,据说中国的大城市都有他们的脚印。\\

可惜精打细算的老马还是耗不过病痛,六十不到的人,年前还背着手在木庄的石板路上检阅风景,年后就蹬腿了。四个儿子回来奔丧,每个人都有一辆小汽车,十六个轮子一码子停靠在木庄的石板街上,成了木庄人眼里一道稀有而复杂的风景。\\

游家班在马家大院里呈扇形散开。八台,也当然是八台。烟酒茶照例是不能少的,还有黄澄澄的糕点,放进嘴里又软又酥,上下颚一合拢,就化掉了。几个师兄都兴奋的交谈着,连平时话最少的三师兄都停不下口,他慌乱的说话,慌乱的把好吃的东西往嘴里扔,好几次该他的锣声响起了,他都还在为他那张嘴在奋斗。我有些火了,吼了他两声,没多久又听不见他的锣声了。\\

我忽然好惶恐。从我们进到马家大院起,好像就没有人关注过这几支呜呜啦啦的唢呐,我开始以为是大家不卖力,白了他们几眼,大家精神就抖擞不少,大师兄两个眼珠子都要给吹飞出来了,可对我们的处境仍没多少改善。人们依旧在院子里穿梭,小孩子依旧在院子里打闹,就是没人看我们。其间还有人碰倒了二师兄脚边的酒瓶子,白酒汩汩的往外流,那人像没看见一样,径直就去了。\\

我正要伸手去扶酒瓶子,眼睛就什么都看不见了。\\

猜猜,我是谁?\\

不用猜我就知道是他,我的师弟蓝玉。他的手粗壮了不少,声音也变得厚实了,嗓子也由男孩儿的蜕变成男人的了。\\

我的眼睛一下就潮湿了,其实我早看见他了的,混在来来往往的人群里,一件红色的外套招招摇摇。他的眼睛还不时的往游家班这边瞟,我没敢过去和蓝玉相认,不知道是没有相认的勇气还是其他的什么原因。\\

我的师弟蓝玉早就看见我们了,他一直没有过来,我想他不会过来了。\\

但现在他却蒙住了我的双眼,让我猜他是谁。\\

蓝玉惊慌的松开了手,惊讶的看着两只手掌中的潮湿,又抬起头看着我的眼睛,忽然他的眼泪也下来了。我和蓝玉面对面站着,我们差不多一样高,他嘴角的胡须比我的要茂盛,身子却比我瘦弱一些。\\

我忽然有了拥抱蓝玉的冲动,那种感觉热乎乎的。好多年前我们家有一条狗,黄毛,短耳朵,有一天突然不见了,刚不见的那几天还会想想它,慢慢的就忘掉了。大约过了两个月,那条狗出现在了我家院子里,一身泥污,一条腿还折了,两只眼睛弥漫着哀伤和委屈。那时候我也是这种热乎乎的感觉,跑过去抱着狗流了一回泪。\\

我看着蓝玉,蓝玉也看着我,我们谁都没有动。\\

师弟!我喊了一声。\\

蓝玉走过来,捶了我一拳。\\

“你有丢过狗的经历吗?”我问蓝玉。\\

“有,丢了整整十年!”蓝玉说。\\

几个师兄的唢呐一下嘹亮起来。\\

晚上蓝玉没有回家,一直陪着我们。喝酒、吹牛、抽烟。\\

下半夜,几个师兄都去睡觉了,人群也大多散去了。我和蓝玉坐在院子里,我把唢呐递给他,说来一调,蓝玉兴致勃勃的把唢呐接过去,苇哨刚送进嘴里又抽出来了。他把唢呐还给我,为难的笑笑说算了吧!好多年没吹了,调子都忘记了。我也笑笑说你那脑袋,十分钟就能把调调找回来。蓝玉拿来两个碗,倒了满满两海碗烧酒,我们就开始喝,一直喝到月亮下去,漫天的红霞上来,没有一点睡意。\\

这么多年来,蓝玉那晚说过的话我基本都记得。甚至他说话时的每一个表情,歪脑袋,大幅度的点头,掏耳朵等等这些细节都还在我的脑海里。比如他说当年离开土庄的时候,我一个人像条野狗一样,茫然的在田间小路上走,连死的心都有了。讲到这里他就把脑袋夸张的往下缩,等脑袋落到肩上了我才听见他喉咙里出来的那声浑浊的长叹;还有他说其实我不怪师傅,师傅让我回家是对的,要换了我,无双镇的唢呐班子早没了,我性子野,干啥都守不了多久,总会有些稀奇古怪的想法。讲到这里蓝玉的脖子忽然伸得老长,都快顶着头上那片红云了,他还呵呵的笑,笑完就猛灌下去一大口烧酒,脸也成了天边的颜色。\\

我的生命里有很多的变化,这些变化就像天气一样的让人琢磨不定,但每次变化之前又隐隐约约的看得见一些预兆。下雨之前是一定要乌云密布的,太阳带晕了,接踵而至的就是干旱,月亮带晕了,那说明接下来就该是一场连绵不绝的细雨时节了。那个木庄的夜晚,我和我的师弟蓝玉十年后相遇了,我们还有了一次酣畅淋漓的谈话,这场谈话让我隐隐的看到,也许,我的命运又到了拐角的地段了。\\
			\fancyhead[LO]{{\scriptsize 【百鸟朝凤】第十四章}} %奇數頁眉的左邊
\fancyhead[RO]{\thepage} %奇數頁眉的右邊
\fancyhead[LE]{\thepage} %偶數頁眉的左邊
\fancyhead[RE]{{\scriptsize 【百鸟朝凤】第十四章}} %偶數頁眉的右邊
\fancyfoot[LE,RO]{}
\fancyfoot[LO,CE]{}
\fancyfoot[CO,RE]{}
\chapter*{十四}
\addcontentsline{toc}{chapter}{\hspace{11mm}第十四章}
%\thispagestyle{empty}
老马的四个儿子比想象中的要阔得多。\\

老马要入土的前一天,一辆卡车开进了木庄。\\

老马的四个儿子都到庄头去列队迎接。车上下来几个人,和老马的大儿子聊了几句,老马的大儿子一挥手,庄上一群年轻人就钻进卡车里卸东西。\\

一开始那些东西还是零零碎碎的一堆,让人不知所以,东拼西凑的一倒腾,我身边的师弟蓝玉惊讶的说。\\

“妈的,这是一只乐队!”\\

游家班呈扇形站在马家大院里,我惊奇的发现,我的师兄们集体陷入了某种迷惘。他们的眼神笔直的指向同一个地方,嘴全都大大的裂着,像咫尺有了一个意想不到的惊人变化,也像遥远的天边出现了神奇的海市蜃楼,他们最后都笨拙的完成了复杂情感下简单的语言传递。\\

“到底是搞哪样卵哦!”\\

“这些狗日的是从哪里冒出来的!”\\

“哎呀!”\\

“哦哟!”\\

……\\

天黑下来,落雨了,一开始那雨细微得让人都觉察不到,落到手背上,脸上,有些淡淡的凉意,用手一抹,什么都没有。渐渐地雨就大起来了,雨滴也变大了,砸在裸露的皮肤上还有些疼痛。人群就开始往屋子里、屋檐下和灵堂里拱。\\

城里来的乐队还在雨中忙碌着。二师兄看着雨幕中的几只落汤鸡,说如何不下刀呢?我看了他一眼,他可能意识到这个愿望着实歹毒了些,又讪讪的矫正说下石头也行的。我也赞成下石头,所以我就没有说话了。但很快我发现,下石头恐怕对城里来的乐队也不会有什么实质性的伤害。老马的大儿子很快招呼人在院子里支起了一个帆布帐篷。还满脸堆笑给他们派烟,每个人的两边耳朵上堆满了他还在乐此不疲的派。\\

很快城里来的乐队就准备就绪了。他们的家伙比起乡村八台唢呐要复杂得多。从我见多识广的师弟的介绍我知道了左边那一排鼓叫架子鼓,站着的那个家伙手里抱着的像机枪一样的东西叫电吉他,案板样的是电子琴。最让我惊奇的是右边的络腮胡手里攥着的那支唢呐,他的唢呐好像更长更粗,腰身没有游家班使用的唢呐腰身好,大大咧咧的一粗到底。我就想这样粗的唢呐如何吹呢。\\

“砰!”,弹吉他的用手指拨出了一个清脆的音符。我现在还会在梦里听见那一声响,它的出现让我的梦总是充满了灰色的格调,每一次醒来,我都会双手枕着头想好久,那一声砰为什么在我的梦里不再是乐器的音符,而是极其怪异的幻化成了各式各样断裂发出的声响。譬如我正在建房,砰,房屋的大梁断裂了;或者我刚爬上高大的桑椹树,砰,大树一折为二;又或者我孤独的在一方悬崖下爬行,砰,悬崖张牙舞爪的迎面扑来。\\

……\\

我唯一可以肯定的是,在木庄马家大院的那个夜晚,仿佛从天而降的一声炸裂,搅乱了某种既定的秩序。每个人的心底都有一些莫名的东西在暗暗涌动着,像夜晚厨房木盆里那团搅和完毕的面团,正悄悄的发生着一些不为人知的变化。\\

就在那支吉他发出那声诡异的“砰”的声响的瞬间,我惊异的看见,马家大院所有一切都静止了。洒落的雨滴停在半空,在灯光下有五彩的颜色;洗菜的妇女扔进大木盆的萝卜也滞留在空中,在灯光下有耀眼的白;还有灵堂里的烛光,瞬间就收束成了一团实心的灼热,坚硬如冰;一个正在奔跑的孩子身体前倾,悬停在大门处,手臂一前一后伸展着,像一尊肉铸的雕塑。我张皇地在静止中游走,伸手去碰了一下半空里的水滴,它竟然炸裂成了一团水雾;我绷起指头弹向那团坚实的火焰,哗啦一声,散落了一桌的橘红。\\

我痛苦地捂着脑袋蹲在院子里。\\

“咚”,一声闷响。杂乱的噪音铺天盖地的向我袭来,震得我耳朵发麻。我站起来,发现一切都是活的,一切都在继续。雨一直在下,萝卜翻滚着跌进木盆,烛火在欢快的燃烧,孩子在院子里不停地奔跑。\\

“你刚才看见什么了吗?”我问蓝玉。\\

蓝玉看着我,说:“你是不是丢东西了?”,我摇头。“那你满院子找什么呢?”。蓝玉问。\\
			\fancyhead[LO]{{\scriptsize 【百鸟朝凤】第十五章}} %奇數頁眉的左邊
\fancyhead[RO]{\thepage} %奇數頁眉的右邊
\fancyhead[LE]{\thepage} %偶數頁眉的左邊
\fancyhead[RE]{{\scriptsize 【百鸟朝凤】第十五章}} %偶數頁眉的右邊
\fancyfoot[LE,RO]{}
\fancyfoot[LO,CE]{}
\fancyfoot[CO,RE]{}
\chapter*{十五}
\addcontentsline{toc}{chapter}{\hspace{11mm}第十五章}
%\thispagestyle{empty}
老马的葬礼新鲜而奇特。\\

乡村的葬礼不一定非得沉痛,但起码是严肃的。七十岁以上的老人去了那头,这叫喜丧,气氛是可以鼓噪些的。老马六十不到,他的葬礼是没有资格欢欣鼓舞的。可就在他入土的头一个晚上,马家大院出现了前所未有的喜气洋洋,那些奔丧迟到的人走进马家大院都一头雾水,以为走错了门,这里怎么看都像是老马家在娶媳妇,说在办丧事打死人家都不相信。\\

让老马由死而生的,是那支乐队。\\

先是几个人叮叮咚咚的乱敲一通,然后就唱开了。\\

鼓捣吉他的边弹边唱,唱的过程中还摇头晃脑的。他唱的是什么我听不懂,我的师弟蓝玉在一旁跟着哼哼,我问蓝玉他唱的是什么,蓝玉说是时下正流行的,只能跟着哼哼几句,整个儿的记不住,曲子叫什么名字也记不住了。\\

开始,木庄的乡亲们站在院子里,脸上都有了怒气。每个人都不很适应,脸上都有矜持的不满,一个上了年纪的阿婆把手里的一棵白菜狠狠的摔在地上,眼神离奇的愤怒,嘴里还咕咕囔囔,最后很沉痛的看了看灵堂。我知道他是在为死去的老马打抱不平呢!\\

渐渐的,大家的神色开始舒展开了,有一些年轻人还饶有兴致的围在乐队的周围,环抱双手,唱到自己熟悉的曲子时还情不自禁的跟着哼哼。\\

游家班站在马家大院的屋檐下,局促得像一群刚进门的小媳妇。我低头看了看手里的唢呐,才忽然想起来我们也是有活干的。\\

雨停了,空气清爽得不行,干干净净的。院子里为游家班准备的呈扇形排开的凳子还在。我们过去坐好。我看了看几个师兄。\\

“还吹啊?”一个师兄问。\\

“怎么不吹?又不是来舔死人干鸡巴的!”我对他的怯懦出离的愤怒。\\

我还拿起脚边的酒瓶子灌了一大口烧酒,悲壮得像即将奔赴战场的战士。\\

呜呜啦啦!呜呜啦啦!\\

平日嘹亮的唢呐声此刻却细弱游丝,我使劲瞪了几个师兄两大眼,大家会意,腮帮子高鼓,眼睛瞪得斗大。还是脆弱,那边的声响骄傲而高亢,这边的声音像临死之人哀婉的残音。一曲完毕,几个师兄都一脸的沮丧,大家你看看我,我看看你。\\

吹,往死里吹,吹死那群狗日的。师弟蓝玉在一边给大家打气。\\

我们吹得很卖力,在那边气势较弱的当口,就会有高亢的唢呐声从杂乱的声音缝隙里飙出去,那是被埋在泥土中的生命扒开生命出口时的激动人心,那是伸手不见五指的暗夜里划燃一根火柴后的欣喜若狂。\\

我们都很快意,那边的几只眼睛不停的往这边看,看得出,眼神里尽是鄙夷和不屑,甚至还有厌恶。\\

说实话,我对这群不速之客眼神里的内容是能够接受的,甚至他们就应该对我手里的这支唢呐感到厌恶才对。只是我没有想到,对我手里这支唢呐感到厌恶的不光是他们。\\

一个围在乐队边唱得最欢的一个年轻人不知什么时候站在我的面前。他斜着脑袋看着我,表情怪怪的,像是在瞻仰一具刚出土的千年干尸。我把唢呐从嘴里拔出来,吞了一口唾沫问:干什么?\\

你们吹一次能得多少钱?他说。\\

和你有关系吗?我答。\\

我付你双倍的钱,条件是你们不要再吹了。\\

我摇头说那不行。\\

没人喜欢听你们几根长鸡巴吹出来的声音。\\

那我也要吹。\\

这时候我的师弟站出来了,他过来推了年轻人一把。说柳三你干啥?叫柳三的说关你啥事?蓝玉说就他妈关我的事,咋了?\\

两个人就你来我往的开始推搡。本来已经有人过来劝住了的,柳三这个时候像想起了什么来,然后他说:“哦!我差点忘记了,你原来也是个吹破唢呐的!”说完还嘿嘿的干笑两声。\\

我看见蓝玉的拳头越过三个人的脑袋,奔着柳三的脑袋呼啸去了。一声闷响后,殷红的鲜血从柳三的鼻孔里奔涌而出。场面一下子就乱了,呼喊声,叫骂声,拳头打中某个部位后的空响,夹杂在癫狂的乐曲声中,活像一锅滚热的辣油。\\

第二天是蓝玉送我们离开的。我的师弟脑袋上缠着一块纱布,左边眼圈像块圆形的晒煤场。在我们身后远处的山梁上,送葬的队伍爬行在蜿蜒的山道上,那利箭一样的乐器声响充斥着木庄的每一个角落。\\

			\fancyhead[LO]{{\scriptsize 【百鸟朝凤】第十六章}} %奇數頁眉的左邊
\fancyhead[RO]{\thepage} %奇數頁眉的右邊
\fancyhead[LE]{\thepage} %偶數頁眉的左邊
\fancyhead[RE]{{\scriptsize 【百鸟朝凤】第十六章}} %偶數頁眉的右邊
\fancyfoot[LE,RO]{}
\fancyfoot[LO,CE]{}
\fancyfoot[CO,RE]{}
\chapter*{十六}
\addcontentsline{toc}{chapter}{\hspace{11mm}第十六章}
%\thispagestyle{empty}
水庄最近变化很多,有些是那种轮回式的变化,比如蒜薹又到了采摘的时候;有些变化则是新鲜的,让人鼓舞的,比如水庄通往县城的水泥路完工了,孩子们在新修完的水泥路上撒欢,大大小小的车辆赶趟儿似的往水庄跑,仿佛一夜之间,水庄就和县城抱成一团了。要知道,以前水庄人要去趟县城可不是那样容易的,不在坑坑洼洼的山路上颠簸五六个小时,你是看不见县城的。现在好了,去趟县城就像到邻居家串个门儿。\\

这个时候,我的父亲游本盛站在自家大蒜地里,满脸堆笑。在他眼里,像水庄有了水泥路这些新鲜事儿和他没有什么关系,他更关心的是他的大蒜地。今年的大蒜地倒是争气得紧,从冒芽儿开始就顺风顺水的,该采摘了,一根根在和风里炫耀着粗壮的身躯。父亲每天都要到大蒜地走一走,看一看,然后啜着纸烟蹲在土坎上,没有比这让他更满足的事情了。\\

父亲弓着腰在剥蒜薹,一阵风过去,我看见了他两扇瘦窄的屁股。我说歇歇吧。他直起腰,回过头,一脸的怒气:“歇歇?歇歇都能有饭吃老子早歇了!”我不说话了,还后悔刚才说出来的话。我想我最好是闭嘴,我说出来的每一句话,我的父亲都能找出让我难堪的理由。\\

可我发现,我不说话也不行,我不说话父亲也会把他的不满通过诸如眼神和动作传递给我。这一年来,父亲看我的眼神总是充满了疑问和警惕,我就像一只潜入他们家偷食的野猫,不幸正好被他发现了。我这只偷食的野猫只好把尾巴藏着掖着,生怕主人那天不高兴了一脚把你踹出门去。\\

初夏是水庄一年中最好的季节,这个时候的水庄可有生机了,天空清澈碧透,水面也清澈碧透,一庄子待收割的蒜薹也清澈碧透。最打动人的不管你走到哪里,每一个水庄人的脸上都带着笑。水庄人真的没有野心,一次理所当然的丰收就能把一个村庄变得天宽地阔。父亲不和我说话,埋下头继续采摘蒜薹。我直起腰,天空没有一丝云彩,一望无际的蒜地在阳光下像一幅油画。远远的,族中的三叔对着我远远的招手。三叔是我请去通知几个师兄弟出活的人。不知道从哪一天开始,无双镇的唢呐班子省掉了接师礼,连运送出活工具这些规矩都一并没了。我三步两跳的跑过去,先递给三叔一支烟,他撩起衣角擦了擦满脸的汗水,把烟点燃后对我说。\\

“都通知了,只有你大师兄同意来。”\\

“其他人呢?他们怎么说?”\\

“还能说啥?不是说忙就是这里那里不利索咯。”\\

三叔说完走了,走出老远了他好像又想起了什么,回头大声喊:\\

“对了,你二师兄说以后不要去叫他了。”\\

“为什么?”我问。\\

“说下个月要出门了。”\\

“去哪里?”\\

“不知道,大城市咯!”\\

我悻悻的回过头,就看见了父亲那张铁青的脸,他两手叉在腰际,眼睛直直的看着我。我低着头从他旁边走过去,他在后面冷冷的笑,笑完了说:\\

“都快孤家寡人了吧?看你以后还怎么吹?吹牛X还差不多。”\\

晚上我没有吃饭,躺在床上,定定的看着天花板。天花板上有一只蜘蛛倒悬着垂下来,一直垂到我的鼻尖处,我伸出手,让蜘蛛降落在我的手心里,它就顺着我的手臂往上爬,时左时右,我不知道哪里是它想去的地方,或者它压根就没有目的地,只是这样一直往前爬,再往前爬,什么时候爬累了,织个网,就算安家落户了;又抑或被天敌给吃掉了,无声无息的,谁又会去关心一只蜘蛛的未来呢!\\

仿佛一眨眼时间,我身边这个世界一下就变得陌生了,眼里的一切都没变,山还是那座山,河也还是那条河。可有些看不见的东西却不一样了,像水庄的那条河,看上去风平浪静的,可事实不是这样的,小时候下河游泳,一个猛子下去,才发现河底下暗潮汹涌。\\

直到父亲睡了,我才从屋子里出来。母亲重新把菜给我热了热。我吃饭时,母亲还是像小时候一样静静的坐在我的旁边,目不转睛的看着我,眼神里流淌着源源不竭的爱怜。\\

“后天是不是要出活?”母亲问。\\

我点点头。\\

“听你爹说几个师兄都不来?”\\

我又点点头。\\

“唉!”母亲长叹一声,然后她接着说:“天鸣,要不这唢呐不吹了!咱干点别的,凭咱这双手干啥不能活命啊!”\\

我放下碗,转过去对着母亲。\\

“我知道这个理,可当年拜师的时候我给师傅发过誓的,只要还有一口气,就要把这唢呐吹下去。”\\

“可你看,就你一个人也吹不来啊!”\\

“过两天我去找师傅。”\\
			\fancyhead[LO]{{\scriptsize 【百鸟朝凤】第十七章}} %奇數頁眉的左邊
\fancyhead[RO]{\thepage} %奇數頁眉的右邊
\fancyhead[LE]{\thepage} %偶數頁眉的左邊
\fancyhead[RE]{{\scriptsize 【百鸟朝凤】第十七章}} %偶數頁眉的右邊
\fancyfoot[LE,RO]{}
\fancyfoot[LO,CE]{}
\fancyfoot[CO,RE]{}
\chapter*{十七}
\addcontentsline{toc}{chapter}{\hspace{11mm}第十七章}
%\thispagestyle{empty}
我还没来得及去找师傅,师傅就先来找我了。\\

师傅一进院子就骂:“你个小狗日的游天鸣给老子出来。”\\

我出来看见师傅站在院子里,他的双脚沾满了泥,连衣服的下摆都有星星点点的泥点子。脸和我当初去拜师的时候一样黑,只是皱纹更多了,看见师傅老了一大截,我忽然上来了一些伤感。这个无双镇当年响当当的焦家班的掌门人,像入了冬的一棵老槐树,尽是令人沮丧的残败。最揪心的就是他一身灰布衣服了,还是老式样,对襟衫,几个地方都是补丁,要知道,现在无双镇像这样有补丁的衣服是不多见了,偶尔看见,不会有人说你艰苦朴素,下意识还会把你往穷人堆里推。\\

我喊了一声师傅。\\

“不要叫我师傅,我没有你这样的徒弟。”师傅往地上狠狠的啐了一口痰:“当初你是怎样说的,有口气就要把这活往下传,可这才过去多久?昨天就有人给我递话了,说无双镇的游家班散伙了,垮台了,有活也不接了,无双镇从今以后就没有唢呐匠了。”\\

我说师傅你先进屋,我们到屋里说。师傅一挥手:“进不起你的宝殿门,你现在哪里还瞧得上吹唢呐的?”。还是母亲出来,说焦师傅你先不要着急,进来说,天鸣正托人到处通知他的师兄弟们呢,这几天就要出活。母亲说话时不断对着我眨眼,我慌忙应和说对对对。师傅火气这才消了些。背着手走进屋,也不看我,只说,不给老子说个一二三,看老子不撕破你那张X嘴。\\

师傅坐下来,接过母亲倒来的茶,怒气冲冲的等我的解释。听完我的解释,师傅把茶碗往桌上狠狠一掼。\\

“我去找他们,几个狗日的还翻天了。”\\

师傅出了院门,看我还站在屋檐下,就吼:“傻了?游家班班主是我还是你?”,我哦了一声,才快步跟上去。\\

我跟在师傅身后,一路上他一句话都没有,但我能清晰的听见他大口大口喘气的声音。\\

二师兄对我和师傅的到来有些意外。当时二师兄正在打点行装,屋檐下,他正把一捆衣物狠命的往一个陈旧的蛇皮口袋里塞,口袋太小,装不下二师兄远涉的必须,就委屈地从口沿处往下撕裂,还发出吱吱的怪叫。二师兄骂了一句,抬起头就看见了师傅和我,他的嘴上下翕动着,是想说些什么,但从师傅的脸色他似乎已经明白了我们的来意,于是就什么也没有说。他放下手里的袋子,直起身子,从屋檐下的檐坎上下来,站在师傅面前,静悄悄的,没有一点声息。\\

师傅没有理二师兄,鼻子有了一声闷哼后,径直走到屋檐下,把口袋拎到院子里,把口袋里的东西一样一样的掏出来往院子里抛撒。师傅的这个动作持续了好长时间,我惊讶于这个看上去个儿不大的口袋居然有如此壮观的吞吐量,等师傅捋直了身子,院子里早成了花花绿绿的晾晒场。\\

师傅把干瘪的口袋踩在脚下,目光盯着二师兄,那眼神像水庄六月的日头,能把人烤晕过去的。\\

二师兄低着头,他一句话没有说,两个手交互搓揉着,这时候有几只麻雀从天而降,欢快的在院子里那些各式各样的衣物上跳跃。二师兄忽然松开了两只互握着的手,低头从师傅旁边走过去,蹲下身子把地上的衣物一件一件的拾起来搭在臂弯处,其间还拍拍打打的扇掉衣物上的灰尘。等他臂弯放不下后,他就慢慢蹲着移到师傅的脚边,伸出一只手扯师傅脚下的蛇皮口袋,师傅一动不动,师兄却执着地扯,力量也越来越大,最后我看见师傅的身体都开始摇晃起来。我站在一边看着这对奇特的师徒,他们就像在出演一出哑剧,每一个动作和眼神都极具深意,所有的表达都在你来我往的无声的动作中了。这时我的师傅伸出一只脚,狠狠的踹向了他二徒弟的面部,我看见二师兄猝然的往后倒了下去,像刚被掏空的蛇皮口袋。好半天,师兄才复苏的蛇一样从地上卷曲着爬起来,两道殷红从他的鼻孔蜿蜒而下,几乎穿越了整个面部。他没有完全站起来,依旧半蹲着,一步步挪到师傅的脚边,伸出一只手,固执的去扯师傅脚下的口袋。\\

这时候,我看见我的师傅面部完全变成了死灰色,五官也剧烈地痉挛着,像一锅煮烂的饺子。良久,他终于仰头长长的叹了一口气,叹气的感觉和水庄冬天的寒风一般,经过皮肤,直抵骨髓,能把人的那颗心都冻僵了。他终于移开了紧紧踩踏着口袋的脚,转身走了,走得很快,留给我一个颤抖不止的背影。\\

			\fancyhead[LO]{{\scriptsize 【百鸟朝凤】第十八章}} %奇數頁眉的左邊
\fancyhead[RO]{\thepage} %奇數頁眉的右邊
\fancyhead[LE]{\thepage} %偶數頁眉的左邊
\fancyhead[RE]{{\scriptsize 【百鸟朝凤】第十八章}} %偶數頁眉的右邊
\fancyfoot[LE,RO]{}
\fancyfoot[LO,CE]{}
\fancyfoot[CO,RE]{}
\chapter*{十八}
\addcontentsline{toc}{chapter}{\hspace{11mm}第十八章}
%\thispagestyle{empty}
道路弯弯拐拐,曲折迂回。乡间小路就是这样,站定一个点,极目远眺,道路伸出去没多远就倏然不见了。赶上去,才发现它又折向了某一个去处,再远眺,还是只能看到一根断面条。我们就在这样一条琢磨不定的道路上走着。最前面是我的师傅,中间两个,一个大师兄,一个蓝玉,我跟在最后头。\\

蓝玉自从离开土庄后,没有出过一次活。今天他能站在游家班的队伍里,我总有一种怪怪的感觉。我也不知道师傅是怎样说服蓝玉跟我们出这次活的。那天师傅离开二师兄家后,就直奔木庄去了。昨天晚上,蓝玉推开了我家的门。\\

师傅今天穿了一件新衣服,衣服上的折痕都还清晰可见。他走得很快,像一只老当益壮的野兔。蓝玉有意把步子放慢,很快我们的队伍就断裂成了两个块,前面是师傅和我的大师兄,后面是我和我的师弟蓝玉。\\

和我并排着的蓝玉忽然说:“师傅老了!”。我点点头,蓝玉又说:“这是我第一次正式出活,也是最后一次。”。我转过头看着蓝玉,不知道他想表达什么。过了半晌,蓝玉自言自语:“我答应师傅的,师傅也答应我的。”。\\

我的师弟蓝玉就是这样,总让我琢磨不透,说话也玄机重重。我说这话什么意思?蓝玉笑笑,没说话。我就低头自己想,等我抬起头的时候,幽静的山路上就看不见人影了。\\

在无双镇,和其他几个庄子比,火庄一直落在后面,房屋还多是拉拉杂杂的茅草屋,道路也没有其他几个庄子来得宽敞。但火庄人老实。无双镇人到集市上买鸡蛋,特别是买土鸡蛋,都要先问问是哪个庄子的。说是其他庄子的,人家不敢买。那是因为吃过亏的,问的时候一个劲给你打包票说真是土鸡蛋,买回去打开,一眼的翻白。只有火庄的土鸡蛋货真价实,黄澄澄的不说,价格也合理。今天出活的人家在火庄的西头,看上去家境一般,房屋翻了新,但屋子里却空闹闹的,只有些日常生活必须的物事,看来是屋子翻新耗光了家资。\\

家境虽是一般,可仍旧热闹。这和死去的人有莫大的关系,死者是火庄的老支书。德高望重的老支书躺在堂屋里,安静得像一只睡去的猫。师傅过去恭恭敬敬的上了三炷香。晚饭毕,我们一班人聚在堂屋里,我百无聊赖,把玩着手里的唢呐。师傅则拿出他那支老黄木杆的唢呐不停地擦拭。\\

大师兄把唢呐放进嘴里调音,咕咕唧唧的。师傅说你们都收起来,今天天鸣一个人吹。说完把擦拭好的唢呐递给我。\\

我出奇的惊讶,大师兄更惊讶,连嘴里的唢呐都忘记卸下来了。\\

“为什么?”我问。\\

“他去过朝鲜,剿过匪,带领金庄人修路被石头压断过四根肋骨。”师傅面无表情的说。\\

“百鸟朝凤!”蓝玉一扫慵懒的模样,绷直了问。\\

架势是摆出来了。灵堂前一张宽大的木靠椅,一群孝子俯首跪倒在我面前。所有的人都站在院子里,仰直了脖子往灵堂里看,连一直撒欢的那条老黄狗也规规矩矩的端坐在院子里。\\

我忽然有了一种神圣感,像一个身负特殊使命的斗士。那些眼光让人着迷,在每天来来往往,平淡无奇的生活中,你是看不到这种眼神的。它是那样的干净无邪,仿佛春雨过后山野里散发着的清新气息,又像是冬雪里萦绕在山巅的蒸腾雾霭。\\

师傅站了出来,对着灵堂鞠了三个躬,然后转过身对众人说:\\

“百鸟朝凤,上祖诸般授技之最,只传次代掌事,乃大哀之乐,非德高者弗能受也。”,我知道这几句是《百鸟朝凤》曲谱扉页上的几句话,下面的人是听不懂这几句话的,所以还是一贯的沉默。师傅接着说:“窦老支书我不多说了,他的所作所为金庄人都看在眼里,记在心里,如果无双镇还有人能受得起‘百鸟朝凤’这个曲子的,窦老支书算一个,今天,给窦老支书吹奏送行的,是游家班的班主游天鸣。”。师傅的诚恳让跪倒在我面前的一干人开始发出呜呜的低鸣声。\\

“大哀至圣,敬送亡人,起奏!”师傅高喊。\\

我把唢呐送到嘴里,忽然眼前一片漆黑。\\

直到今天我都活在那段悔恨中,我本可以从容的完成一个乡村乐师所能完成的最高使命,可以让后人提起这段近乎传奇的事件时还能提起我的名字,本可以让乐师这个职业在乡村实现最动人的谢幕演出,甚至可以用一种近于神圣的方式结束我的乐师生涯。可就在那一瞬间,这些可能统统没有了,我的行为让无双镇这个古老的职业用一种异常丑陋的形式完结掉了,连在湮没于时代变化中的最后一刻也未能保持它曾经拥有的尊严。所以,在记录下这段经历的时候,我面临着可怕的记忆煎熬,我感觉我心灵深处的一块被时间慢慢治愈的伤疤又被重新揭开,我清楚的看见它鲜血淋漓,继而是透骨的疼痛。\\

重新睁开眼,一双双焦渴的眼睛全都在看着我。我把唢呐从嘴里慢慢抽出来,站起来对我的师傅说:\\

“对不起大家,这个曲子我忘了!”\\

出人意料,师傅笑了,下面的人也笑了。下面的人还在笑,师傅却哭了,他蹲在地上放声痛哭,我、我的大师兄,还有我的师弟蓝玉,我们站在师傅的身边,谁都不说话。师傅哭了一阵,站起来对还跪在地上的孝子鞠了三个躬,说我们对不起窦老支书,也对不起各位孝子。\\

焦三爷吹一个不就行了!人群中有人建议。\\

师傅摆摆手,说我早就没有这个资格了,这个班子不是焦家班,只有游家班的班主才有这个资格。师傅说完转过身从我手里抢过那支唢呐,抬起膝盖,两手握着唢呐猛力一沉。\\

咔嚓!\\

师傅走了,他迅速消失在了金庄伸手不见五指的黑夜里。\\

蓝玉从地上把断成两截的唢呐拾起来,又看看我,说:“看来我这辈子是听不了百鸟朝凤了!”\\
			\fancyhead[LO]{{\scriptsize 【百鸟朝凤】第十九章}} %奇數頁眉的左邊
\fancyhead[RO]{\thepage} %奇數頁眉的右邊
\fancyhead[LE]{\thepage} %偶數頁眉的左邊
\fancyhead[RE]{{\scriptsize 【百鸟朝凤】第十九章}} %偶數頁眉的右邊
\fancyfoot[LE,RO]{}
\fancyfoot[LO,CE]{}
\fancyfoot[CO,RE]{}
\chapter*{十九}
\addcontentsline{toc}{chapter}{\hspace{11mm}第十九章}
%\thispagestyle{empty}
父亲对我的态度是越来越坏了,他看我什么都不顺眼,水缸空了,他骂我眼瞎了,连水缸没水了也看不见;我把水缸挑满了,他还骂我,说我除了挑水还能干啥?\\

父亲骂得对,我都二十六七岁的人了,还窝在家里。你看水庄和我一般年纪的人,娶妻的娶妻,生子的生子,还有大部分早就打点好行装,爬上开往县城、省城的客车走了,除了过年过节能看到他们一两眼,平时像我这样的年轻人村里几乎就看不到了。\\

自从游家班解散后,我再没吹过一天唢呐。\\

游家班的解散没有什么仪式,自自然然的,仿佛空气蒸发了一样,请也没人请了,吹就更没有人吹了。我和大师兄在无双镇的集市上遇到过一次,我们互相问候,还谈了今年庄稼的长势,最后还到无双镇的馆子里喝了一顿烧酒,可谁都没有说关于游家班的事情,哪怕一丁点也没有,像这个班子从来就没有存在过似的。\\

我二十八岁了,水庄的冬天又来了,水庄的冬天如今是越来越随便了,连场像模像样的雪都没有,最近两年更是蹬鼻子上脸,连点缀性的雾凇也看不见了,整个冬天都邋里邋遢,只知道一个劲的落冰雨,钉得人脸手生疼不说,还把一个水庄搅得稀泥遍地。\\

我现在最怕和父亲照面,不光是怕他骂我,是看着他一天天老去的模样我就会内疚。别人的儿子每年都能给家里寄回来数目不等的钱,我却只能坐在家里吃吃喝喝。母亲不像父亲那样责骂我,但她总是一声接着一声的叹气,叹气的声息像一块永远挤不干水的海绵,这比父亲的责骂让我更难受。就这样,我不得不在这个狭窄的空间里逃避。父亲每天吃完饭就去庄上看人打牌去了,他不参与,只是看,其实父亲很想坐上去摸一摸的,可他的口袋不允许。母亲则是每天都在灯下一直坐着忙,忙到实在疲乏得不行了才去睡觉。\\

我每个夜晚都早早爬到床上,却往往到了天亮还没有睡着。\\

今年从稻谷返青开始就没有落过一泼雨。本来都乌云密布了的,天地也陡然黑暗了,眼看一切前奏都摆足了,一庄子人都站在天地间等着瓢泼的雨水了。结果呢,稀稀拉拉的下来几滴,在地上留小几个濡湿的坑点,立马就云开雾绽了。反复几次,水庄人的希望和耐心像田里的稻谷一样,都干枯瘪壳了。\\

父亲的背越来越佝偻,像一张松垮垮的泥弓。父亲每天都守在他的稻田边,脸色和稻子一样枯黄。他的眼神散漫无力地在一坝子干瘪的稻浪上翻滚,跟着风的摆动,晃来荡去,软弱无力。就这样一直到黄昏,他才直起腰来,在一阵吱吱嘎嘎的骨头摩擦声中,开始把枯朽的身躯往自家屋子里搬运。\\

偶尔我会在院子里遇见他,他总是呆呆的看着我,没有了愤怒,也没有了讥讽,目光蛛丝一般的柔软,缠得我有些透不过气来。\\

我清楚地记得,那一季的稻谷最后全枯死在了田里。我站在水庄后面的山头,视野里是一片灼人的枯黄,那黄一直向天边延伸,这样的颜色真让我绝望。但水庄的游本盛更让我绝望。一张脸黄得肆无忌惮。肝癌晚期,我和母亲竭力要求把圈里的老牛卖掉给他治病,可游本盛说:算了,我就是田里的稻子了,再大的雨水也缓不过来了。\\

一个月来,父亲的身体在木床上越来越小。从医院回来,父亲就再没有离开过家里那张宽大的木床。木床是爷爷留下来的,父亲当年就在这张大床上降生,如今,他又即将在这张大床上死去,像完成了一个可笑的轮回。\\

早晨我把家里的老牛牵到水庄的河滩边吃了一些草。中午回家的时候,我居然看见父亲站在庄头,阳光把他捏成一小团,他把身体靠在土坎上,土坎上有茂密的青色,这样他就像一朵从草丛里长出来的黄色蘑菇。我远远就看见了他,惊讶过后眼泪就下来了。\\

我怕他看见我的眼泪,拭干了才走近他。他颤颤巍巍地过来,像刚学走路的小孩儿。拍了拍老牛的脖子,父亲说:“把它卖了吧!”,说完了居然下来了两滴眼泪。我明白了,父亲还不想死,他毕竟才五十出头,这样年纪的水庄人,都身强体健的穿梭于田间地头,还有使不完的劲,眼前的路还远得看不到头呢!“早该卖了,早卖早治的话,也不至于这样了。”我说。\\

牛卖掉那天,我在无双镇给父亲买了一双软底布鞋,我想过了,进城治病难免要走来走去的,软底布鞋穿上不硌脚,父亲全身只剩下骨头了,什么都该是软的才对。\\

晚上回来把鞋子递到父亲手里,他竟然从床上翘起来给了我一耳光。\\

“谁叫你费这钱?狗日的就是手散!”\\

耳光一点不响亮,听见的反而是骨头炸裂的声音。\\

我没有说话,把父亲扶下躺好,他两个鼻孔和嘴都大口大口的呼着浊气。喘了好一阵子,父亲终于平静了下来,他先是长长的吁了一口气,艰难地把身体侧过来对着我说:“天鸣,我听说金庄的唢呐也吹起来了。”我点点头。\\

其实不光金庄,无双镇除了水庄其他几个庄子都有唢呐了。也不知道是从哪天开始,城里下来的乐队就从无双镇消失了,就像停留在河滩上的一团雾,一阵风过,就无影无踪了。乐队一消失,唢呐声就嘹亮起来了。\\

“把游家班捏拢来。”父亲说:“无双镇不能没有唢呐。”\\

“有哩!除了水庄其他庄子都有了。”我说。\\

“日娘,那叫啥子唢呐哟!”父亲面色灰土,喘气声也大了许多,额头上还有汗出来。\\

我呆坐在床边,不说话。父亲的喉咙里有咕咕的声音,像地下的暗河,涌动着不为人知的秘密。良久,我听见父亲发出呜呜的哭声,哭声尖而细,如同一柄锋利的尖刀,划过屋子里凝滞的气息,继而如撕裂的布匹,陡然凄厉得紧。\\

此刻我才发现,我的父亲,水庄的游本盛心里一直都希望他的儿子吹唢呐的。在游家班解散后,父亲那种看似寡毒的蔑视、打击、嘲讽,其实是伤心欲绝,是理想被终结后的破罐子破摔。我又想起了父亲带着我拜师的那个湿漉漉的日子,还有他跌倒后爬起来脸上那道殷红的血痕。\\

我伸出手,摸到了父亲夸张的锁骨,它坚硬地硌着我的手,更硌着我的心。\\

我试试吧。我说,声音很小,但父亲还是听见了。\\

尽管屋子里光线很暗,但我还是看见了父亲眼里的亮光,我的话像一根划燃的火柴,腾地点亮了父亲这盏即将油尽的枯灯。\\

“我就知道,你狗日的还想着唢呐。”笑容在父亲枯瘦狭窄的面容上铺开,氲成一团凄苦和苍凉。“知道我为什么卖牛吗?”父亲纯真得像一个孩子:“我那是给游家班买家什用的,我想过了,啥子鼓啊锣啊,都老旧了,该换新的了。”接下来就是一阵咳嗽,父亲太兴奋了,又呼啸了一阵才平静了下来,父亲又说:“我死了,给我吹个四台就行了。”\\

“我给你吹‘百鸟朝凤’。”我说。\\

父亲摆了摆枯瘦的手,半天才说:“使不得,我不配!”\\
			\fancyhead[LO]{{\scriptsize 【百鸟朝凤】第二十章}} %奇數頁眉的左邊
\fancyhead[RO]{\thepage} %奇數頁眉的右邊
\fancyhead[LE]{\thepage} %偶數頁眉的左邊
\fancyhead[RE]{{\scriptsize 【百鸟朝凤】第二十章}} %偶數頁眉的右邊
\fancyfoot[LE,RO]{}
\fancyfoot[LO,CE]{}
\fancyfoot[CO,RE]{}
\chapter*{二十}
\addcontentsline{toc}{chapter}{\hspace{11mm}第二十章}
%\thispagestyle{empty}
父亲病得越来越重了,话也越来越少了,开始是整夜整夜睡不着,后来是睡过去就醒不来。母亲总是守在父亲旁边,隔一阵子就看一回,探探他的鼻孔,摸摸他的额头,怕他睡过去就永远醒不来了。\\

我则在无双镇几个庄子之间昼夜奔走。\\

在无双镇生活了这么多年,我第一次在如此密集的时间里听田间的蛙鸣,山谷的鸟叫。夜晚,我一个人在狭窄的山间小路上行走,天边的一弯冷月漠然地朗照,大地如逝者的巴掌一样冰凉,裹紧衣服才发现,寒冷正不可抗拒地到来。脑子里又浮现出父亲孤独无助的眼神和日渐枯槁的面孔。我怕他等不到我把游家班捏拢他就走了,那样我的父亲就听不到唢呐声了。对于水庄的游本盛来说,没有唢呐的葬礼是不可想象的。\\

无双镇被我的双脚丈量完毕了,我仍像一个出海旬月却两手空空的渔人。我的师兄师弟们,此刻正在繁华而遥远的城市挥汗如雨,他们就像商量好了一般,整整齐齐地离开了生养他们的土地。\\

大师兄还在。他不去城市不是他不想去,而是一次意外让他拥有了一条断腿,而这条腿也成了他和城市之间永远的屏障。我把香烟递到他手上的时候,他还满含神往的给我讲述了师弟蓝玉去年来看他时的情景。“小屁股,抽的烟一支顶你这个一盒,你还别不服气,那烟抽起来就是他奶奶的顺口。”“看来,城里这钱还真他奶奶的好挣。”\\

听完我的来意,大师兄惊奇地盯着我,然后他说,你见过两个人吹的唢呐吗?旧时一般穷苦人家都四台,你想造个两台?埋条死狗还差不多。我说不是埋死狗,是埋我的父亲。大师兄脸上才起来了一层歉意,他大大的吸了一口烟,说去火庄吧,那里起来了好几个班子,听说场面很大,都有十六台了。奶奶的,十六个人一起吹唢呐,怕死人都能给吹活呢!\\

我走了好远,大师兄站在山梁上喊:“去看看吧!如今无双镇的唢呐都成他们的天下了。”\\

我到火庄正赶上这里的唢呐班子出活。\\

确实很让人惊讶。\\

十六个唢呐匠占据了整个院坝,连死者这个理所当然的主角都被逼到了狭窄的一隅。一排条桌浩浩荡荡的拉出了雄壮的架势。条桌上的茶盘里有香烟和瓜子。瓶装的润嗓酒也精神抖擞的站成一列。唢呐匠一色暗红色西服,大宽领,下摆还卷了圆边,一个个像即将走入洞房的新郎。条桌顶头是一件银灰色西服,还扎了根猩红的领带,胸前挂了一块亮闪闪的牌子。看样子,他就该是班主了。\\

最显眼的还不是班主,而是他面前盘子里的一沓钞票,百元面额的,摞出了一道耀眼的风景。“起!”班主发声,接下来就是一场宏大的鼓噪,唢呐太多了,在步调上很难达成一致,于是就出现了群鸟出林的景象,呼啦一片,沸沸扬扬,让人感到一些惶然的惊惧。我甚至满含恶意地发现,有两个年轻的唢呐匠腮帮子从头到尾都瘪着,要知道,这个样子是吹不响唢呐的。这是我见过场面最大的唢呐班子,也是我听过的最难听的唢呐声。我的大师兄说得不对,十六台的唢呐不能把死人吹活,但没准会把活人吹死。\\

我回到家,父亲已经不能说话了,我凑到他的耳朵边说:给你请个火庄的八台吧!父亲忽然睁大眼睛,脑袋拼命地摆动,喉咙里咕咕地响着。我知道,他不要火庄的唢呐,他说过的,火庄那不是真正的唢呐。\\

水庄的游本盛是水庄的河湾开始结冰时离开这个世界的,他静悄悄的就走了,头天晚上还挣扎着吃了半碗稀饭,第二天一早,发现身体都已经变得冰凉了。他死的时候瘦的像个刚出生的婴儿,把一张木床映衬得硕大无比。我把卖牛的钱将父亲安葬了。他的葬礼冷清得如同这个季节,唢呐声自然是没有的,倒是北风从头到尾都在不停地呼啸。\\

那个黄昏,我守在父亲的坟边。从此以后,水庄再没有游本盛了,他和深秋的落叶一起,凄凄惶惶地飘落、腐烂。我在夕阳里想了好久,都没有想起我到底给了我的父亲什么。而我对于他,只有一个又一个的失望。我的唢呐没了,游家班也没了,直到死去,他连一台送葬的唢呐都没有。\\

好久没有看到水庄这样的黄昏了,在我的印象中,水庄的黄昏总是转瞬即逝的,刚发现它,它就一头栽进黑夜。其实心细一点观察,水庄的黄昏是很好看的,落日静止在山头,草的须穗摩挲着它的脸面,有了麻酥酥的微痒;风翻滚着从山梁上滑下来,撩开大山的衣襟,露出暗红的裸背。大地,就在这样简单的组合中,变得古老而温暖。\\

我从怀里抽出唢呐,对着太阳的方向,铜碗里就有了满满的一窝儿夕阳。\\

曲子黏稠地淌出来,打了几个旋儿,跌落在新鲜的坟堆上,它们顺着泥土的缝隙,渗透进了冰冷的黄土。我知道,我的父亲能听见他儿子的唢呐声。从我学艺到他离开这个世界,他还没有听我吹奏过这曲“百鸟朝凤”。开始唢呐声还高亢嘹亮着,渐渐地就低沉了,泪水把曲子染得潮湿而悲伤,低沉婉回的曲子中,我看到父亲站在我的面前,他的眼神如阳光一般温暖,那些已经一去不复返的日子,在朦胧的视线里逐渐清晰起来。\\

起风了,唢呐声愈发凌乱,褪掉了肃穆的色彩,却有了更多的凄凉。我的喉咙被一大团悲伤嗝得生疼,唢呐终于哭了,先是呜咽,继而大恸。连绵不绝的群山,被一杆唢呐搅得撕心裂肺。\\
			\fancyhead[LO]{{\scriptsize 【百鸟朝凤】第二十一章}} %奇數頁眉的左邊
\fancyhead[RO]{\thepage} %奇數頁眉的右邊
\fancyhead[LE]{\thepage} %偶數頁眉的左邊
\fancyhead[RE]{{\scriptsize 【百鸟朝凤】第二十一章}} %偶數頁眉的右邊
\fancyfoot[LE,RO]{}
\fancyfoot[LO,CE]{}
\fancyfoot[CO,RE]{}
\chapter*{二十一}
\addcontentsline{toc}{chapter}{\hspace{11mm}第二十一章}
%\thispagestyle{empty}
今年第一场雪刚过,村长领着几个人到了我家。\\

我站在院子里,村长拍着我的肩膀说:这就是无双镇游家唢呐班子的班主。\\

很年轻啊!一个戴着眼镜的中年人说。\\

是这样的,他说,我们是省里面派下来挖掘和收集民间民俗文化的。\\

我说你就说找我什么事情吧。\\

戴眼镜的说我们想听一听你的唢呐班子吹一场完整的唢呐。我说游家班已经没有了,火庄有,你们去看看吧。那人笑笑,说我们刚从那里过来,怎么说呢!他干咳了一声:“我们听过了,他们那个严格说起来还不能算纯正的唢呐。”\\

你看?他递给我一支烟说。\\

我说怕不行了,我的师兄弟们全进城了。\\

这时候站出来一个年轻一些的,村长赶忙出来介绍说这是县里来的宣传部长。年轻的部长很豪迈的一挥手,说去把他们都叫回来,费用我们来出。他的语调和姿势让我热血一下涌了上来,我仿佛看到了我的游家班整齐出场的场景,那是多么让人神往的一个场面啊!七八个人一字排开,悠悠扬扬的吹上一场。我梦里经常出现这样的场景。\\

我说好。\\

冬天快过去了,我接到了蓝玉的一封信,他在信上说,他已经在省城站住了,拥有了自己的纸箱厂。我决定去省城把我的师兄弟们找回来,我要把我的游家班重新捏拢来,我要无双镇有最纯正的唢呐。\\

省城真大,走下客车我有了溺水的感觉。\\

根据地址东寻西找了一整天,我终于在一个胡同里找到了蓝玉的纸箱厂。\\

推开铁门,一个守门的老头在门里一间昏暗的屋子里看报纸。\\

请问蓝玉在吗?\\

“蓝厂长出门去了。”老头答:“你找他什么事?”老头抬起头问。\\

“师傅!!”\\

……\\

那天夜里,蓝玉把在这个城市的师兄弟们都通知到了一处,还请大家去了一家金碧辉煌的饭店吃了一顿饭。师傅还是老样子,饭桌上一句话没有,沉默寡言的吃。我说明来意,师傅的眼里掠过一抹亮光,然后他抹了抹嘴,说上面都重视了,这是好事啊!\\

好多年没摸那玩意了。二师兄感叹。\\

我从包裹里取出来一支唢呐递给二师兄,说试试?二师兄把唢呐接过去,端平,刚把哨管放进嘴里,他的眼神暮然黯淡,然后他举起右手,我看见我在木材厂打工的二师兄中指齐根没有了。\\

让锯木机吃掉了。他说,这辈子都吹不了唢呐了。\\

在水泥厂负责卸货的四师兄接过唢呐,说我试试,他架子还在,像模像样的摆好姿势,唢呐在他嘴里没有想象和期待中的嘹亮,只闷哼了一声,就痛苦地停滞了。他抽出唢呐吐出一口浓痰,我看见地上的浓痰有水泥一样的颜色。\\

别回去了,留下来吧!蓝玉看着我说。我喝了一大口酒,说我要回去,我一定要回去。看着桌子上的师兄师弟们,我忍不住哭了,师傅也哭了。\\

我知道,唢呐已经彻底离我而去了,这个在我的生命里曾经如此崇高和诗意的东西,如同伤口里奔涌而出的热血,现在,它终于流完了,淌干了。\\

夜晚,师傅还有师兄弟们送我去火车站。我们沿着城市冰冷的道路一直走,没有人说话,只有往来的车辆拉出让人心悸的呼啸,偶尔有行人经过,都一色的低着头,把脑袋往前伸,急冲冲的扑进城市迷离慌乱的大街小巷。\\

在车站外一块巨大的广告牌下,一个衣衫褴褛的老乞丐正举着唢呐呜呜地吹,唢呐声在闪烁的夜色里凄凉高远。\\

这是一曲纯正的“百鸟朝凤”。\\

\begin{center}
	- 全书完 -
\end{center}
		\backmatter
		{\color{TEXTColor}
			\begin{figure}[ht]
				\begin{center}
					\includepdf[width=\paperwidth,height=\paperheight]{Backmatter.jpeg}
				\end{center}
			\end{figure}
\end{document}